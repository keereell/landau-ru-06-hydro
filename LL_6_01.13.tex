\section{Внутренние волны в несжимаемой жидкости}
\label{sec:p13}

Своеобразные гравитационные волны могут распространяться внутри несжимаемой
жидкости. Их происхождение связано с вызываемой наличием поля тяжести
неоднородностью жидкости: ее давление (а с ним и энтропия $s$) непременно будет
меняться с высотой; поэтому всякое смещение какого-либо участка жидкости по
высоте приведет к нарушению механического равновесия, а потому к возникновению
колебательного движения. Действительно, ввиду адиабатичности движения этот
участок принесет с собой в новое место свое значение энтропии $s$, отличное от
ее равновесного значения в этом месте.

Мы будем ниже предполагать, что длина распространяющейся в жидкости волны мала
по сравнению с расстояниями, на которых поле тяжести вызывает заметное изменение
плотности\footnote{Градиент плотности связан с градиентом давления равнством
\[
\nabla p = \ddp{p}{\rho}{s}\nabla \rho = c^2 \nabla \rho,
\]
где $с$ - скорость звука в жидкости. Поэтому из гидростатического уравнения
$\nabla p =\rho \vect g$ имеем $\nabla \rho = (\rho/c^2)\vect g$.Отсюда видно,
что существенное изменение плотности в поле тяжести происходит на расстояниях
$l \approx c^2/g$. Для воздуха $l \approx 10$ км, для воды $l \approx 200$ км.
}. Самую жидкость мы будем при этом рассматривать как несжимаемую. Это
значит, что можно пренебречь изменением ее плотности, связанным с изменением
давления в волне. Изменением же плотности, связанным с тепловым расширением,
отнюдь нельзя пренебречь, так как именно оно определяет собой все явление.

Выпишем систему гидродинамических уравнений для рассматриваемого движения. Будем
отмечать значения величин в состоянии механического равновесия индексом нуль, а
малые отклонения от этих значений в волне — штрихом. Тогда уравнение сохранения
энтропии $s = s_0 + s'$ напишется с точностью до величии первого порядка малости
в виде
\begin{equation}
   \label{eq:13.1}
   \pd{s'}{t} + \vv\nabla s_0 = 0,
\end{equation}
где $s_0$, как и равновесные значения других величин, является заданной функцией
вертикальной координаты $z$.

Далее, в уравнении Эйлера снова пренебрегаем (в силу малости колебаний) членом
$\vnv$; учитывая также, что равновесное распределение давления определяется
уравнением $\nabla p_0 = \rho_0 \vect g$, получим с той же точностью
\[
   \pd{\vv}{t} = - \frac{\nabla p}{\rho} + \vect g =
   - \frac{\nabla p'}{\rho_0} + \frac{\nabla p_0}{\rho_0^2}\rho' .
\]

Поскольку согласно сказанному выше изменение плотности связано только с
изменением энтропии, но не давления, то можно написать:
\[
   \rho' = \ddp{\rho_0}{s_0}{p} s',
\]
и мы получим уравнение Эйлера в виде
\begin{equation}
   \label{eq:13.2}
   \pd{\vv}{t} = \frac{ \vect g}{\rho_0}\ddp{\rho_0}{s_0}{p}s' - \nabla \frac{p'}{\rho_0}.
\end{equation}
$\rho_0$ можно ввести под знак градиента, так как изменением равновесной
плотности на расстояних порядка длины волны мы, согласно сказанному выше, все
равно пренебрегаем. По этой же причине можно считать плотность постоянной и в
уравнении непрерывности, которое сводится при этом к
\begin{equation}
   \label{eq:13.3}
    \Div \vv = 0.
\end{equation}

Будем искать решение системы уравнений (\ref{eq:13.1} - \ref{eq:13.3}) в виде плоской волны:
\[
   \vv = \const \cdot e^{i(\vect{kr}-\omega t)}
\]
и аналогично для $s'$ и $p'$. Подстановка в уравнение непрерывности (\ref{eq:13.3}) дает

\begin{equation}
   \label{eq:13.4}
   \vect{vk}=0,
\end{equation}
т. е. скорость жидкости везде перпендикулярна к волновому вектору (поперечная
волна). Уравнения же (\ref{eq:13.1}) и (\ref{eq:13.2}) дают
\[
   i \omega s' = \vv \nabla s_0, \;
   -i \omega \vv = \frac1{\rho_0}\ddp{\rho_0}{s_0}{p} s' \vect g - \frac{i\vect k}{\rho_0}p'.
\]
Условие $\vect{kv} = 0$, примененное ко второму из этих равенств, приводит к
соотношению
\[
   ik^2p' = \ddp{\rho_0}{s_0}{p} s' (\vect{gk}),
\]
и исключая затем из обоих уравнений $\vv$ и $s'$, получим искомый закон
дисперсии — соотношение между частотой и волновым вектором:
\begin{equation}
   \label{eq:13.5}
   \omega^2 = \omega^2_0 \sin^2\theta,
\end{equation}
где обозначено
\begin{equation}
   \label{eq:13.6}
   \omega^2_0 = - \frac{g}{\rho}\ddp{\rho}{s}{p}\D{s}{z}.
\end{equation}
Мы опускаем здесь и ниже индекс нуль у равновесных значений термодинамических
величин; ось $z$ направлена вертикально вверх, а $\theta$ есть угол между осью
$z$ и направлением $\vect k$. Положительность выражения (\ref{eq:13.6}) обеспечивается
условием устойчивости равновесного распределения $s(z)$ (условием отсутствия
конвекции, см. \S4).

Мы видим, что частота оказывается зависящей только от направления волневого
вектора, но не от его величины. При $\theta = 0,\pi$, л получается $\omega = 0$;
это означает, что волны рассматриваемого типа с волновым вектором, направленным
вертикально, вообще невозможны.

Если жидкость находится не только в механическом, но и в полном
термодинамическом равновесии, то ее температура постоянна и можно написать:
\[
   \D{s}{z} = \ddp{s}{p}{\tau}\D{p}{z} = - \rho g \ddp{s}{p}{T}.
\]
Наконец, воспользовавшись известными термодинамическими соотношениями

\[
   \ddp{s}{p}{T} = \frac1{\rho^2}\ddp{\rho}{T}{p},\;
   \ddp{\rho}{s}{p} = \frac{T}{c_p}\ddp{\rho}{T}{p}
\]
($c_p$ — теплоемкость единицы массы жидкости), получим:
\begin{equation}
   \label{eq:13.7}
   \omega_0 = \sqrt{\frac{T}{c_p}}\frac{g}{\rho}\left\vert \ddp{\rho}{T}{p} \right\vert .
\end{equation}
В частности, для термодинамически идеального газа эта формула дает

\begin{equation}
   \label{eq:13.8}
   \omega_0 = \frac{g}{\sqrt{c_p T}}.
\end{equation}

Зависимость частоты от направления волнового вектора приводит к тому, что
скорость распространения волны $\vect U = \partial \omega/\partial k$ не
совпадает по направлению с $\vect k$. Представив зависимость $\omega (\vect k)$
в виде
\[
   \omega = \omega_0\sqrt{1-\left( \frac{\vect{k\nu}}{k} \right)^2}
\]
($\nu$ — единичный вектор в направлении вертикально вверх) и произведя
дифференцирование, получим

\begin{equation}
   \label{eq:13.9}
   \vect U = - \frac{\omega^2_0}{\omega k} (\vect{n \nu})
   \lbrace \vect{\nu - (n\nu)n} \rbrace,
\end{equation}
где $\vect n = \vect{k}/k$. Эта скорость перпендикулярна к вектору $\vect k$, а
по величине равна
\[
   U = \frac{\omega_0}{k}\cos\theta.
\]
Ее проекция на вертикаль:
\[
   \vect{U\nu} = - \frac{\omega_0}{k}\cos\theta\sin\theta.
\]



