\section{Распространение возмущений в потоке сжимаемого газа}\label{sec:p82}


Когда скорость движения жидкости делается сравнимой со скоростью звука или превышает ее, на передний план выдвигаются эффекты, связанные с сжимаемостью жидкости.
С такого рода движениями приходится на практике иметь дело у газов. Поэтому о гидродинамике больших скоростей говорят обычно как о \emph{газодинамике} .

Прежде всего следует заметить, что в газодинамике практически всегда приходится иметь дело с очень большими значениями числа Рейнольдса. 
Действительно, кинематическая вязкость газа, как известно из кинетической теории газов, - порядка величины произведения длины свободного пробега молекул $l$ на их среднюю скорость теплового движения; последняя же совпадает по порядку величины со скоростью звука, так что $v \sim cl$.
Если характеристическая скорость газодинамической задачи - порядка величины скорости звука или больше, то число Рейнольдса $R \sim Lu/v \sim Lu/lc$, т. е. содержит заведомо очень большое отношение характеристических размеров $L$ к длине свободного пробега 
\footnote{Мы не рассматриваем вопроса о движении тел в очень разреженных газах, в которых длина пробега молекул сравнима с размерами тел. Этот вопрос по существу не является гидродинамической проблемой и должен рассматриваться с помощью кинетической теории газов.}.
Как всегда, при очень больших значениях $R$ вязкость оказывается не существенной для движения газа практически во всем пространстве, и в дальнейшем мы везде (за исключением лишь особо оговоренных мест) рассматриваем газ как идеальную (в гидродинамическом смысле слова) жидкость.

Движение газа имеет существенно различный характер в зависимости от того, является ли оно \emph{дозвуковым} или \emph{сверхзвуковым}, т. е. меньше или больше его скорость, чем скорость звука.
Одним из наиболее существенных принципиальных отличий сверхзвукового потока является возможность существования в нем так называемых ударных волн, свойства которых будут подробно рассмотрены в следующих параграфах.
Здесь же мы рассмотрим другую характерную особенность сверхзвукового движения, связанную со свойствами распространения в газе малых возмущений.


Если в каком-нибудь месте стационарно движущийся газ подвергается слабому возмущению, то влияние этого возмущения распространяется затем по газу со скоростью (относительно самого газа), равной скорости звука.
Скорость же распространения возмущения относительно неподвижной системы координат складывается из двух частей: во-первых, возмущение сносится потоком газа со скоростью $\V v$ и, во-вторых, распространяется относительно газа со скоростью $c$ в некотором направлении $\V n$. Рассмотрим для простоты однородный плоско-параллельный поток газа с постоянной скоростью $\V v$.
Пусть в некоторой (неподвижной в пространстве) точке $O$ газ подвергается малому возмущению.
Скорость $\V v + c \V n$ распространения исходящего из точки $O$ возмущения (относительно неподвижной системы координат) различна в зависимости от направления единичного вектора $\V n$.
Все возможные ее значения мы получим, отложив из точки $O$ вектор $\V v$, а из его конца, как из центра, построив сферу радиуса $c$; векторы, проведенные из $O$ в точки этой сферы, и определят возможные величины и направления скорости распространения возмущения.
Предположим сначала, что $v < c$. Тогда векторы $\V v + c \V n$ могут иметь любое направление в пространстве (рис. 50, а).
Другими словами, в дозвуковом потоке возмущение, исходящее из некоторой точки, распространяется в конце концов по всему газу.
Напротив, в сверхзвуковом потоке, $v > c$, направления векторов $\V v + c \V n$, как видно из рис. 50,6, могут лежать только внутри конуса $c$ вершиной в точке $O$, касающегося построенной из конца вектора $\V v$, как из центра, сферы.
Для угла раствора $2\alpha$ этого конуса имеем, как видно из чертежа:
\begin{equation}
    \label{eq:82.1}
    \sin\alpha = c/v.
\end{equation}
Таким образом, в сверхзвуковом потоке исходящее из некоторой точки возмущение распространяется только вниз по течению внутри конуса с углом раствора тем меньшим, чем меньше отношение $c/v$.
На всей области потока вне этого конуса возмущение в точке $O$ не отразится вовсе.

Определяемый равенством (\ref{eq:82.1}) угол называют \emph{углом Маха}.
Отношение же $v/c$, весьма часто встречающееся в газодинамике, называют \emph{числом Маха}:
\begin{equation}
    \label{eq:82.2}
    M=v/c.
\end{equation}
Поверхность, ограничивающую область, которую достигает исходящее из заданной точки возмущение, называют \emph{поверхностью Маха} или \emph{характеристической поверхностью}.

В общем случае произвольного стационарного течения эта поверхность не является уже конической во всем объеме потока.
Можно, однако, по-прежнему утверждать, что она пересекает в каждой своей точке линию тока под углом, равным углу Маха.
Значение же угла Маха меняется от точки к точке соответственно изменению скоростей $v$ и $c$.
Подчеркнем здесь, кстати, что при движении с большими скоростями скорость звука различна в разных местах газа-она меняется вместе с термодинамическими величинами (давлением, плотностью и т. д.), функцией которых она является
\footnote{При изучении звуковых воли в главе VIII мы могли считать скорость звука постоянной.}.
О скорости звука как функции координат точки говорят как о \emph{местной скорости звука}.

Описанные свойства сверхзвукового течения придают ему характер, совершенно отличный от характера дозвукового движения.
Если дозвуковой поток газа встречает на своем пути какое-либо препятствие, например, обтекает какое-либо тело, то наличие этого препятствия изменяет движение во всем пространстве как вверх, так и вниз по течению,влияние обтекаемого тела исчезает лишь асимптотически при удалении от тела.
Сверхзвуковой же поток натекает на препятствие "слепо"; влияние обтекаемого тела простирается лишь на область вниз по течению
\footnote{Во избежание недоразумений оговорим, что если перед обтекаемым телом возникает ударная волна, то эта область несколько увеличивается (см. \ref{sec:p122}).},
а во всей остальной области пространства вверх по течению газ движется так, как если бы никакого тела вообще не было.

В случае плоского стационарного течения газа вместо характеристических поверхностей можно говорить о \emph{характеристических линиях} (или просто \emph{характеристиках}) в плоскости движения.
Через всякую точку $O$ этой плоскости проходят две характеристики ($AA'$ и $BB'$ на рис. 51), пересекающие проходящую через эту же точку линию тока под углами, равными углу Маха.
Ветви $OA$ и $OB$ характеристик, направленные вниз по течению, можно назвать исходящими из точки $O$; они ограничивают область $AOB$ течения, на которую могут влиять исходящие из точки $O$ возмущения.
Ветви же $B'O$ и $A'O$ можно назвать приходящими в точку $O$; область $A'OB$ между ними есть та область течения, которая может влиять на движение в точке $O$.

Понятие о характеристиках (в трехмерном случае - характеристических поверхностях) имеет и несколько иной аспект.
Это - лучи, вдоль которых распространяются возмущения, удовлетворяющие условиям геометрической акустики.
Если, например, стационарный сверхзвуковой поток газа обтекает достаточно малое препятствие, то вдоль отходящих от этого препятствия характеристик расположится стационарное возмущение движения газа.
К этому результату мы пришли еще в \S \ref{sec:p68} при изучении геометрической акустики движущихся сред.

Говоря о возмущении состояния газа, мы подразумеваем слабое изменение каких-либо характеризующих это состояние величин: скорости, плотности, давления и т. п.
По этому поводу необходимо сделать следующую оговорку: со скоростью звука не распространяются возмущения значений энтропии газа (при постоянном давлении) и ротора его скорости.
Эти возмущения, раз возникнув, не перемещаются вовсе относительно газа, а относительно неподвижной системы координат переносятся вместе с газом со скоростью, равной скорости каждого данного его элемента.
Для энтропии это является непосредственным следствием закона ее сохранения (в идеальной жидкости), который как раз и означает, что энтропия каждого элемента газа остается постоянной при его перемещении.
Для ротора скорости (завихренности) то же самое следует из закона сохранения циркуляции.
Для этих возмущений характеристиками являются сами линии тока.

Подчеркнем, что последнее обстоятельство не меняет, разумеется, общей справедливости высказанных выше утверждений об областях влияния, так как для них был существен лишь факт существования наибольшей возможной (равной скорости звука) скорости распространения возмущений относительно самого газа.

\subsection*{Задача}

Найти соотношения между малыми изменениями скорости и термодинамических величин при произвольном малом возмущении в однородном потоке газа.

\texttt{Решение}. Обозначим малые изменения величин при возмущении символом $\delta$ (вместо штриха, как в \S \ref{sec:p64}). В линейном по этим величинам приближении уравнение Эйлера принимает вид

\begin{equation}
    \label{eq:82T1}
    \pd{\delta \V v}{t} + (\V v \nabla) \delta \V v + \frac1{\rho} \nabla \delta p = 0
\end{equation}
($\V v$ - постоянная певозмущеиная скорость потока), уравнение сохранении энтропии:
\begin{equation}
    \label{eq:82T2}
    \pd{\delta s}{t} + \V v \nabla \delta s = 0,
\end{equation}
и уравнение непрерывности:
\begin{equation}
    \label{eq:82T3}
    \pd{\delta p}{t} + \V v \nabla \delta p + \rho c^2 \Div \delta \V v = 0
\end{equation}
(здесь подставлено $\delta \rho = c^{-2} \delta p + (\partial\rho/\partial s)_p$;
члены с $\delta s$ выпадают в силу (\ref{eq:82T2})).
Для возмущения вида $\exp [i(\V{kr}-\omega t)]$ находим систему алгебраических уравнений:
\begin{eqnarray}
    (\V{vk}-\omega)\delta s = 0,   \quad (\V{vk}-\omega) \delta \V v + \V k \delta p/\rho = 0, \nonumber \\
    (\V{vk}-\omega) \delta p + \rho c^2 \V k \delta \V v = 0.
\end{eqnarray} Отсюда видно, что возможны два вида возмущений.

В одном из них (энтропийно-вихревая волна)
\[
    \omega = \V{vk},
    \quad \delta s \neq 0, \delta p = 0, 
    \quad \delta \rho = \ddp{\rho}{s}{p} \delta s,
    \quad \V k \delta \V v = 0;
\]
отличиа от нуля также и завихренность $\rot \delta \V v = i [\V k\delta \V v]$.
Возмущения $\delta s$ и $\delta \V v$ в этой волне независимы.
Равенство $\omega = \V{vk}$ означает перенос возмущения движущимся газом.

В другом типе возмущений

\begin{eqnarray}
    (\omega - \V{vk})^2 = c^2 k^2, \quad \delta s = 0, \quad \delta p = c^2 \delta \rho,\nonumber \\
    (\omega - \V{vk})\delta p = \rho c^2 \V k \delta \V v, \quad  [\V k \delta \V v] = 0. \nonumber
\end{eqnarray}

Это - звуковая волна с частотой, сдвинутой эффектом Доплера.
Задание возмущения одной из величин в этой волие определяет возмущения всех остальных величин.


\begin{comment}



\end{comment}

