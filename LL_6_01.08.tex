%parent: current_section.tex
\section{Сохранение циркуляции скорости}
\label{sec:p8}

Интеграл
\[
   \bm\Gamma = \oint \vv \; d\vect l,
\]
взятый вдоль замкнутого контура, называют \textit{циркуляцией скорости} вдоль
этого контура.

Рассмотрим замкнутый контур, проведенный в жидкости в некоторый момент времени.
Будем рассматривать его как "жидкий", т. е. как составленный из находящихся на
нем частиц жидкости. С течением времени эти частицы передвигаются, а с ними
перемещается и весь контур. Выясним, что происходит при этом с циркуляцией
скорости вдоль контура. Другими словами, вычислим производную по времени
\[
   \frac{d}{dt}\oint \vv\; dl.
\]
Мы пишем здесь полную производную по времени соответственно тому, что ищем
изменение циркуляции вдоль перемещающегося жидкого контура, а не вдоль контура,
неподвижного в пространстве.

Во избежание путаницы будем временно обозначать дифференцирование по координатам
знаком $\delta$, оставив знак $d$ для дифференцирования по времени. Кроме того,
заметим, что элемент $d\vect l$ длины контура можно написать в виде разности
$\delta\vect{r}$ радиус-векторов $r$ точек двух концов этого элемента. Таким
образом, напишем циркуляцию скорости в виде
\[
   \oint\vv\dr
\]
При дифференцировании этого интеграла по времени надо иметь в виду, что меняется
не только скорость, но и сам контур (т. е. его форма). Поэтому, внося знак
дифференцирования по времени под знак интеграла, надо дифференцировать не только
$\vv$, но и $\dr$:
\[
   \frac{d}{dt}\oint\vv\dr = \oint{\D{\vv}{t}\dr}+\oint{\vv\D{\dr}{t}}.
\]
Поскольку скорость $\vv$ есть не что иное, как производная по времени от
радиус-вектора $\vect r$, то
\[
   \vv \D{\dr}{t} = \vv\delta\D{\vect r}{t} = \vv\delta\vv = \delta \frac{v^2}{2}
\]
Но интеграл по замкнутому контуру от полного дифференциала равен нулю. Поэтому
второй из написанных интегралов исчезает и остается
\[
   \frac{d}{dt}\oint \vv \dr = \oint{\D{\vv}{t}\dr}.
\]

Теперь остается подставить сюда для ускорения $d\vv/dt$ его выражение согласно
(2,9):
\[
   \D{\vv}{t} = - \grad w.
\]
Применив формулу Стокса, получаем тогда (поскольку $\rot \grad w = 0$):
\[
   \oint{\D{\vv}{t}\dr} = \int{\rot \D{\vv}{t}\delta \vect f} = 0.
\]
Таким образом, переходя к прежним обозначениям, находим окончательно
\footnote{Этот результат сохраняет силу и в однородном поле тяжести, так как
$\rot \vect g \equiv 0$.}:
\[
   \frac{d}{dt}\oint \vv\; d \vect l = 0,
\]
или
\begin{equation}
   \label{eq:8.1}
   \oint \vv\; d \vect l = \const.
\end{equation}

Мы приходим к результату, что (в идеальной жидкости) циркуляция скорости вдоль
замкнутого жидкого контура остается неизменной со временем. Это утверждение
называют \textit{теоремой Томсона} (\textit{W. Thomson}, 1869) или
\textit{законом сохранения циркуляции скорости}. Подчеркнем, что он получен
путем использования уравнения Эйлера в форме (\ref{eq:2.9}) и потому связан с
предположением об изэнтропичности движения жидкости. Для неизэнтропического
движения этот закон не имеет места \footnote{С математической точки зрения необходимо,
чтобы между $p$ и $\rho$ существовала однозначная связь (при изэнтропическом движении
она определяется уравнением $s(p,\rho) = \const$). Тогда вектор $- \nabla p/\rho$ может
быть написан в виде градиента некоторой функции, что и требуется для вывода теоремы Томсона.}.

Применив теорему Томсона к бесконечно малому замкнутому контуру $\delta C$
и преобразовав интеграл по теореме Стокса, получим:
\begin{equation}
   \label{eq:8.2}
   \oint \vv\; d\vect l = \int \rot \vv\; d\vect f \approx \delta \vect f \cdot
   \rot \vv = \const,
\end{equation}
где $d\vect f$ — элемент жидкой поверхности, опирающийся на контур $\delta C$.
Вектор $\rot \vv$ часто называют \textit{завихренностью}\footnote{По английской терминологии
- vorticity.} течения жидкости в
данной ее точке. Постоянство произведения (\ref{eq:8.2}) можно наглядно истолковать,
сказав, что завихренность переносится вместе с движущейся жидкостью.

\subsection*{Задача}
Показать, что при нензэнтропическом течении для каждой перемещающейся частицы
остается постоянным связанное с ней значение произведения
$(\nabla s \cdot \rot \vv)/\rho$ (\textit{H. Ertel}, 1942).

\texttt{Решение.} При нензэнтропическом движении правая сторона уравнения Эйлера
(\ref{eq:2.3}) не может быть заменена на $- \nabla w$ и вместо уравнения (\ref{eq:2.11})
получается
\[
   \pd{w}{t} = \rot \lbrack \vv \vo \rbrack + \frac1{\rho^2}
   \lbrack \nabla \rho \cdot \nabla p \rbrack
\]
(для краткости обозначено $\vo = \rot \vv$). Умножим это равенство на
$\nabla s$; поскольку $s = s(p,\rho)$, $\nabla s$ выражается линейно через
$\nabla p$ и $\nabla \rho$ и произведение
$\nabla s \lbrack \nabla \rho \cdot \nabla p \rbrack$. После этого выражение
в правой стороне уравнения преобразуем следующим образом:
\begin{eqnarray*}
   \nabla s \pd{\vo}{t} =
   \nabla s \cdot \rot \lbrack \vv \vo \rbrack = \\
   - \Div \lbrack \nabla s \lbrack \vv \vo \rbrack \rbrack =
   - \Div (\vv (\vo \nabla s)) + \Div (\vo (\vv \nabla s)) = \\
   - (\vo \nabla s) \Div \vv - \vv \grad (\vo \nabla s) + \vo \grad (\vv \nabla s)
\end{eqnarray*}
Согласно (\ref{eq:2.6}) заменяем $(\vv \nabla s) = - \partial s/\partial t$ и получаем
уравнение
\[
   \pdt (\vo \nabla s) +\vv \grad (\vo \nabla s) + (\vo \nabla s) \Div v = 0.
\]
Первые два члена объединяются в $d(\vo \nabla s)/dt$ (где
$d/dt = \partial/\partial t + (\vv \nabla)$), а в последнем заменяем согласно
(\ref{eq:1.3}) $\rho \Div \vv = - d \rho / dt$. В результате получаем
\[
   \frac{d}{dt} \frac{\vo \nabla s}{\rho} = 0,
\]
чем и выражается искомый закон сохранения.

