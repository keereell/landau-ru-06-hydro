\section{Волны во вращающейся жидкости}
\label{sec:p14}

Другой своеобразный тип внутренних волн может распространяться в равномерно
вращающейся как целое несжимаемой жидкости. Их происхождение связано с
возникающими при вращении кориолисовыми силами.

Будем рассматривать жидкость в системе координат, вращающейся вместе с ней. Как
известно, при таком описании в механические уравнения движения должны быть
введены дополнительные силы — центробежная и кориолисова. Соответственно этому,
надо добавить такие же силы (отнесенные к единичной массе жидкости) в правую
сторону уравнения Эйлера. Центробежная сила может быть представлена в виде
градиента $\nabla\lbrack\bm{\Omega}\vect{r}\rbrack^2/2$, где $\bm\Omega$—
вектор угловой скорости вращения жидкости. Этот член можно объединить с силой
$-\nabla p/\rho$, введя эффективное давление
\begin{equation}
   \label{eq:14.1}
   P = p - \frac{\rho}{2}\left\lbrack \bm\Omega\vect r \right\rbrack^2.
\end{equation}
Кориолисова же сила равна $2\left\lbrack \vv\bm\Omega \right\rbrack$, она
появляется лишь при движении жидкости относительно вращающейся системы координат
($\vv$ — скорость в этой системе). Перенеся этот член в левую сторону уравнения
Эйлера, напишем его в виде
\begin{equation}
   \label{eq:14.2}
   \pd{\vv}{t} + \vnv + 2\left\lbrack\bm\Omega\vv\right\rbrack =
   - \frac1{\rho}\nabla P.
\end{equation}
Уравнение же непрерывности сохраняет свой прежний вид, сводясь для несжимаемой
жидкости к равенству $\Div\vv = 0$.

Снова будем считать амплитуду волны малой и пренебрежем квадратичным по скорости
членом в уравнении (\ref{eq:14.2}), которое примет вид
\begin{equation}
   \label{eq:14.3}
   \pd{\vv}{t}+ 2\left\lbrack \bm\Omega\vv \right\rbrack =
   - \frac1{\rho}\nabla p',
\end{equation}
где $p'$ — переменная часть давления в волне, а $\rho = \const$. Сразу же
исключим давление, применив к обеим сторонам уравнения (\ref{eq:14.3}) операцию $\rot$.
Правая сторона уравнения обращается в нуль, а в левой имеем, с учетом
несжимаемости жидкости:
\[
   \rot \left\lbrack \bm\Omega\vv \right\rbrack =
   \bm\Omega\Div\vv - (\bm\Omega\nabla)\vv =
   - (\bm\Omega\nabla)\vv
\]
Выбрав направление $\bm\Omega$ в качестве оси $z$, запишем получающееся
уравнение в виде
\begin{equation}
   \label{eq:14.4}
   \pdt\rot\vv = 2\Omega \pd{\vv}{z}.
\end{equation}

Ищем решение в виде плоской волны
\begin{equation}
   \label{eq:14.5}
   \vv = \vect A e^{i(\vect{kr} - \omega t)},
\end{equation}
удовлетворяющей (в силу уравнения $\Div\vv = 0$) условию поперечности
\begin{equation}
   \label{eq:14.6}
   \vect{kA} = 0.
\end{equation}
Подстановка (\ref{eq:14.5}) в уравнение (\ref{eq:14.4}) дает
\begin{equation}
   \label{eq:14.7}
   \omega \left\lbrack \vect{kv} \right\rbrack =
   2i\Omega k_z \vv.
\end{equation}

Закон дисперсии волн получается исключением $\vv$ из этого векторного равенства.
Умножив его с обеих сторон векторно на $k$, переписываем его в виде
\[
   - \omega k^2\vv = 2i\Omega k_z \left\lbrack \vect{kv} \right\rbrack
\]
и, сравнив друг с другом оба равенства, находим искомую зависимость $\omega$ от
$\vect k$:
\begin{equation}
   \label{eq:14.8}
   \omega = 2\Omega \frac{k_z}{k} = 2\Omega\cos\theta,
\end{equation}
где $\theta$ — угол между $\vect k$ и $\bm\Omega$.

С учетом (\ref{eq:14.4}) равенство (\ref{eq:14.7}) принимает вид
\[
   \left\lbrack \vect{n\nu} \right\rbrack = i\vv,
\]
где $\vect n = \vect{k}/k$. Если представить комплексную амплитуду волны как
$\vect A = \vect a + i \vect b$ с вещественными векторами $\vect a$ и $\vect b$,
то отсюда следует, что $\left\lbrack \vect{nb} \right\rbrack = \vect a$, —
векторы $\vect a$ и $\vect b$ (оба лежащие в плоскости, перпендикулярной вектору
$\vect k$) взаимно перпендикулярны и одинаковы по величине. Выбрав их
направления в качестве осей $x$ и $y$ и отделив в (\ref{eq:14.5}) вещественную и мнимую
части, найдем, что
\[
   v_x =   a\cos(\omega t - \vect{kr}),\;
   v_y = - a\sin(\omega t - \vect{kr}).
\]
Таким образом, волна обладает круговой поляризацией: в каждой точке пространства
вектор $\vv$ вращается со временем, оставаясь постоянным по величине\footnote{Напомним,
что речь идет по отношению к вращающейся системе координат! По отношению к неподвижной
системе на это движение налагается еще и вращение всей жидкости как целого.}.

Скорость распространения волны:
\begin{equation}
   \label{eq:14.9}
   \vect U = \pd{\omega}{k} = \frac{2\Omega}{k}\lbrace \nu - \vect{n(n\nu)} \rbrace,
\end{equation}
где $\nu$ — единичный вектор в направлении $\bm\Omega$; как и в гравитационных
внутренних волнах, эта скорость перпендикулярна волновому вектору. Ее абсолютная
величина и проекция на направление $\bm\Omega$:

\[
   U = \frac{2\Omega}{k}\sin\theta,\;
   \vect U \nu = \frac{2\Omega}{k}\sin^2\theta = U\sin\theta.
\]

Рассмотренные волны называют \textit{инерционными}. Поскольку кориолисовы силы
не совершают работы над движущейся жидкостью, заключенная в этих волнах энергия
— целиком кинетическая.

Особый вид инерционных осесимметричных (не плоских) волн может распространяться
вдоль оси-вращения жидкости — см. задачу.

В заключение сделаем еще одно замечание, относящееся к стационарным движениям во
вращающейся жидкости, а не к распространению волн в ней.

Пусть $l$ — характерный параметр длины такого движения, а $u$ — характерная
скорость. По порядку величины член $\vnv$ в уравнении (\ref{eq:14.2}) равен $u^2/l$, а
член $2 \left\lbrack \bm\Omega\vv \right\rbrack \sim \Omega u$. Если
$u/l\Omega \ll 1$, то первым можно пренебречь по сравнению со вторым и тогда
уравнение стационарного движения сводится к
\begin{equation}
   \label{eq:14.10}
   2 \left\lbrack \bm\Omega\vv \right\rbrack = - \frac1{\rho}\nabla P
\end{equation}
или
\[
   2\Omega v_y = \frac1{\rho}\pd{P}{x},\;
   2\Omega v_x = \frac1{\rho}\pd{P}{y},\;
   \pd{P}{z} = 0,
\]
где $x,y$ — декартовы координаты в плоскости, перпендикулярной оси вращения.
Отсюда видно, что $P$, а потому и $v_x, v_y$, не зависят от продольной
координаты $z$. Далее, исключив $P$ из двух первых уравнений, получим
\[
   \pd{v_x}{x} + \pd{v_y}{y} = 0,
\]
после чего из уравнения непрерывности $\Div\vv$ следует, что $\partial
v_z/\partial z = 0$. Таким образом, стационарное движение (относительно
вращающейся системы координат) в быстро вращающейся жидкости представляет собой
наложение двух независимых движений: плоского течения в поперечной плоскости и
осевого движения, не зависящего от координаты $z$ (\textit{J. Proudman}, 1916).


\subsection*{Задачи}

\paragraph*{1}
Определить движение в осесимметричной волне, распространяющейся вдоль оси
вращающейся как целое несжимаемой жидкости (\textit{W. Thomson}, 1880).

\texttt{Решение.} Введем цилиндрические координаты $r,\varphi,z$ с осью $z$
вдоль вектора $\Omega$. В осесимметричной волне все величины не зависят от
угловой переменной $\varphi$. Зависимость же от времени и координаты $z$ дается
множителем вида $\exp{\lbrace i(kz - \omega t)\rbrace}$. Раскрыв уравнение
(\ref{eq:14.3}) в компонентах, получим
\begin{eqnarray}
   \label{eq:14.T1.1}
   -i\omega v_r - 2\Omega v_{\varphi} = - \frac1{\rho}\pd{p'}{r},\\
   \label{eq:14.T1.2}
   -i\omega v_{\varphi} + 2\Omega v_r = 0, \; -i\omega v_z = - \frac{ik}{\rho}p'.
\end{eqnarray}
Сюда надо присоединить уравнение непрерывности
\begin{equation}
   \label{eq:14.T1.3}
   \frac1{r}\frac{\partial}{\partial r}(r v_r) + ikv_z = 0.
\end{equation}
Выразив $\varphi$ и $p'$ через $v_r$ из (\ref{eq:14.T1.2}) и (\ref{eq:14.T1.3}) и подставив в (\ref{eq:14.T1.1}), получим
уравнение
\begin{equation}
   \label{eq:14.T1.4}
   \frac{d^2F}{dr^2} + \frac1{r}\D{F}{r} +
   \left\lbrack \frac{4\Omega^2k^2}{\omega^2} - k^2 - \frac1{r^2} \right\rbrack F = 0
\end{equation}
для функции $F(r)$, определяющей радиальную зависимость скорости $v_r$:
\[
   v_r = F(r)e^{i(\omega t - kz)}.
\]
Решение этого уравнения, обращающееся в нуль при $r = 0$, есть
\begin{equation}
   \label{eq:14.T1.5}
   F = \const\cdot J_1 \left( kr\sqrt{\frac{4\Omega^2}{\omega^2}-1} \right),
\end{equation}
где $J_1$ — функция Бесселя порядка 1.

Вся картина движения в волне распадается на области, ограниченные коаксиальными
цилиндрическими поверхностями с радиусами $r_n$, определяемыми равенствами
\[
   kr_n\sqrt{\frac{4\Omega^2}{\omega^2}-1} = x_n,
\]
где $x_1,x_2,\dots$ — последовательные нули функции $J_1(x)$. На этих
поверхностях $v_r= 0$; другими словами, жидкость никогда не пересекает их.

Отметим, что для рассматриваемых волн в неограничениой жидкости частота $\omega$
не зависит от $k$. Возможные значения частоты ограничены, однако, условием
$\omega < 2\Omega$; в противном случае уравнение (\ref{eq:14.T1.4}) не имеет решения,
удовлетворяющего необходимым условиям конечности.

Если же вращающаяся жидкость ограничена цилиндрической стенкой (радиуса $R$), то
должно быть учтено условие $v_r = 0$ на стенке. Отсюда возникает соотношение
\[
   ka\sqrt{\frac{4\Omega^2}{\omega^2}-1} = x_n,
\]
устанавливающее связь между $\omega$ и $k$ для волны с заданным значением $n$
(т.е. числом коаксиальных областей в ней).


\paragraph*{2}
2. Получить уравнение, описывающее произвольное малое возмущение давления во
вращающейся жидкости.

\texttt{Решение.} Уравнение (\ref{eq:14.3}), расписанное в компонентах, дает

\begin{eqnarray}
   \label{eq:14.T2.1}
   \pd{v_x}{t} - 2\Omega v_y = - \frac1{\rho} \pd{p'}{x}, \nonumber \\
   \pd{v_y}{t} + 2\Omega v_x = - \frac1{\rho} \pd{p'}{x},  \\
   \pd{v_z}{t}               = - \frac1{\rho} \pd{p'}{z}. \nonumber
\end{eqnarray}
Продифференцировав эти три уравнения соответственно по $x,y,z$ и сложив их с
учетом уравнения $\Div\vv = 0$, получим:
\[
   \frac1{\rho}\nabla p' = 2\Omega \left( \pd{v_y}{x} - \pd{v_x}{y} \right).
\]
Дифференцирование этого уравнения по $t$, снова с учетом уравнений (\ref{eq:14.T2.1}), дает
\[
   \frac1{\rho}\pdt \nabla p' = 4 \Omega^2 \pd{v_z}{z},
\]
а еще одно дифференцирование по $t$ приводит к окончательному уравнению
\begin{equation}
   \label{eq:14.T2.2}
   \frac{\partial^2}{\partial t^2} \nabla p'
   + 4 \Omega^2 \frac{\partial^2 p'}{\partial z^2} = 0
\end{equation}

Для периодических возмущений с частотой $\omega$ это уравнение сводится к
\begin{equation}
   \label{eq:14.T2.3}
   \frac{\partial^2p'}{\partial x^2} +
   \frac{\partial^2p'}{\partial y^2} +
   \left( 1-\frac{4 \Omega^2}{\omega^2} \frac{\partial^2 p'}{\partial z^2} \right)
\end{equation}
Для волн вида (\ref{eq:14.5}) отсюда получается, разумеется, уже известное дисперсионное
соотношение (\ref{eq:14.8}); при этом $\omega < 2 \Omega$ и коэффициент при $\partial
^2p'/\partial z^2$ в уравнении (\ref{eq:14.T2.3}) отрицателей. Возмущения из точечного
источника распространяются вдоль образующих конуса с осью вдоль $\Omega$ и углом
раствора $2\theta$, где $\sin\theta = \omega/2 \Omega$.

При $\omega > 2 \Omega$ коэффициент при $\partial ^2p'/\partial z^2$ в уравнении
(\ref{eq:14.T2.3}) положителен, и путем очевидного изменения масштаба вдоль оси $z$ оно
приводится к уравнению Лапласа. Влияние точечного источника возмущений
простирается в этом случае по всему объему жидкости, причем убывает прн удалении
от источника по степенному закону.
