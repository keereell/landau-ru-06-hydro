\section{Гофрировочная неустойчивость ударных волн}\label{sec:p90}

Соблюдение условий эволюционности само по себе необходимо, но еще недостаточно
для гарантирования устойчивости ударной волны. Волна может оказаться
неустойчивой по отношению к возмущениям, характеризующимся периодичностью вдоль
поверхности разрыва и представляющим собой как бы «рябь», или «гофрировку», на
этой поверхности (такого рода возмущения рассматривались уже в § 29 для
тангенциальных разрывов)1). Покажем, каким образом исследуется этот вопрос для
ударных волн в произвольной среде (С. П. Дьяков, 1954).

Пусть ударная волна покоится, занимая плоскость х = 0; жидкость движется сквозь
нее слева направо, в положительном направлении оси х. Пусть поверхность разрыва
испытывает возмущение, при котором ее точки смещаются вдоль оси х на малую
величину


где ky — волновой вектор «ряби». Эта рябь на поверхности вызывает возмущение
течения позади ударной волны, в области х > 0 (течение же перед разрывом, х <
0, не испытывает возмущения в силу своей сверхзвуковой скорости).

Произвольное возмущение течения складывается из энтропийно-вихревой волны и
звуковой волны (см. задачу к § 82). В обоих зависимость величин от времени и
координат дается множителем вида expi(kr — ) с той же частотой со, что и в
(\ref{eq:90.1}). Из соображений симметрии очевидно, что волновой вектор к лежит
в плоскости ху его -компонента совпадает с ky в (\ref{eq:90.1}), а х-компонента
различна для возмущений двух типов.

В энтропийно-вихревой волне kv2 = со, т. е. = co/y2 (V2 — невозмущенная
скорость газа за разрывом). В этой волне возмущение давления отсутствует,
возмущение удельного объема связано с возмущением энтропии, 6У(ЭНТ) =
(dV/ds)p8s, а возму-

1) Неустойчивость по отношению к таким возмущениям называют гофри-ровочной
(corrugation Instability по английской терминологии).


щение скорости подчинено условию


(\ref{eq:90.2})

В звуковой волне в движущемся газе связь между частотой и волновым вектором
дается равенством (со — kv)2 = c262 (см. (\ref{eq:68.1})); поэтому kx в этой
волне определяется уравнением

(со -kxv2y = d(kl + kl).	(\ref{eq:90.3})

Возмущения давления, удельного объема и скорости связаны соотношениями:

(\ref{eq:90.4})

(\ref{eq:90.5})

Возмущение в целом представляется линейной комбинацией возмущений обоих типов:

(\ref{eq:90.6})

Оно должно удовлетворять определенным граничным условиям на возмущенной
поверхности разрыва.

Прежде всего, на этой поверхности должна быть непрерывна тангенциальная к ней
составляющая скорости, а скачок нормальной составляющей должен выражаться через
возмущенные давление и плотность равенством (\ref{eq:85.7}). Эти условия
записываются как


где t и n — единичные векторы касательной и нормали к поверхности разрыва (рис.
59). С точностью до величин первого порядка малости компоненты этих векторов (в
плоскости ху) равны	1) и n(1,—ik) выражение iki возникает как производная
dt,/dy. С этой же точностью граничные условия для скорости принимают вид



2 L pi — pi Vi Далее, возмущенные значения p2 + бр и V2 + 6V2 должны
удовлетворять тому же уравнению адиабаты Гюгонио, что и невозмущенные р2 и У2.
Отсюда получаем условие, связывающее бр и 8V:

где производная берется вдоль адиабаты.

(\ref{eq:90.8})


Наконец, еще одно соотношение возникает из связи между потоком вещества через
поверхность разрыва и скачками давления и плотности на ней. Для невозмущенного
разрыва это соотношение выражается формулой (\ref{eq:85.6}), а для возмущенного
аналогичное соотношение есть


где u — скорость точек поверхности разрыва. В первом приближении по малым
величинам имеем un = —uразлагая написанное равенство также и по степеням бр и
6V, получим:


Равенства (\ref{eq:90.2}), (90,4—5), (90,7—9) составляют систему восьми
линейных алгебраических уравнений для восьми величин Условие совместности этих
уравнений (выражаемое равенством нулю определителя их коэффициентов) имеет вид:


где для краткости обозначено h — (dVifdpi), а / имеет обычный смысл: j = Vi/Vi
= V2/V2. Величину kx в (\ref{eq:90.10}) надо понимать как функцию ky и <о,
определяемую равенством (\ref{eq:90.3}).

Условие неустойчивости состоит в существовании возмущений, экспоненциально
возрастающих со временем, причем они должны экспоненциально убывать с удалением
от поверхности разрыва (т. е. при лс-:оо); последнее условие означает, что
источником возмущения является сама ударная волна, а не какой-то внешний по
отношению к ней источник. Другими словами, волна неустойчива, если уравнение
(\ref{eq:90.10}) имеет решения, у которых

(\ref{eq:90.11})

Исследование уравнения (\ref{eq:90.10}) на предмет выяснения условий
существования таких решений довольно громоздко. Мы не будем производить его
здесь, ограничившись указанием окончательного результата2). Гофрировочная
неустойчивость ударной

') Все эти равенства берутся при х =»= 0, и под перечисленными величинами в них
могут подразумеваться постоянные амплитуды, без переменных экспоненциальных
множителей.

2) Это исследование можио иайти в оригинальной статье: Дьяков С. П.— ЖЭТФ,
1954, т. 27, с. 288. В следующем параграфе будет приведено еще и менее строгое,
но более наглядное обоснование условий (90,12—13),


волны возникает если


или

(\ref{eq:90.13})

напомним, что производная берется вдоль ударной адиабаты (при заданных рь Vi)
').

Условия (90,12—13) отвечают наличию у уравнения (\ref{eq:90.10}) комплексных
корней, удовлетворяющих требованиям (\ref{eq:90.11}). Но в определенных
условиях это уравнение может иметь также и корни с вещественными © и kx,
отвечающие «уходящим» от разрыва реальным незатухающим звуковым и энтропийным
вол-нам, т. е. спонтанному излучению звука поверхностью разрыва. Мы будем
говорить о такой ситуации как об особом виде неустойчивости ударной волны, хотя
неустойчивости в буквальном смысле здесь нет, — раз созданное на поверхности
разрыва возмущение (рябь) неограниченно долго продолжает излучать волны, не
затухая и не усиливаясь при этом; энергия, уносимая излучаемыми волнами,
черпается из всей движущейся среды2).

Для определения условий возникновения этого явления, преобразуем уравнение
(\ref{eq:90.10}), введя угол 0 между к и осью jq тогда


(coo — частота звука в системе координат, движущейся вместе с газом за ударной
волной), и получаем квадратное относительно cos 9 уравнение:


Скорость распространения звуковой волны в движущемся со скоростью газе, по
отношению к неподвижной поверхности

') Отметим, что при выводе (90,12—13) используется только обязательное условие
(\ref{eq:88.1}), но не используется неравенство рг> р. Поэтому эти условия
неустойчивости относятся и к ударным волнам разрежения, которые могли бы
существовать при (d2 Vldp2)s <. 0.

2) Сравните с аналогичной ситуацией для тангенциальных разрывов — задача 2 §
84.


разрыва, есть y2 + c2cos0. Звуковая волна будет уходящей, если эта сумма
положительна, т. е. если

(\ref{eq:90.16})

(значения cosOCO отвечают случаям, когда вектор к направлен в сторону разрыва,
но снос звуковой волны движущимся газом делает ее все же «уходящей»).
Спонтанное излучение звука ударной волной возникает, если уравнение
(\ref{eq:90.15}) имеет корень, лежащий в этих пределах. Простое исследование
приводит к следующим неравенствам, определяющим область этой неустойчивости '):


(\ref{eq:90.17})


(нижний и верхний пределы здесь фактически отвечают нижнему и верхнему пределам
в условиях (\ref{eq:90.16})). Область (\ref{eq:90.17}) примыкает к области
неустойчивости (\ref{eq:90.13}), расширяя ее.

К происхождению неустойчивости ударных волн в области (\ref{eq:90.17}) можно
подойти также и с несколько иной точки зрения, рассмотрев отражение от
поверхности разрыва звука, падающего на нее со стороны сжатого газа. Поскольку
ударная волна движется относительно газа впереди нее со сверхзвуковой
скоростью, то в этот газ звук не проникает. В газе же позади волны будем иметь,
наряду с падающей звуковой волной, еще и отраженную звуковую и
энтропийно-вихревую волны (а на самой поверхности разрыва возникает рябь).
Задача об определении коэффициента отражения по своей постановке близка к
задаче об исследовании устойчивости. Разница состоит в том, что наряду с
подлежащими определению амплитудами исходящих от разрыва (отраженных) волн в
граничных условиях фигурирует еще и заданная амплитуда приходящей (падающей)
звуковой волны.  Вместо системы однородных алгебраических уравнений мы будем
иметь теперь систему неоднородных уравнений, в которых роль неоднородности
играют члены с амплитудой падающей волны. Решение этой системы дается
выражениями, в знаменателях которых стоит определитель однородных уравнений,—
как раз тот, приравнивание которого нулю дает дисперсионное уравнение
спонтанных возмущений (\ref{eq:90.10}). Тот факт, что в области
(\ref{eq:90.17}) это уравнение имеет вещественные корни для cos 0, означает,
что существуют определенные значения угла отражения (и тем самым угла падения),
при которых коэффициент отражения становится бесконечным.  Это — другая фор-

') Эга неустойчивость тоже была указана С. П. Дьяковым (1954); правильное
значение нижней границы в (\ref{eq:90.17}) найдено В. М. Конторовичем (1957).


мулировка возможности спонтанного излучения звука, т. е. излучения без падающей
извне звуковой волны.

То же самое относится и к коэффициенту прохождения звука, падающего на
поверхность разрыва спереди, навстречу ей. В этом случае не существует
отраженной волны, а позади поверхности разрыва возникают прошедшие звуковая и
энтропийно-вихревая волны. В области (\ref{eq:90.17}) возможно обращение
коэффициента прохождения в бесконечность ').

Скажем несколько слов о некоторых возможных, в принципе, типах ударных адиабат,
содержащих области рассмотренных неустойчивостей2).

Условие (\ref{eq:90.12}) требует отрицательной производной dp2/dV2, причем
ударная адиабата должна быть наклонена (к оси абсцисс) в точке 2 менее круто,
чем проведенная в нее хорда 12 (т. е. обратно тому, что имеет место в обычных
случаях — рис. 53). Для этого адиабата должна перегнуться, как показано на рис.
60; условие неустойчивости (\ref{eq:90.12}) выполняется на участке ab.


Условие (\ref{eq:90.13}) требует положительности производной dpi/dVi, причем
наклон адиабаты должен быть достаточно мал. На рис. 60 это условие выполняется
на определенных отрезках адиабаты, непосредственно примыкающих к точкам а и b и
расширяющих, таким образом, область неустойчивости. Условие (\ref{eq:90.13})
может оказаться выполненным и на участке (cd на рис. 61) адиабаты, не
содержащей участка типа ab.

') Вычисление коэффициентов отражения и прохождения звука на ударной волне при
произвольных направлениях падения в произвольных средах — см. Дьяков С. П. —
ЖЭТФ, 1957, т. 33, с. 948, 962; Конторович В. М,—ЖЭТФ, 1957, т. 33, с. 1527;
Акустический журнал, 1959, т. 5, с. 314.

2) В политропном газе А =» —(йМ)2, в чем легко убедиться с помощью полученных в
§ 89 формул. Ни одно из условий (90,12—13) и (\ref{eq:90.17}) при этом заведомо
не выполняется, так что ударная волна устойчива. Устойчивы, конечно, также и
ударные волны слабой интенсивности в произвольной среде.


Условие (\ref{eq:90.17}) еще менее жестко, чем (\ref{eq:90.13}) и еще
дополнительно расширяет область неустойчивости на адиабатах Гю-гонио с dpi/dV2
> 0. Более того, нижний предел в (\ref{eq:90.17}) может быть отрицательным, так
что неустойчивость этого типа может, в принципе, иметь место и в некоторых
участках адиабат обычного вида, со всюду отрицательной производной dp2/dV

Вопрос о судьбе гофрировочно-неустойчивых ударных волн тесно связан
со,следующим замечательным обстоятельством: при выполнении условий
(\ref{eq:90.12}) или (\ref{eq:90.13}) решение ггдродинами-ческих уравнений
оказывается неоднозначным (С. S.  Gardner, 1963). Для двух состояний среды, / и
2, связанных друг с другом соотношениями (85,1—3), ударная волна является
обычно единственным решением задачи (одномерной) о течении, переводящем среду
из состояния 1 в 2.  Оказывается, что если в состоянии 2 выполнены условия
(\ref{eq:90.12}) или (\ref{eq:90.13}), то решение указанной гидродинамической
задачи не однозначно: переход из состояния 1 в 2 может быть осуществлен не
только в ударной волне, но и через более сложную систему волн. Это второе
решение (его можно назвать распадным) состоит из ударной волны меньшей
интенсивности, следующего за ней контактного разрыва и из изэнтропической
нестационарной волны разрежения (см. ниже § 99), распространяющейся
(относительно газа позади ударной волны) в противоположном направлении; в
ударной волне энтропия увеличивается от si до некоторого значения s3 С s2, а
дальнейшее увеличение от S3 до заданного s2 происходит скачком в контактном
разрыве (эта картина относится к типу, изображенному ниже на рис. 78, б;
предполагается выполненным неравенство (\ref{eq:86.2}))').

Вопрос о том, чем определяется отбор одного из двух решений в конкретных
гидродинамических задачах, не ясен. Если отбирается распадное решение, то это
означало бы, что неустойчивость ударной волны с самопроизвольным усилением
поверхностной ряби вообще не осуществляется. По-видимому, однако, такой отбор
не может быть связан именно с этой неустойчивостью, поскольку неоднозначность
решения не ограничена условиями (90,12—13) 2).

Задачи

1. На ударную волну падает сзади (со стороны сжатого газа) нормально к ней
плоская звуковая волна. Определить коэффициент отражения звука.

') В статье Gardner С. S. — Phys. Fluids, 1963, v. 6, p. 1366 это показано для
области (\ref{eq:90.13}). Более общее рассмотрение, включающее н область
(\ref{eq:90.12}), дано Кузнецовым Н. А1 — ЖЭТФ, 1985, т. 88, с. 470; там же
рассмотрены ударные адиабаты с нарушением условия (d2V/dp2)s > 0, когда
распадные решения складываются из других совокупностей воли.

2) По-видимому, область неоднозначности простирается иа ударной адиабате
несколько за пределы области неустойчивости, определяемой этими условиями. Cmi
об этом указанную выше статью Н. М. Кузнецова,


Решение. Рассматриваем процесс в системе координат, в которой ударная волна
покоится, а газ движется через нее в положительном направлении оси х\ падающая
звуковая волна распространяется в отрицательном направлении оси х. При
нормальном падении (а потому и отражении) в отраженной энтропийной волне
скорость 6v,3HI = 0. Возмущение давления: бр = = бр,зв + бр'0, где индекс
(0) относится к падающей, а индекс (зв) — к отраженной звуковым волнам. Для
скорости 5vx = имеем


(разность вместо суммы возникает ввиду противоположных направлений
распространения обоих волн). Второе из граничных условий (\ref{eq:90.7}) имеет
прежний вид (но в нем теперь;с учетом (\ref{eq:90.8}) и формулы (\ref{eq:85.6})
переписываем его как


Приравняв друг другу оба выражения dv, получим для искомого отношения амплитуд
давления в отраженной и падающей звуковых волнах:

(где М2 = о2/с2). Оно обращается в бесконечность иа верхней границе области
(\ref{eq:90.17})

Для политропного газа h — — Mj~2. При слабой интенсивности ударной волны (рг —
pipi) отношение (1) стремится к нулю как (рг — pi)2, а в обратном случае
большой интенсивности стремится к постоянному пределу


2. На ударную волну падает спереди, нормально к ней, плоская звуковая волна.
Определить коэффициент прохождения звука ').

Решение. Возмущение в газе 1 перед ударной волной



а в газе 2 позади нее:

(индексы (0), (зв), (энт) относятся к падающей звуковой и к прошедшим звуковой
и энтропийным волнам). Возмущения брг и SV2 связаны друг с другом соотношением,
следующим из уравнения ударной адиабаты: если последнее выражено в виде V2 =
V2(p2: pi, Vi), то


' Для политропного газа эта задача рассматривалась Д. И. Блохинце-вым (1945) и
Бюргерсом (/. М. Burgers, 1946).


(индекс Н у производных указывает, что они берутся вдоль адиабаты Гю-гонно1).
Граничное условие (\ref{eq:90.7}) заменяется теперь на

Приравняв два выражения для бог — боь получим для искомого отношения амплитуд в
прошедшей и падающей звуковых волнах:

где h имеет прежнее значение, а


Для политропного газа


у + 1 М и коэффициент прохождения:


При слабой интенсивности ударной волны отсюда получается УЗВ| , V+1 Pi-Pi

а в обратном случае большой интенсивности:


В обоих случаях амплитуда давления в прошедшей звуковой волне возрастает по
сравнению с давлением в падающей волне.

