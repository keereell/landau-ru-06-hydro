%parent: current_section.tex

\section{Стационарный поток сжимаемого газа}\label{sec:p83}

Уже непосредственно из уравнения Бернулли можно получить ряд общих результатов, касающихся произвольного адиабатического стационарного движения сжимаемого газа.
Уравнение Бернулли для стационарного движения гласит
\[
    w + \frac{v^2}{2} = \const,
\]
где $\const$ - величина, постоянная вдоль каждой из линий тока (если же движение потенциально, то $\const$ одинакова и для разных яиний тока, т.е. во всем объеме жидкости).
Если на одной линии тока есть точка, в которой скорость газа равна нулю, то можно написать уравнение Бернулли в виде
\begin{equation}
    \label{eq:83.1}
    w + \frac{v^2}{2} = w_0,
\end{equation}
где $w_0$ - значение тепловой функции в точке с $v = 0$.

Уравнение сохранения энтропии при стационарном движении сводится к 
$\V v \nabla s = v \partial s/ \partial l = 0$, т.е. $s = \const$, где $\const$ есть опять величина, постоянная вдоль линии тока. Напишем это уравнение в виде, аналогичном (\ref{eq:83.1}):

\begin{equation}
    \label{eq:83.2}
    s = s_0.
\end{equation}


Из уравнения (\ref{eq:83.1}) видно, что скорость $v$ больше в тех местах, где тепловая функция $w$ меньше.
Максимальное (вдоль данной линии тока) значение скорость имеет в точке, в которой $w$ минимально.
Но при постоянной энтропии имеем $dw = dp/\rho$; поскольку $\rho>0$, то дифференциалы $dw$ и $dp$ имеют одинаковые знаки и потому изменение $w$ и $p$ направлено всегда в одну сторону. 
Следовательно, можно сказать, что вдоль линии тока скорость всегда падает с увеличением давления, и наоборот.

Наименьшее возможное значение давление и тепловая функция получают (при адиабатическом процессе) при равной нулю абсолютной температуре $T = 0$.
Соответствующее значение давления есть $p = 0$, а значение $w$ при $T = 0$ примем условно за нулевое значение, от которого отсчитывается энергия;
тогда будет и $w = 0$ при $T = 0$. Из (\ref{eq:83.1}) заключаем теперь, что наибольшее возможное значение скорости (при заданном значении термодинамических величин в точке с $v = 0$) равно
\begin{equation}
    \label{eq:83.3}
    v_{max} = \sqrt{2w_0}.
\end{equation}
Эта скорость может достигаться при стационарном вытекании газа в вакуум 
\footnote{В действительности, конечно, при сильном понижении температуры должна произойти конденсация газа и образование двухфазной системы — тумана.}.

Выясним теперь характер изменения вдоль линии тока плотности потока жидкости $j = \rho v$.
Из уравнения Эйлера $\vnv = - \nabla p/\rho$ находим, что вдоль линии тока имеет место соотношение
\[
    vdv + \frac{dp}{\rho} = 0
\]
между дифференциалами $dv$ и $dp$. Написав $dp = c^2 d \rho$, имеем отсюда

\begin{equation}
    \label{eq:83.4}
    \frac{d \rho}{dv} = - \frac{\rho v}{c^2}
\end{equation}
и затем:

\begin{equation}
    \label{eq:83.5}
    \frac{d(\rho v)}{dv} = \rho \left( 1 - \frac{v^2}{c^2} \right).
\end{equation}
Отсюда видно, что по мере возрастания скорости вдоль линии тока плотность потока возрастает до тех пор, пока скорость остается дозвуковой.
В области же сверхзвукового движения плотность потока падает с увеличением скорости и обращается в нуль вместе с $p$ при $v = v_{max}$ (рис. 52).
Это существенное различие между до- и сверхзвуковыми стационарными потоками может быть истолковано наглядно еще и следующим образом.
В дозвуковом потоке линии тока сближаются друг с другом в направлении увеличения скорости.
При сверхзвуковом же движении линии тока расходятся по мере увеличения скорости.

Поток $j$ имеет максимальное значение $j_*$ в точке, в которой скорость газа равна местному значению скорости звука:
\begin{equation}
    \label{eq:83.6}
    j_* = \rho_* c_* ,
\end{equation}
где буквы с индексом * показывают значения соответствующих величин в этой точке.
Скорость $v_* = c_*$ называют \emph{критической}.

В общем случае произвольного газа критические значения величин могут быть выражены через значения величин в точке с $v = 0$ в результате совместного решения уравнений

\begin{equation}
    \label{eq:83.7}
    s_* = s_0, \quad w_* + \frac{c^2_*}{2} = w_0.
\end{equation}
Очевидно, что всякий раз, когда число $M = v/c < 1$, мы будем также, иметь $v/c_*<1$, а когда $M>1$, то и $v/c_* >1$.
Поэтому в данном случае отношение $M_* = v/c$ может служить критерием, аналогичным числу Маха, и даже более удобным, поскольку $c_*$ есть величина постоянная в противоположность скорости $c$, меняющейся вдоль потока.

В применениях общих уравнений гидродинамики особое место занимает термодинамически идеальный газ.
Говоря о таком газе, мы будем всегда (за исключением только особо оговоренных случаев) считать, что его теплоемкость является постоянной величиной, не зависящей от температуры (в интересующей нас температурной области).
Такой газ часто называют \emph{политропным}; мы будем пользоваться этим термином, имея в виду подчеркнуть каждый раз, что речь идет о предположении, идущем гораздо дальше термодинамической идеальности.
Для политропного газа известны все соотношения между термодинамическими величинами, выражающиеся к тому же весьма простыми формулами; это часто дает возможность до конца решать уравнения гидродинамики.
Выпишем здесь, для справок, эти соотношения, которыми нам неоднократно придется пользоваться в дальнейшем.


Уравнение состояния термодинамически идеального газа гласит

\begin{equation}
    \label{eq:83.8}
    pV = p/\rho = RT/\mu,
\end{equation}
где $R = 8,314 \cdot 10^7\quad erg/K\cdot mol$  - газовая постоянная, а $\mu$  - молекулярная масса газа.
Скорость звука в таком газе была вычислена в § \ref{sec:p64} и дается формулой
\begin{equation}
    \label{eq:83.9}
    c^2 = \gamma RT/\mu = \gamma p/\rho,
\end{equation}
где введено отношение теплоемкостей
\[
    \gamma = c_p/c_v.
\]
Это отношение всегда больше единицы, а для политропного газа оно постоянно.
Для одноатомных газов $\gamma = 5/3$, а для двухатомных $\gamma = 7/5$ (при обычных температурах
\footnote{Название газа "политропный" происходит от термина "политропный процесс" - процесс, в котором давление меняется обратно пропорционально некоторой степени объема.
Для газа с постоянными теплоемкостями таковым является не только изотермический, но и адиабатический процесс, для которого $pV^{\gamma}$ (адиабата Пуассона).
Отношение теплоемкостей $\gamma$ называют \emph{показателем адиабаты}.}).

Внутренняя энергия политропного газа с точностью до несущественной аддитивной постоянной равна

\begin{equation}
    \label{eq:83.10}
    \epsilon = c_v T = \frac{pV}{\gamma - 1} = \frac{c^2}{\gamma(\gamma - 1)}.
\end{equation}
Для тепловой функции имеют место аналогичные формулы
\begin{equation}
    \label{eq:83.11}
    w = c_p T = \frac{\gamma pV}{\gamma - 1} = \frac{c^2}{\gamma - 1}.
\end{equation}
Здесь учтено известное соотношение $c_p - c_v = R/\mu$. Наконец, энтропия газа
\begin{equation}
    \label{eq:83.12}
    s = c_v \ln \frac{p}{\rho^\gamma} = c_p \ln \frac{p^{1/\gamma}}{\rho}.
\end{equation}

Вернемся к изучению стационарного движения и применим полученные выше общие соотношения к политропному газу. Подставив (\ref{eq:83.11}) в (\ref{eq:83.3}), найдем, что максимальная скорость стационарного вытекания равна
\begin{equation}
    \label{eq:83.13}
    v_{max} = c_0 \sqrt{\frac{2}{\gamma - 1}}.
\end{equation}
Для критической же скорости из второго уравнения (\ref{eq:83.7}) получим:
\[
    \frac{c^2_*}{\gamma - 1} + \frac{c^2_*}{2} = w_0 = \frac{c^2_0}{\gamma - 1},
\]
откуда \footnote{
На рис. 52 дан график отношения $j/j_*$ в функции от $v/c_*$ для воздуха ($\gamma =1.4$, $v_{max} = 2,45c_*$).}

\begin{equation}
    \label{eq:83.14}
    c_* = c_0 \sqrt{\frac{2}{\gamma+1}}.
\end{equation}

Уравнение Бернулли (\ref{eq:83.1}) после подстановки выражения (\ref{eq:83.11}) для тепловой функции даст соотношение между температурой и скоростью в произвольной точке линии тока; аналогичные соотношения для давления и плотности можно затем написать с помощью уравнения адиабаты Пуассона:
\begin{equation}
    \label{eq:83.15}
    \rho=\rho_0 \left(\frac{T}{T_0}\right)^{\frac1{\gamma - 1}}, \quad
    p =p_0 \left(\frac{\rho}{\rho_0}\right)^{\frac1{\gamma}}.
\end{equation}
Таким образом, получим следующие важные формулы:

\begin{equation}
    \label{eq:83.16}
    \begin{array}{l}
    T = T_0 \left( 1 - \frac{\gamma - 1}{2} \frac{v^2}{c^2_0} \right) = 
    T_0 \left( 1 - \frac{\gamma - 1}{\gamma + 1} \frac{v^2}{c^2_*} \right),  \\
    \rho = \rho_0 \left( 1 - \frac{\gamma - 1}{2} \frac{v^2}{c^2_0} \right)^{\frac1{\gamma-1}} = 
    \rho_0 \left( 1 - \frac{\gamma - 1}{\gamma + 1} \frac{v^2}{c^2_*} \right)^{\frac1{\gamma-1}},  \\
    p = p_0 \left( 1 - \frac{\gamma - 1}{2} \frac{v^2}{c^2_0} \right)^{\frac{\gamma}{\gamma-1}} = 
    p_0 \left( 1 - \frac{\gamma - 1}{\gamma + 1} \frac{v^2}{c^2_*} \right)^{\frac{\gamma}{\gamma-1}}, 
    \end{array}
\end{equation}
Иногда удобно пользоваться этими соотношениями в виде, определяющем скорость через другие величины:
\begin{equation}
    \label{eq:83.17}
    v^2 = \frac{2\gamma}{\gamma - 1} \frac{p_0}{\rho_0}\left\lbrack 1 - \left(\frac{p}{p_0}\right)^{\frac{\gamma-1}{\gamma}}\right\rbrack =
    \frac{2\gamma}{\gamma-1} \frac{p_0}{\rho_0} \left\lbrack 1 - \left(\frac{\rho}{\rho_0}\right)^{\gamma-1} \right\rbrack.
\end{equation}
Выпишем также соотношение, связывающее скорость звука со скоростью $v$:
\begin{equation}
    \label{eq:83.18}
    c^2 = c^2_0 - \frac{\gamma-1}{2}v = \frac{\gamma+1}{2} c^2_* - \frac{\gamma-1}{2} v^2.
\end{equation}
Отсюда найдем, что числа $M$ и $M_*$ связаны друг с другом посредством
\begin{equation}
    \label{eq:83.19}
    M^2_* = \frac{\gamma+1}{2/M^2+\gamma-1}.
\end{equation}
Когда $M$ растет от 0 до $\infty, M^2_*$ растет от 0 до $(\gamma + 1)/(\gamma - 1)$.

Наконец, приведем выражения для критических значений температуры, давления и плотности; они получаются при
$v = c_*$, из формул (\ref{eq:83.16}):
\begin{equation}
    \label{eq:83.20}
    T_*=\frac{2T_0}{\gamma+1}, \quad
    p_* = p_0 \left( \frac{2}{\gamma+1} \right)^\frac{\gamma}{\gamma-1}, \quad
    \rho_* = \rho_0  \left( \frac{2}{\gamma+1} \right)^\frac{1}{\gamma-1}.
\end{equation}


Подчеркнем в заключение, что полученные здесь результаты относятся к движению, при котором не возникают ударные волны. При наличии ударных волн не имеет места уравнение (\ref{eq:83.2}): при прохождении линии тока через ударную волну энтропия газа возрастает.

Мы увидим, однако, что уравнение Бернулли (\ref{eq:83.1}) остается справедливым и при наличии ударной волны, так как $w+v^2/2$ является как раз одной из величин, сохраняющихся при прохождении через поверхность разрыва (\S\ref{sec:p85}); вместе с ним остается, например, справедливой и формула (\ref{eq:83.14}).

\subsection*{Задача}
Выразить температуру, давление и плотность вдоль линии тока через число $M=v/c$.

\texttt{Решение.} С помощью полученных в тексте формул получим

\begin{eqnarray}
    \frac{T_0}{T} = 1 + \frac{\gamma-1}{2} M^2, \quad
    \frac{p_0}{p} = \left\lbrack 1 + \frac{\gamma-1}{2} M^2 \right\rbrack^\frac{\gamma}{\gamma - 1}, \nonumber \\
    \frac{\rho_0}{\rho} = \left\lbrack 1 + \frac{\gamma-1}{2} M^2 \right\rbrack^\frac{1}{\gamma - 1}. \nonumber
\end{eqnarray}

\begin{comment}




\end{comment}
