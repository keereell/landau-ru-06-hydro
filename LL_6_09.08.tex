\section{Ударные волны в политропном газе}\label{sec:p89}


Применим полученные в предыдущих параграфах общие соотношения к ударным волнам
в политропном газе.

Тепловая функция такого газа дается простой формулой (\ref{eq:83.11}). Подставив это
выражение в (\ref{eq:85.9}), получим после про-стого преобразования р2/р1 следующую
формулу:

(\ref{eq:89.1})

По этой формуле можно определить по трем из величин ри Vx, р2, Vi четвертую.
Отношение Vi/V\ является монотонно убывающей функцией отношения pi/pi,
стремящейся к конечному пределу (Y— 1)/(у+1). Кривая, изображающая зависимость
между pi и Vi при заданных ри V\ (ударная адиабата), представлена на рис. 58*.
Это — равнобочная гипербола с асимптотами


') Во всех перечисленных на рис. 57 неэволюционных случаях возмущение
недоопределеио — число произвольных параметров превышает число уравнений.
Упомянем, что в магнитной гидродинамике ударные волны могут быть
неэволюционными в силу как недоопределенности, так и переопределенности
возмущений (см. VIII, § 73).


Реальным смыслом обладает, как мы знаем, только верхняя часть кривой над точкой
V2/Vi = p2/pi = 1, изображенная на рис. 58 (для у = 1,4) сплошной линией.

Для отношения температур с обеих сторон разрыва имеем согласно уравнению
состояния термодинамически идеального газа Т2/Ту = p2V2/pVi, так что

Для потока / получаем из (\ref{eq:85.6}) и (\ref{eq:89.1}):


и отсюда для скорости распространения ударной волны относительно газов впереди
и позади нее:


и для разности скоростей:


В применениях полезны формулы, выражающие отношения плотностей, давлений и
температур в ударной волне через число Mi = vi/cr, эти формулы без труда
выводятся из полученных выше соотношений и гласят:


Число же M2 = v2/c2 выражается через число Mj	посредством


(\ref{eq:89.9})

Это соотношение симметрично относительно Mi и М2, как это •становится
очевидным, если записать его в виде уравнения



Выпишем предельные формулы для ударных волн очень большой интенсивности
(требуется, чтобы бьмо (7 — > (у— 1 )pi). Имеем из (89,1—2):


Отношение T2/Ti неограниченно растет вместе с p2/pi, т. е. скачок температуры,
как и скачок'давления, в ударной волне может быть сколь угодно большим.
Отношение же плотностей стремится к постоянному пределу; так, для одноатомного
газа предельное значение р2 = 4рь для двухатомного р2 = брь Скорости
распространения ударной волны большой интенсивности равны


Они растут пропорционально корню из давления р2.

Наконец, приведем соотношения для ударных волн слабой интенсивности,
представляющие собой первые члены разложений по степеням малого отношения z =s
(р2— Pi)/Pi'-

Здесь сохранены члены, дающие первую поправку к значениям, отвечающим звуковому
приближению.

Задачи

1. Получить формулу


где с* — критическая скорость (L. Prandtl).

Решение. Поскольку величина w + i>2/2 непрерывна на ударной волне, можио ввести
критическую скорость, одинаковую для газов 1 н 2 согласно-



(ср. (\ref{eq:83.7})). Определяя из этих равенств р2/р2 и р,/р, и подставляя их в~
уравнение


(результат комбинирования (\ref{eq:85.1}) и (\ref{eq:85.2})), получим Y +

 Ввиду того что vt Ф кг, отсюда следует искомое соотношение.


2. Определить отношение рг/pi по заданным температурам Ti, Т2 для ударной волны
в термодинамически идеальном газе с непостоянной теплоемкостью.

Решение. Для такого газа можно лишь утверждать, что w (как и е) есть функции
только от температуры и что р, V, Т связаны уравнением состояния pV = RT/\i.
Решая уравнение (\ref{eq:85.9}) относительно рг/pi, получаем:



где = w(Т), w2 — w(T2).
\begin{comment}
\end{comment}

