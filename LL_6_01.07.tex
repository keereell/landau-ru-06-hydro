\section{Поток импульса}
\label{sec:p7}

Произведем теперь аналогичный вывод для импульса жидкости. Импульс единицы
объема жидкости есть $\rho \vv$. Определим скорость его изменения:
\[
   \pdt \rho \vv
\]

Будем  производить вычисления  в тензорных обозначениях. Имеем:
\[
   \pdt \rho v_i = \rho \pd{v_i}{t} + \pd{\rho}{t} v_i .
\]
Воспользуемся  уравнением   непрерывности   (\ref{eq:1.2}),   написав  его в виде
\[
   \pd{\rho}{t} = - \pd{(\rho v_k)}{x_k}
\]
и уравнением Эйлера (\ref{eq:2.3}) в форме
\[
   \pd{v_i}{t} = -v_k \pd{v_i}{x_k} - \frac1{\rho}\pd{p}{x_i}
\]
Тогда получим:
\[
   \pdt \rho v_i = - \rho v_k \pd{v_i}{x_k} - \pd{p}{x_i}
    -  v_i \pd{(\rho v_k)}{x_k} = - \pd{p}{x_i} - \pd{\rho v_i v_k}{x_k}
\]
Первый член справа напишем в виде
\[
   \pd{p}{x_i} = \delta_{ik}\pd{p}{x_k}
\]
и находим окончательно:
\begin{equation}
   \label{eq:7.1}
   \pdt \rho v_i = - \pd{\Pi_{ik}}{x_k},
\end{equation}
где тензор $\Pi_{ik}$ определяется как
\begin{equation}
   \label{eq:7.2}
   \Pi_{ik} = p \delta_{ik} + \rho v_i v_k .
\end{equation}
Он, очевидно, симметричен.

Для выяснения смысла тензора $\Pi_{ik}$ проинтегрируем уравнение (\ref{eq:7.1}) по
некоторому объему:
\[
   \pdt \int{\rho v_i\; dV} = - \int{\pd{\Pi_{ik}}{x_k}\; dV}
\]
Стоящий в правой стороне равенства интеграл преобразуем в интеграл по
поверхности \footnote{Правило преобразования интеграла по замкнутой поверхности
в интеграл по охватываемому этой поверхности объему можно сформулировать следующим
образом: оно осуществляется заменой элемента поверхности $df_i$ оператором $dV \frac{\partial}{\partial x_i}$, который должен быть применен ко всему подинтегральному выражению
\[
    df_i \rightarrow dV \frac{\partial}{\partial x_i}
\]
}:
\begin{equation}
   \label{eq:7.3}
   \pdt \int{\rho v_i\; dV} = - \oint{\Pi_{ik}\; df_k}
\end{equation}

Слева стоит изменение в единицу времени $i$-й компоненты импульса в
рассматриваемом объеме. Поэтому стоящий справа интеграл по поверхности есть
количество этого импульса, вытекающего в единицу времени через ограничивающую
объем поверхность. Следовательно, $\Pi_{ik}\; df_k$ есть $i$-я компонента
импульса, протекающего через элемент $df$ поверхности. Если написать $df_k$ в
виде $n_k df$ - абсолютная величина элемента поверхности, $\vect n$ - единичный
вектор внешней нормали к нему), то мы найдем, что $\Pi_{ik}n_k$ есть поток $i$-й
компоненты импульса, отнесенный к единице площади поверхности. Заметим, что
согласно (\ref{eq:7.2}) $\Pi_{ik}n_k = pn_i + \rho v_i v_k n_k$; это вырзжение может быть
написано в векторном виде как
\begin{equation}
   \label{eq:7.4}
   p \vect n + \rho \vect v (\vect{vn}).
\end{equation}

Таким образом, $\Pi_{ik}$ есть $i$-я компонента количества импульса,
протекающего в единицу времени через единицу поверхности, перпендикулярную к оси
$x_k$. Тензор $\Pi_{ik}$ называют \textit{тензором плотности потока импульса.}
Поток энергии, являющейся скалярной величиной, определяется вектором; поток же
импульса, который сам есть вектор, определяется тензором второго ранга.

Вектор (\ref{eq:7.4}) определяет поток вектора импульса в направлении п, т. е. через
поверхность, перпендикулярную к $\vect n$. В частности, выбирая направление
единичного вектора $\vect n$ вдоль направления скорости жидкости, мы найдем, что
в этом направлении переносится лишь продольная компонента импульса, причем
плотность ее потока равна
\[
   p + \rho v^2.
\]
В направлении же, перпендикулярном к скорости, переносится лишь поперечная
(по отношению к $\vv$) компонента импульса, а плотность ее потока равна просто
$p$.
