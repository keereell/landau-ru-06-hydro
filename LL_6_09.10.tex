\section{Распространение ударной волны по трубе}\label{sec:p91}

Рассмотрим распространение ударной волны по среде, заполняющей длинную трубку с
переменным сечением. Наша цель состоит при этом в выяснении влияния,
оказываемого изменением площади ударной волны на ее скорость (G. В. Whitham,
1958).

Будем считать, что площадь S(x) сечения трубки лишь медленно меняется вдоль ее
длины (ось л;) — мало на расстояниях

') Производная (дУг!дрг)н есть то, что мы обозначали выше просто как dVi/api,
подразумевая, что производная берется при постоянных pi, Vi.


порядка ширины трубки. Это дает возможность применить приближение (его называют
гидравлическим), которое уже было использовано в § 77: можно считать все
величины в потоке постоянными вдоль каждого поперечного сечения трубки, а
скорость—направленной вдоль ее оси; другими словами, течение рассматривается
как квазиодномерное. Такое течение описывается уравнениями

Первое из них — уравнение Эйлера, второе — уравнение адиаба-тичности, а третье
— уравнение непрерывности, представленное в виде (\ref{eq:77.1}).

Для выяснения интересующего нас вопроса достаточно рассмотреть трубку, в
которой изменение площади Sx не только медленно, но и по абсолютной величине
остается относительно малым на протяжении всей длины. Тогда будут малы и
связанные с непостоянством сечения возмущения потока, и уравнения (91,1—3)
могут быть линеаризованы. Наконец, должны быть поставлены начальные условия,
исключающие появление каких-либо посторонних возмущений, которые могли бы
повлиять на движение ударной волны; нас интересуют только возмущения, связанные
с изменением S(x). Эта цель будет достигнута, если принять, что ударная волна
первоначально движется с постоянной скоростью по трубе постоянного сечения, и
площадь сечения начинает меняться только вправо от некоторой точки (которую
примем за х = 0).

Линеаризованные уравнения (91,1—3) имеют вид



где буквы без индекса обозначают постоянные значения величин в однородном
потоке в однородной части трубки, а символ б обозначает изменение этих величин
в трубке переменного сечения. Умножив первое и третье из этих уравнений
соответственно на ре и с2 и сложив затем все три уравнения, напишем следующую
их комбинацию:

Общее решение этого уравнения дается суммой общего решения однородного
уравнения и частного решения уравнения с правой частью. Первое есть F(x — vt —
ct), где F — произвольная функция; оно описывает звуковые возмущения,
приходящие слева. Но в однородной области, при х < 0, возмущений нет; поэтому
надо положить F = 0. Таким образом, решение сводится к интегралу неоднородного
уравнения:

(\ref{eq:91.5})

Ударная волна движется слева направо со скоростью vi > С по неподвижной среде
с заданными значениями ри р:. Движение же среды позади ударной волны (среда 2)
определяется решением (\ref{eq:91.5}) во всей области трубки слева от точки, достигнутой
разрывом к данному моменту времени. После прохождения волны все величины в
каждом сечении трубки остаются постоянными во времени, т. е. равными тем
значениям, которые qhh получили в момент прохождения разрыва: давление р2,
плотность р2 и скорость v — v2 (в соответствии с принятыми в этой главе
обозначениями, v2 обозначает скорость газа относительно движущейся ударной
волны; скорость же его относительно стенок трубки есть тогда V — v2). В этих
обозначениях (и снова выделив переменные части этих величин) равенство (\ref{eq:91.5})
запишем в виде


Все величины 6уь 6у2, бр2 можно выразить через одну из них, например, биь Для
этого пишем варьированные соотношения (85,1—2) на разрыве (при заданных р, и
pi):


(где / = pit»i = р2и2 — невозмущенное значение потока); к ним надо еще
присоединить соотношение

где производная берется вдоль адиабаты Гюгонио. Вычисление приводит к
следующему окончательному соотношению, связывающему Изменение 6ui скорости
ударной волны относительно неподвижного газа перед ней, с изменением 65 площади
сечения трубки:


где снова введено обозначение

(\ref{eq:91.8})


Коэффициент при квадратной скобке в (\ref{eq:91.7}) положителен. Поэтому знак отношения
8vi/8S определяется знаком выражения в этой скобке. Для всех устойчивых ударных
волн этот знак положителен, так что 601/6S < 0. Но при выполнении какого-либо
из условий (90,12—13) гофрировоч-ной неустойчивости выражение в скобках
становится отрицательным, так что 8vi/8S > 0.

Этот результат дает возможность наглядного истолкования происхождения
неустойчивости. На рис. 62 изображена «гофрированная» поверхность ударной
волны, перемещающаяся направо; стрелками схематически показано направление
линий тока. При перемещении ударной волны на выдавшихся вперед участках
поверхности площадь 6S растет, а на отставших участках— уменьшается. При
6i>i/6S < 0 это приводит к замедлению выступивших участков и ускорению
отставших, так что поверхность стре- Рис. 62 мится сгладиться. Напротив, при
6vi/8S >0 возмущение формы поверхности будет усиливаться: выступающие участки
будут уходить все дальше, а отставшие — все более отставать').
