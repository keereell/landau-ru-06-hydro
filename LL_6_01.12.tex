\section{Гравитационные волны}
\label{sec:12}

Свободная поверхность жидкости, находящейся в равновесии в поле тяжести, -
плоская. Если под влиянием какого-либо внешнего воздействия поверхность жидкости
в каком-нибудь месте выводится из ее равновесного положения, то в жидкости
возникает движение. Это движение будет распространяться вдоль всей поверхности
жидкости в виде волн, называемых \textit{гравитационными}, поскольку они
обусловливаются действием поля тяжести. Гравитационные волны происходят в
основном на поверхности жидкости, захватывая внутренние ее слои тем меньше, чем
глубже эти слои расположены.

Мы будем рассматривать здесь такие гравитационные волны, в которых скорость
движущихся частиц жидкости настолько мала, что в уравнении Эйлера можно
пренебречь членом $\vnv$ по сравнению с $\partial \vv/\partial t$. Легко
выяснить, что означает это условие физически. В течение промежутка времени
порядка периода $\tau$ колебаний, совершаемых частицами жидкости в волне, эти
частицы проходят расстояние порядка амплитуды $a$ волны. Поэтому скорость их
движения — порядка $v \sim a/\tau$. Скорость $v$ заметно меняется на протяжении
интервалов времени порядка $\tau$ и на протяжении расстояний порядка $\lambda$
вдоль направления распространения волны ($\lambda$ - длина волны). Поэтому
производная от скорости по времени — порядка $v/\tau$, а по координатам —
порядка $v/\lambda$. Таким образом, условие $\vnv \ll \partial \vv/\partial t$
эквивалентно требованию
\[
   \frac1{\lambda} \left( \frac{a}{\tau} \right)^2 \ll \frac{a}{\tau}\frac1{\tau},
\]
или
\begin{equation}
 \label{eq:12.1}
    a \ll \lambda,
\end{equation}
т. е. амплитуда колебаний в волне должна быть мала по сравнению с длиной волны.
В \S9 мы видели, что если в уравнении движения можно пренебречь членом $\vnv$,
то движение жидкости потенциально. Предполагая жидкость несжимаемой, мы можем
воспользоваться поэтому уравнениями (\ref{eq:10.6}) и (\ref{eq:10.7}). В уравнении (\ref{eq:10.7}) мы можем
теперь пренебречь членом $\vnv$, содержащим квадрат скорости; положив $f(t) = 0$
и введя в поле тяжести член $pgz$, получим:
\begin{equation}
 \label{eq:12.2}
  p = - \rho gz - \rho \pd{\varphi}{t}.
\end{equation}
Ось z выбираем, как обычно, вертикально вверх, а в качестве плоскости $x,y$
выбираем равновесную плоскую поверхность жидкости.

Будем обозначать $z$-координату точек поверхности жидкости посредством
$\zeta$;$\zeta$ является функцией координат $x,y$ и времени $t$. В равновесии
$\zeta=0$, так что $\zeta$ есть вертикальное смещение жидкой поверхности при ее
колебаниях. Пусть на поверхность жидкости действует постоянное давление $p_0$.
Тогда имеем на поверхности согласно (12,2)
\[
   p_0 = - \rho g \zeta - \rho \pd{\varphi}{t}.
\]
Постоянную $p_0$ можно устранить переопределением потенциала $\varphi$
(прибавлением к нему независящей от координат величины $p_0t/\rho$). Тогда
условие на поверхности жидкости примет вид
\begin{equation}
 \label{eq:12.3}
       g \zeta + \left.\pd{\varphi}{t}\right\vert_{z=\zeta} = 0.
\end{equation}
Малость амплитуды колебаний в волне означает, что смещение $\zeta$ мало. Поэтому
можно считать, в том же приближении, что вертикальная компонента скорости
движения точек поверхности совпадает с производной по времени от смещения
$\zeta$: $v_z=\partial$. Но $v_z=\partial\varphi/\partial z$, так что имеем:
\[
   \left.\pd{\varphi}{t}\right\vert_{z=\zeta} = \pd{\zeta}{t} = - \frac1{g}
   \left.\frac{\partial^2\varphi}{\partial t^2}\right\vert_{z=\zeta}.
\]

В силу малости колебаний можно в этом условии взять значения производных при
$z=0$ вместо $z=\zeta$. Таким образом, получаем окончательно следующую систему
уравнений, определяющих движение в гравитационной волне:
\begin{eqnarray}
 \label{eq:12.4}
   \nabla\varphi = 0, \\
 \label{eq:12.5}
   \left( \pd{\varphi}{t} +
   \frac1{g} \frac{\partial^2 \varphi}{\partial t^2} \right)_{z=0} = 0.
\end{eqnarray}

Будем рассматривать волны на поверхности жидкости, считая эту поверхность,
неограниченной. Будем также считать, что длина волны мала по сравнению с
глубиной жидкости; тогда можно рассматривать жидкость как бесконечно глубокую.
Поэтому мы не пишем граничных условий на боковых границах и на дне жидкости.

Рассмотрим гравитационную волну, распространяющуюся вдоль оси $x$ и однородную
вдоль оси $y$ в такой волне все величины не зависят от координаты $y$. Будем
искать решение, являющееся простой периодической функцией времени и координаты
$x$:
\[
   \varphi = \cos (kx - \omega t)f(z),
\]
где ($\omega$ - циклическая частота (мы будем говорить о ней просто как о
частоте), $k$ - волновой вектор волны, $\lambda = 2\pi/k$ - длина волны.
Подставив это выражение в уравнение $\nabla\varphi = 0$, получим для функции
$f(z)$ уравнение
\[
   \frac{d^2f}{dz^2} - k^2f = 0.
\]

Его решение, затухающее в глубь жидкости (т.е. при $z \to - \infty$):
\begin{equation}
 \label{eq:12.6}
   \varphi = Ae^{kz}\cos (kx - \omega t).
\end{equation}

Мы должны еще удовлетворить граничному условию (\ref{eq:12.5}). Подставив в него (\ref{eq:12.5}),
найдем связь между частотой и волновым вектором (или, как говорят,\textit{закон
дисперсии волн}):
\begin{equation}
 \label{eq:12.7}
   \omega^2 = kg.
\end{equation}

Распределение скоростей в жидкости получается дифференцированием потенциала по
координатам:

\begin{equation}
 \label{eq:12.8}
\begin{array}{l}
   v_x = -Ake^{kz}\sin (kx - \omega t),\; \\
   v_z =  Ake^{kz}\cos (kx - \omega t).
\end{array}
\end{equation}
Мы видим, что скорость экспоненциально падает по направлению в глубь жидкости. В
каждой заданной точке пространства (т. е. при заданных $x,z$) вектор скорости
равномерно вращается в плоскости $x,z$, оставаясь постоянным по своей величине.

Определим еще траекторию частиц, жидкости в волне. Обозначим временно
посредством $x,z$ координаты движущейся частицы жидкости (а не координаты
неподвижной точки в пространстве), а посредством $x_0,z_0$ — значения $x,z$ для
равновесного положения частицы. Тогда $v_x = dx/dt,\; v_z = dz/dt$, а в правой
части (\ref{eq:12.8}) можно приближенно написать $x_0, z_0$ вместо $x,z$,
воспользовавшись малостью колебаний. Интегрирование по времени дает тогда:
\begin{equation}
 \label{eq:12.9}
\begin{array}{l}
   x - x_0 = - A \frac{k}{\omega}e^{kz_0} \cos (kx_0 - \omega t); \\
   z - z_0 = - A \frac{k}{\omega}e^{kz_0} \sin (kx_0 - \omega t).
\end{array}
\end{equation}
Таким образом, частицы жидкости описывают окружности вокруг точек $x_0, z_0$ с
радиусом, экспоненциально убывающим по направлению в глубь жидкости.

Скорость $U$ распространения волны равна, как будет показано в \S67, $U =
\partial \omega/\partial k$. Подставив сюда $\omega = \sqrt{kg}$, находим, что
скорость распространения гравитационных волн на неограниченной поверхности
бесконечно глубокой жидкости равна
\begin{equation}
 \label{eq:12.10}
   U = \frac{1}{2}\sqrt{\frac{g}{k}}=\frac{1}{2}\sqrt{\frac{g\lambda}{2\pi}}.
\end{equation}
Она растет при увеличении длины волны.

\subsection*{Длинные гравитационные волны}

Рассмотрев гравитационные волны, длина которых мала по сравнению с глубиной
жидкости, остановимся теперь на противоположном предельном случае волн, длина
которых велика по сравнению с глубиной жидкости. Такие волны называются
\textit{длинными}.

Рассмотрим сначала распространение длинных волн в канале. Длину канала
(направленную вдоль оси $x$) будем считать неограниченной Сечение канала может
иметь произвольную форму и может меняться вдоль его длины. Площадь поперечного
сечения жидкости в канале обозначим посредством $S = S(x,t)$. Глубина и ширина
канала предполагаются малыми по сравнению с длиной волны.

Мы будем рассматривать здесь продольные длинные волны, в которых жидкость
движется вдоль канала. В таких волнах компонента $v_x$ скорости вдоль длины
канала велика по сравнению с компонентами $v_y, v_z$.

Обозначив $v_x$, просто как $c$ и опуская малые члены, мы можем написать
$x$-компоненту уравнения Эйлера в виде
\[
   \pd{v}{t} = - \frac1{\rho} \pd{p}{x},
\]
а $z$ - компоненту - в виде
\[
   \frac1{\rho} \pd{p}{z} = -g
\]
(квадратичные по скорости члены опускаем, поскольку амплитуда волны по-прежнему
считается малой). Из второго уравнения имеем, замечая, что на свободной
поверхности ($z = \zeta$) должно быть $p = p_0$:
\[
   p = p_0 + g\rho(\zeta - z).
\]
Подставляя это выражение в первое уравнение, получаем:
\begin{equation}
 \label{eq:12.11}
   \pd{v}{t} = - g \pd{\zeta}{x}.
\end{equation}

Второе уравнение для определения двух неизвестных $v$ и $\zeta$ можно вывести
методом, аналогичным выводу уравнения непрерывности. Это уравнение представляет
собой по существу уравнение непрерывности применительно к рассматриваемому
случаю. Рассмотрим объем жидкости, заключенный между двумя плоскостями
поперечного сечения канала, находящимися на расстоянии $dx$ друг от друга. За
единицу времени через одну плоскость войдет объем жидкости, равный $(Sv)_x$, а
через другую плоскость выйдет объем $(Sv)_{x+dx}$. Поэтому объем жидкости между
обеими плоскостями изменится на
\[
   (Sv)_{x+dx} - (Sv)_x = \pd{(Sv)}{x}dx.
\]
Но в силу несжимаемости жидкости это изменение может произойти только за счет
изменения ее уровня. Изменение объема жидкости между рассматриваемыми
плоскостями в единицу времени равно
\[
   \pd{S}{t}dx.
\]

Следовательно, можно написать:
\[
   \pd{S}{t}dx = - \pd{(Sv)}{x}dx,
\]
или
\begin{equation}
\label{eq:12.12}
       \pd{S}{t}+ \pd{(Sv)}{x} = 0,
\end{equation}
Это и есть искомое уравнение непрерывности.

Пусть $S_0$ есть площадь поперечного сечения жидкости в канале при равновесии.
Тогда $S = S_0 + S'$, где $S'$ — изменение этой площади благодаря наличию волны.
Поскольку изменение уровня жидкости в волне мало, то $S'$ можно написать в виде
$b\zeta$ где $b$ — ширина сечения канала у самой поверхности жидкости в нем.
Уравнение (\ref{eq:12.12}) приобретает тогда вид
\begin{equation}
    \label{eq:12.13}
    b \pd{\zeta}{t} + \pd{(S_0v)}{x} = 0.
\end{equation}
Дифференцируя (\ref{eq:12.13}) по $t$ и подставляя $\pd{v}{t}$ из (\ref{eq:12.11}), получим:
\begin{equation}
\label{eq:12.14}
   \frac{\partial^2\zeta}{\partial t} - \frac{g}{b} \frac{\partial}{\partial x}
   \left( S_0 \pd{\zeta}{x} \right) = 0.
\end{equation}
Если сечение канала одинаково вдоль всей его длины, то $S_0 = \const$ и
\begin{equation}
\label{eq:12.15}
 \frac{\partial^2\zeta}{\partial t} - \frac{gS_0}{b} \frac{\partial^2\zeta}{\partial x^2} = 0.
\end{equation}
Уравнение такого вида называется \textit{волновым}; как будет показано в \S64,
оно соответствует распространению волн с не зависящей от частоты скоростью $U$,
равной квадратному корню из коэффициента при $\partial^2\zeta/\partial x^2$.
Таким образом, скорость распространения длинных гравитационных волн в каналах
равна
\begin{equation}
\label{eq:12.16}
   U=\sqrt{\frac{gS_0}{b}}.
\end{equation}
Аналогичным образом можно рассмотреть длинные волны в обширном бассейне, который
мы будем считать неограниченным в двух измерениях (вдоль плоскости $x,y$).
Глубину жидкости в бассейне обозначим посредством $h$. Из трех компонент
скорости малой является теперь компонента $v_z$. Уравнения Эйлера приобретают
вид, аналогичный (\ref{eq:12.11}):
\begin{equation}
\label{eq:12.17}
   \pd{v_x}{t}+g \pd{\zeta}{x} = 0; \\
   \pd{v_y}{t}+g \pd{\zeta}{y} = 0.
\end{equation}
Уравнение непрерывности выводится аналогично (12,12) и имеет вид
\[
   \pd{h}{t} + \pd{(hv_x)}{x} + \pd{hv_y}{y} = 0.
\]
Глубину $h$ пишем в виде $h = h_0 + \zeta$, где $h_0$ - равновесная глубина. Тогда
\begin{equation}
\label{eq:12.18}
      \pd{\zeta}{t} + \pd{(h_0v_x)}{x} + \pd{h_0v_y}{y} = 0.
\end{equation}

Предположим, что бассейн имеет плоское горизонтальное дно ($h_0 = \const$).
Дифференцируя (\ref{eq:12.18}) по $t$ и подставляя (\ref{eq:12.17}), лолучим:
\begin{equation}
\label{eq:12.19}
   \pd{^2\zeta}{t^2} - gh_0 \left( \pd{^2\zeta}{x^2} + \pd{^2\zeta}{y^2} \right) = 0.
\end{equation}
Это — опять уравнение типа волнового (двухмерного) уравнения; оно соответствует
волнам со скоростью распространения, равной
\begin{equation}
\label{eq:12.20}
   U = \sqrt{gh_0}.
\end{equation}


\subsection*{Задачи}
\paragraph*{1}
Определить скорость распространения гравитационных волн на неограниченной
поверхиости жидкости, глубина которой равна $h$.

\texttt{Решение.}
На дне жидкости нормальная составляющая скорости должна быть равна нулю, т. е.
\[
   v_z = \pd{\varphi}{z} = 0, \text{ при } z = -h.
\]
Из этого условия определяется отношение между постоянными $A$ и $B$ в общем решении
\[
   \varphi = \cos (kx - \omega t) \lbrace Ae^{kz} + Be^{-kz} \rbrace.
\]
В результате находим:
\[
   \varphi = A\cos (kz - \omega t) \cosh k(z+h).
\]
Из предельного условия (\ref{eq:12.5}) находим соотношение между $k$ и $\omega$ в виде
\[
   \omega^2 = gk \tanh kh.
\]
Скорость распространения волны
\[
   U = \frac{\sqrt g}{2\sqrt{k\tanh kh}} \left\lbrace \tanh kh + \frac{kh}{\cosh^2 kh} \right\rbrace .
\]

При $kh \gg 1$ получается результат (\ref{eq:12.10}), а при $kh \ll 1$ - результат (\ref{eq:12.20}).


\paragraph*{2} Определить связь между частотой и длиной волны для гравитационных
волн на поверхности раздела двух жидкостей, причем верхняя жидкость ограничена
сверху, а нижняя — снизу горизонтальными неподвижными плоскостями. Плотность н
глубина слоя нижней жидкости $\rho$ и $h$, а верхней $\rho'$ и $h'$ (причем
$\rho > \rho'$).

\textit{Решение.} Плоскость $x,y$ выбираем по плоскости раздела обеих жидкостей
в равновесии. Ищем решение в обеих жидкостях соответственно в виде
\begin{equation}
\label{eq:12t2.1}
\begin{array}{l}
   \varphi  = A\cosh k(z+h ) \cos(kx - \omega t),\\
   \varphi' = B\cosh k(z-h') \cos(kx - \omega t)
\end{array}
\end{equation}
(так, чтобы удовлетворялись условия на верхней и нижней границах, — см. решение
задачи 1). На поверхности раздела давление должно быть непрерывным; согласно
(\ref{eq:12.2}) это приводит к условию

\[
   \rho g \zeta + \rho \frac{\varphi}{t} = \rho' g\zeta + \rho'\pd{\varphi'}{t}
\]
(при $z = 0$) или
\begin{equation}
\label{eq:12t2.2}
   \zeta = \frac1{g(\rho - \rho')}\left( \rho'\pd{\varphi'}{t} - \rho\pd{\varphi}{t} \right) .
\end{equation}
Кроме того, скорости $v_z$ обеих жидкостей на поверхности раздела должны быть
одинаковыми. Это приводит к условию (при $z = 0$)
\begin{equation}
\label{eq:12t2.3}
   \pd{\varphi}{z} = \pd{\varphi'}{z}.
\end{equation}
Далее, $v_z = \partial \varphi/\partial z = \partial \zeta/\partial t$ и,
подставляя сюда (\ref{eq:12t2.2}), получаем:
\begin{equation}
\label{eq:12t2.4}
   g(\rho - \rho')\pd{\varphi}{z} = \rho' \pd{^2\varphi'}{t^2} - \rho\pd{^2\varphi}{t^2}.
\end{equation}
Подставляя (\ref{eq:12t2.1}) в (\ref{eq:12t2.3}) и (\ref{eq:12t2.4}), получим два однородных линейных уравнения для $A$ и
$B$, из условия совместности которых найдем:
\[
   \omega^2 = \frac{kg(\rho - \rho')}{\rho \tanh kh + \rho' \tanh kh'}.
\]
При $kh \gg 1$, $kh' \gg 1$ (обе жидкости очень глубоки):

\[
   \omega^2 = kg \frac{\rho - \rho'}{\rho + \rho'},
\]
а при $kh \ll 1$, $kh' \ll 1$; 1 (длинные волны):
\[
   \omega^2 = k^2 \frac{g(\rho - \rho')hh'}{\rho h' + \rho' h}
\]
Наконец, если $kh \geq 1$, $kh' \ll 1$:
\[
   \omega^2 = k^2 gh'\frac{\rho - \rho'}{\rho}
\]



\paragraph*{3} Определить связь между частотой и длиной волны для гравитационных
волн, распространяющихся одновременно по поверхности раздела и верхней
поверхности двух слоев жидкости, из которых нижняя (плотность $\rho$) бесконечно
глубока, а верхняя (плотность $\rho'$) имеет толщину $h'$ и свободную верхнюю
поверхность.

\textit{Решение.}
Выбираем плоскость $x,y$ в плоскости раздела обеих жидкостей в равновесии. В
нижней и верхней жидкостях ищем решение соответственно в виде
\begin{equation}
\label{eq:12t3.1}
   \varphi  = A e^{kz} \cos (kz - \omega t); \;
   \varphi' = \lbrack B e^{-kz} + C e^{kz} \cos (kz - \omega t)
\end{equation}
На поверхности раздела обеих жидкостей (т. е. при $z = 0$) имеют место условия
(см. задачу 2):
\begin{equation}
\label{eq:12t3.2}
   \pd{\varphi}{z} = \pd{\varphi'}{z};\;
   g(\rho - \rho') \pd{\varphi}{z} = \rho'\pd{^2\varphi'}{t^2} - \rho\pd{^2\varphi}{t^2},
\end{equation}
а на верхней свободной границе (т.е. при $z = h'$):
\begin{equation}
\label{eq:12t3.3}
\pd{\varphi'}{z} + \frac1{g}\pd{^2\varphi'}{t^2} = 0.
\end{equation}
Первое из уравнений (\ref{eq:12t3.2}) при подстановке (\ref{eq:12t3.1}) дает $A = C - B$, а два остальных
условия дают два уравнения для $B$ и $C$, из условия совместности которых
получаем квадратное уравнение для $\omega^2$ с корнями:
\[
   \omega^2 = kg \frac{(\rho - \rho')(1-e^{-2kh'})}{\rho + \rho' + (\rho - \rho')e^{-2kh'}},\; \omega^2 = kg.
\]
При $h' \to \infty$ эти корни соответствуют волнам, распространяющимся
независимо по поверхности раздела и по верхней поверхности жидкости.



\paragraph*{4} Определить собственные частоты колебаний (см. \S69) жидкости
глубины $h$ в прямоугольном бассейне ширины $a$ и длииы $b$.

\textit{Решение.} Оси $x$ и $y$ выбираем по двум боковым сторонам бассейна. Ищем
решение в виде стоячей волны:
\[
   \varphi = \cos \omega t \cosh k(z+h) f(x,y).
\]
Для $f$ получаем уравнение
\[
   \frac{\partial^2 f}{\partial x^2} + \frac{\partial^2 f}{\partial y^2} + k^2 f = 0,
\]
а условие на свободной поверхности приводят, как и в задаче 1, к соотношению
\[
   \omega^2 = gk \tanh kh.
\]
Решение уравнения для $f$ берем в виде
\[
   f = \cos px \cos qy, \; p^2 + q^2 = k^2.
\]
На боковых сторонах сосуда должны выполняться условия:
\begin{eqnarray}
   v_x = \pd{\varphi}{x} = 0\;& x = 0,a; \nonumber \\
         \pd{\varphi}{y} = 0\;& y = 0,b. \nonumber
\end{eqnarray}
Отсюда находим:
\[
   p = \frac{m\pi}{a},\; q = \frac{n\pi}{b},
\]
где $m,n$ - целые числа. Поэтому возможные значения $k$ равны
\[
   k^2 = \pi^2 \left( \frac{m^2}{a^2} + \frac{n^2}{b^2} \right).
\]


