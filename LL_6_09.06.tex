\section{Направление и изменение величин в ударной волне}\label{sec:p87}

Таким образом, в предположении положительности производной \ref{eq:86.2} для
ударных волн слабой интенсивности можно весьма просто показать, что условие
возрастания энтропии с необходимостью приводит также и к неравенствам
\begin{eqnarray}
\label{eq:87.1}
	p_2 > p_1 & \\
\label{eq:87.2}
	v_1 > c_1, \quad v_2 < c_2.
\end{eqnarray}

Из замечания, сделанного по поводу	формулы \ref{eq:85.6} следует, что если $p_2>p_1$ то
\begin{equation}
	\label{eq:87.3}
	V_2 < V_1,
\end{equation}
а поскольку $j = v_1/V_1 = v_2/V_2$, то и \footnote{Если перейти в систему
отсчета, в которой газ $1$ перед ударной волной покоится, а волна движется, то
неравенство $v_1 > v_2$ означает, что газ позади ударной волиы будет двигаться (со
скоростью $v_1 - v_2$) в ту же сторону, куда движется сама волна.}
\begin{equation}
\label{eq:87.4}
	v_1 > v_2.
\end{equation}

Неравенства \ref{eq:87.1} и \ref{eq:87.3} означают, что при прохождении газа
через ударную волну происходит его сжатие — его давление и плотность
возрастают. Неравенство $v_1>c_1$ означает, что ударная волна движется
относительно находящегося перед ней газа со сверхзвуковой скоростью; ясно
поэтому, что в этот газ не могут проникнуть никакие исходящие от ударной волны
возмущения. Другими словами, наличие ударной волны вовсе не сказывается на
состоянии газа впереди нее.

Покажем теперь, что все неравенства (\ref{eq:87.1}-\ref{eq:87.4}) справедливы и для
ударных волн произвольной интенсивности — при том же предположении о знаке
производной $(\partial^2 V/\partial p^2)_s$ \footnote{Неравенства
(\ref{eq:87.1}-\ref{eq:87.4}) былн получены для ударных волн произвольной
интенсивности в политропном газе \emph{Жуге (Е. Jouguet, 1904) и Цемпленом (G,
Zemplen, 1905)}. Излагаемое ниже доказательство для произвольной среды дано
\emph{Л. Д. Ландау (1944).}}.

Величина $j^2$ определяет наклон хорды, проведенной из начальной точки ударной
адиабаты $1$ в произвольную точку $2$ ($-j^2$ есть тангенс угла наклона этой
хорды к оси $V$). Покажем, прежде всего, что направление изменения этой
величины при перемещении точки $2$ вдоль адиабаты однозначно связано с
направлением изменения энтропии $s_2$ при том же перемещении.

Продифференцируем соотношения \ref{eq:85.5} и \ref{eq:85.8} по величинам,
относящимся к газу 2 при заданном состоянии газа 1. Это значит, что
дифференцируются $p_2, V_2, w_2$ и $j$ при заданных значениях $p_2, V_2, w_2$. Из
\ref{eq:85.5} получаем:

\begin{equation}
	\label{eq:87.5}
	dp_2 + j^2 dV_2 = (V_1 - V_2) d(j^2),
\end{equation}
а из \ref{eq:85.8}:
\[
	dw_2 + j^2 V_2 dV_2 = \frac{1}{2}(V_1^2 - V_2^2) d(j^2)
\]
или, раскрыв дифференциал $dw_2$:
\[
T_2 ds_2 + V_2 (dP-2 + j^2 dV_2) = \frac{1}{2}(V^2_1 - V^2_2) d(j^2).
\]
Подставив сюда $dp_2 + j^2 dV_2$ из \ref{eq:87.5}, получим соотношение
\begin{equation}
	\label{eq:87.6}
	T_2 ds_2 = \frac{1}{2} (V_1 - V_2)^2 d(j^2).
\end{equation}
Отсюда видно, что

\begin{equation}
	\label{eq:87.7}
	d(j^2)/ds_2 > 0,
\end{equation}

т. е. $j^2$ и $s_2$ меняются в одинаковом направлении.

Дальнейшие рассуждения имеют своей следующей целью показать, что на ударной
адиабате не может быть точек, в которых бы она касалась проведенной из точки 1
прямой (как это имело бы место в точке $O$ на рис. 56.)

В такой точке угол наклона хорды (проведенной из точки 1) имеет минимум, а $j^2$ —
соответственно максимум, и потому
\[
	d(j^2)/dp_2 = 0.
\]
Из соотношения \ref{eq:87.6} видно, что в таком случае будет и 
\[
	ds_2/dp_2 = 0.
\]
Далее, вычислим производную $d(j^2)/dp_2$ в произвольной точке ударной
адиабаты. Подставив в соотношение \ref{eq:87.5} дифференциал $dV_2$ в виде
\[
	dV_2 = \ddp{V_2}{p_2}{s_2}dp_2 + \ddp{V_2}{s_2}{p_2}ds_2,
\]
взяв для $ds_2$ выражение \ref{eq:87.6} и разделив все равенство на $dp_2$, получим
\begin{equation}
	\label{eq:87.8}
	\frac{d(j^2)}{p_2} = \frac{1 + j^2 \ddp{V_2}{p_2}{s_2}}{(V_1 - V_2) \left\lbrack 1 - \frac{j^2(V_1-V_2)}{2T_2} \ddp{V_2}{s_2}{p_2} \right\rbrack}.
\end{equation}

Отсюда видно, что обращение этой производной в нуль влечет за собой также и
равенство
\[
	1 + j^2 \ddp{V_2}{p_2}{s_2} = 1 - \frac{v^2_2}{c^2_2} = 0,
\]
т. е. $v_2 = c_2$. Обратно, из равенства $v_2 = c_2$ следует, что производная
$d(j^2)/dp_2$; последняя могла бы не обратиться в нуль, лишь если бы вместе с
числителем в \ref{eq:87.8} обратился бы в нуль также и знаменатель; но
выражения в числителе и знаменателе представляют собой две различные функции
точки 2 на ударной адиабате, их одновременное обращение в нуль могло бы
произойти лишь чисто случайно и потому невероятно \footnote{Подчеркнем, во
избежание недоразумений, что сама производная $d(j^2)/dp_2$ не является еще
одной независимой функцией точки 2; выражение \ref{eq:87.8} есть ее
определение.}.

Таким образом, все три равенства
\begin{equation}
	\label{eq:87.9}
	\frac{d(j^2)}{dp_2} = 0, \quad \frac{s_2}{p_2} = 0, \quad v_2 = c_2
\end{equation}
являются следствиями друг друга и имели бы место одновременно в точке $O$ на
кривой рис. 56 (имея в виду последнее из этих равенств, будем условно называть
такую точку \emph{звуковой}). Наконец, для производной от $(v_2/c_2)^2$ в этой
точке имеем
\[
	\frac{d}{dp_2}\left( \frac{v^2_2}{c^2_2} \right) =
	- \frac{d}{dp_2} \left\lbrack j^2 \ddp{V_2}{p_2}{s_2} \right\rbrack =
	- j^2  \ddp{^2 V_2}{p^2_2}{s_2}  .
\]

Ввиду предполагаемой везде положительности производной $(\partial^2 V/\partial
p^2)_s$ имеем, следовательно, в звуковой точке:
\begin{equation}
	\label{eq:87.10}
	\frac{d}{dp_2} \frac{v_2}{c_2} < 0.
\end{equation}

Теперь уже легко доказать невозможность существования звуковой точки на ударной
адиабате. В точках, лежащих вблизи начальной точки 1 над ней, имеем $v_2 < c_2$
(см. конец предыдущего параграфа). Поэтому равенство $v_2 = c_2$ может быть
достигнуто лишь при увеличении $v_2 / c_2$; другими словами, в звуковой точке
должно было бы быть $d(v_2/c_2)/dp_2 > 0$, между тем как согласно
\ref{eq:87.10} мы имеем как раз обратное неравенство. Аналогичным образом можно
убедиться в невозможности обращения $v_2/c_2$ в единицу и на нижней части
ударной адиабаты, под точкой 1.

Имея в виду доказанную таким образом невозможность существования звуковых
точек, можно заключить непосредственно из графика ударной адиабаты, что угол
наклона хорды 12 уменьшается при передвижении точки 2 вверх по кривой, a $j^2$
соответственно монотонно возрастает; ввиду неравенства \ref{eq:87.7} отсюда
следует, что монотонно возрастает и энтропия $s_2$. Таким образом, при
соблюдении необходимого условия $s_2>s_1$ будет и $p_2>p_1$.

Легко, далее, убедиться в том, что на верхней части ударной адиабаты
справедливы также и неравенства $v_2 < c_2, v_1>c_1$. Первое следует прямо из
того, что оно справедливо вблизи точки 1, а сделаться равным единице отношение
$v_2/c_2$ нигде не может. Второе следует из того, что ввиду невозможности
такого перегиба адиабаты, какой изображен на рис. 56, всякая хорда из точки 1 в
находящуюся над ней точку 2 расположена более круто, чем касательная к ударной
адиабате в точке 1.

Таким образом, на верхней части ударной адиабаты выполняются условие $s_2>s_1$
и все три неравенства (\ref{eq:87.1}—\ref{eq:87.2}). Наоборот, на нижней части
адиабаты все эти условия не выполняются. Следовательно, все эти условия
оказываются эквивалентными друг другу и выполнение одного из них автоматически
влечет за собой также и выполнение всех остальных.

Напомним лишний раз, что в изложенных рассуждениях все время предполагалось
выполненным условие положительности производной $(\partial^2 V/\partial
p^2)_s$. Если эта производная могла бы менять знак, то из необходимого
термодинамического неравенства $s_2>s_1$ уже нельзя было бы сделать никаких
универсальных заключений о неравенствах для остальных величин.
