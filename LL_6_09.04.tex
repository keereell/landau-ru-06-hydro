\section{Ударная адиабата}
\label{sec:p85}


Перейдем к подробному изучению ударных волн.
\footnote{Сделаем одно терминологическое замечание. Под ударной волной мы понимаем самую поверхность разрыва. В литературе, однако, можно встретить и другую терминологию, в которой поверхность разрыва называют фронтом ударной волны, а под ударной волной понимают поверхность разрыва вместе со следующим за ним течением газа.}
Мы видели, что в этих разрывах тангенциальная компонента скорости газа непрерывна.
Можно поэтому выбрать систему координат, в которой рассматриваемый элемент поверхности разрыва покоится, а тангенциальная компонента скорости газа по обе стороны поверхности равна нулю
\footnote{Такой выбор системы координат будет подразумеваться везде в этой главе, за исключением \S \ref{sec:p92}.

Неподвижную ударную волну часто называют \emph{скачком уплотнения}.
Если неподвижная ударная волна перпендикулярна к направлению потока, то
юворят о прямом скачке уплотнения; если же она наклонна к направлению
движения, то говорят о косом скачке уплотнения.}
).
Тогда можно писать вместо нормальной компоненты $v_x$ просто $v$ и условия (\ref{eq:84.7}) напишутся в виде
\begin{eqnarray}
    \label{eq:85.1}
    \rho_1 v_1 = \rho_2 v_2 \equiv j.\\
    \label{eq:85.2}
    p_1 + \rho_1 v_1^2=p_2 + \rho_2 v_2^2, \\
    \label{eq:85.3}
    w_1 + \frac{v_1^2}{2}=w_2 + \frac{v_2^2}{2},
\end{eqnarray}
где $j$ обозначает плотность потока газа через поверхность разрыва.
Мы условимся в дальнейшем всегда считать $j$ положительным, причем газ переходит
со стороны $1$ на сторону $2$.
Другими словами, мы будем называть газом $1$ тот, в сторону которого движется
ударная волна, а газом $2$ --- газ, остающийся за ней.
Сторону ударной волны, обращенную к газу $1$, будем называть передней, а
обращенную к газу $2$ --- задней.

Выведем ряд соотношений, являющихся следствием написанных условий. Введем удельные объемы
$V_1=1/\rho_1$, $V_2=1/\rho_2$ газа. Из (\ref{eq:85.1}) имеем:

\begin{equation}
    \label{eq:85.4}
    v_1 = j V_1, v_2 = j V_2,
\end{equation}
и, подставляя в (\ref{eq:85.2}):
\begin{equation}
    \label{eq:85.5}
    p_1 + j^2 V_1 = p_2 + j^2 V_2,
\end{equation}
или
\begin{equation}
    \label{eq:85.6}
    {j}^{2}=\frac{p_2-p_1}{V_1-V_2}.
\end{equation}


Эта формула (вместе с (\ref{eq:85.4})) связывает скорость распространения ударной волны с давлениями и плотностями газа по обеим сторонам поверхности.


Поскольку $j^2$ --- величина положительная, то должно быть одновременно $p_2 >
p_1, \quad V_1 > V_2$ или $p_2 < p_1, \quad V_1 < V_2$; мы увидим в дальнейшем, что в действительности возможен лишь первый случай.

Отметим еще следующую полезную формулу для разности скоростей $v_1 - v_2$.
Подставляя (\ref{eq:85.6}) в $v_1 - v_2 = j(V_1 - V_2)$, получаем \footnote{
Мы пишем здесь квадратный корень с положительным знаком, заранее имея в виду,что должно быть $v_1-v_2>0$, как это будет выяснено ниже \S 87.
}:
\begin{equation}
	\label{eq:85.7}
	v_1 - v_2 = \sqrt{(p_2-p_1)(V_1-V_2)}.
\end{equation}

Далее, пишем (\ref{eq:85.3}) в виде


\begin{equation}
	\label{eq:85.8}
	w_1 + \frac{j^2 V^2_1}{2} = w_2 + \frac{j^2 V^2_2}{2}
\end{equation}
и, подставляя $j^2$ из (\ref{eq:85.6}), получаем:

\begin{equation}
	\label{eq:85.9}
	w_1 - w_2 + \frac{1}{2} (V_1 +V_2)(p_2-p_1) = 0.
\end{equation}

Если ввести вместо тепловой функции внутреннюю энергию е согласно $\varepsilon =
w-pV$, то полученное соотношение можно написать в виде
\begin{equation}
	\label{eq:85.10}
	\varepsilon_1 - \varepsilon_2 + \frac{1}{2} (V_1+V_2)(p_1+p_2) = 0.
\end{equation}
Эти соотношения определяют связь между термодинамическими величинами по обе
стороны поверхности разрыва.

При заданных $p_1, V_1$ уравнение (\ref{eq:85.9}) или (\ref{eq:85.10})
определяет зависимость между $p_2$ и $V_2$. Об этой зависимости говорят как об
\emph{ударной адиабате или адиабате Гюгонио (W. J. Rankine, 1870; Н. Hugoniot,
1885)}. Графически она изображается (рис.53) в плоскости $p,V$ кривой,
проходящей через заданную точку $p_1, V_1$, отвечающую состоянию газа $1$ перед
ударной волной; эту точку ударной адиабаты мы будем называть ее \emph{начальной
точкой}. Отметим, что ударная адиабата не может пересечь вертикальной прямой
$V=V_1$ нигде, кроме только начальной точки.  Действительно, наличие такого
пересечения означало бы, что одному и тому же объему соответствуют два различных
давления, удовлетворяющих уравнению (\ref{eq:85.10}). Между тем, при $V_1 = V_2$
имеем из (\ref{eq:85.10}) также и $\varepsilon_1 = \varepsilon_2$, а при
одинаковых объемах и энергиях давления тоже должны быть одинаковыми.  Таким
образом, прямая $V=V_1$ делит ударную адиабату на две части, из которых каждая
находится целиком по одну сторону от этой прямой. По аналогичной причине ударная
адиабата пересекает только в одной точке $(p_1,V_1)$ также и горизонтальную
прямую $p = p_1$.


Пусть $aa'$  (рис. 54) есть ударная адиабата, проведенная через точку  $p_1,
V_1$в качестве начальной. Выберем на ней какую-нибудь точку $p_2, V_2$, и
проведем через нее другую адиабату $bb'$, для которой бы эта точка была
начальной. Очевидно, что пара значений $p_1, V_1$ будет удовлетворять также и
уравнению этой второй адиабаты.  Таким образом, адиабаты $aa'$ и $bb'$
пересекутся в обеих точках $p_1, V_1$ и $p_2, V_2$.  Подчеркнем, что обе эти
адиабаты отнюдь не совпадают полностью друг с другом, как это имело бы место для
адиабат Пуассона, проведенных через заданную точку.

Рис. 53	Рис. 54

Это обстоятельство является одним из следствий того факта, что уравнение ударной
адиабаты не может быть написано в виде $f(p,V)=\const$, где $f$ есть некоторая
функция своих аргументов, как это, например, имеет место для адиабаты Пуассона
(уравнение которой есть $s(p,V)=\const$. В то время как адиабаты Пуассона (для
заданного газа) составляют одно параметрическое семейство кривых, ударная
адиабата определяется заданием двух параметров: начальных значений $p_1, V_1$. С
этим же связано и следующее важное обстоятельство: если две (или более)
последовательные ударные волны переводят газ соответственно из состояния  $1$ в
состояние $2$ и из $2$ в $3$, то переход из состояния $1$ в $3$ путем
прохождения какой-либо одной ударной волны, вообще говоря, невозможен.

При заданном начальном термодинамическом состоянии газа (т.е. заданных $p_1,
V_1$) ударная волна определяется всего одним каким-либо параметром: если,
например, задать давление $p_2$ за волной, то по адиабате Гюгонио определится
$V_2$, а затем по формулам (\ref{eq:85.4}) и (\ref{eq:85.6}) --- плотность
потока $j$ и скорости $v_1$ и $v_2$ . Напомним, однако, что мы рассматриваем
здесь ударную волну в системе координат, в которой газ движется нормально к ее
поверхности. Если же учесть возможность расположения ударной волны под косым
углом к направлению потока, то понадобится еще один параметр, например, значение касательной к ее поверхности составляющей скорости.

Укажем здесь на следующее удобное графическое истолкование формулы
(\ref{eq:85.6}). Если соединить хордой точку $p_1, V_1$ на ударной адиабате (рис. 53)
с некоторой произвольной точкой $p_2, V_2$ на ней, то $(p_2-p_1)/(V_2-V_1)=j^2$ есть
не что иное, как тангенс угла наклона этой хорды к оси абсцисс (к ее
положительному направлению). Таким образом, значение $j$, а с ним и скорости
ударной волны, определяется в каждой точке ударной адиабаты углом наклона хорды,
проведенной в эту точку из начальной точки.

Наряду с другими термодинамическими величинами в ударной волне испытывает разрыв
также и энтропия. В силу закона возрастания энтропии последняя для газа может
лишь возрастать при его движении. Поэтому энтропия $s_2$ газа, прошедшего через
ударную волну, должна быть больше его начальной энтропии $s_1$:
\begin{equation}
	\label{eq:85.11}
	s_2 > s_1.
\end{equation}
Мы увидим ниже, что это условие налагает существенные ограничения на характер
изменения всех величин в ударной волне.

Подчеркнем здесь следующее обстоятельство. Наличие ударных волн приводит к
возрастанию энтропии при таких движениях, которые можно рассматривать во всем
пространстве как движение идеальной жидкости, не обладающей вязкостью и
теплопроводностью. Возрастание энтропии означает необратимость движения, т. е.
наличие диссипации энергии. Таким образом, разрывы представляют собой механизм,
который приводит к диссипации энергии при движении идеальной жидкости. В связи с
этим для движения тел в идеальной жидкости, сопровождающегося возникновением
ударных волн, не имеет места парадокс Даламбера (\S 11)---при таком движении тело
испытывает силу сопротивления.

Разумеется, истинный механизм возрастания энтропии в ударных волнах заключен в
диссипативных процессах, происходящих в тех весьма тонких слоях вещества,
которые в действительности представляют собой физические ударные волны (см.\S
93). Замечательно, однако, что величина этой диссипации целиком определяется
одними лишь законами сохранения массы, энергии и импульса, примененными к обеим
сторонам этих слрев: их ширина устанавливается как раз такой, чтобы дать
требуемое этими законами сохранения увеличение энтропии.

Возрастание энтропии в ударной волне оказывает еще и другое существенное влияние
на движение: если движение газа впереди ударной волны потенциально, то за ней
оно, вообще говоря, становится вихревым; мы вернемся к этому обстоятельству в \S
111.

% vim: ft=tex tw=80
