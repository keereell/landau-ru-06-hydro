\section{Гидростатика}
\label{sec:p3}

Для покоящейся жидкости, находящейся в однородном поле тяжести, уравнение Эйлера
(\ref{eq:2.4}) принимает вид
\begin{equation}
   \label{eq:3.1}
   \grad p = \rho \vect g
\end{equation}
Это уравнение описывает механическое равновесие жидкости. (Если внешние силы
вообще отсутствуют, то уравнение равновесия гласит просто $\nabla p = 0$, т.е.
$p = \const$ — давление одинаково во всех точках жидкости.)

Уравнение (\ref{eq:3.1}) непосредственно интегрируется, если плотность жидкости можно
считать постоянной во всем ее объеме, т е. если не происходит заметного сжатия
жидкости под действием внешнего поля. Направляя ось $z$ вертикально вверх,
имеем:
\[
   \pd{p}{x}=\pd{p}{y}=0,\: \pd{p}{z} = -\rho g.
\]
Отсюда
\[
   p = -\rho gz + \const
\]
Если покоящаяся жидкость имеет свободную поверхность (на высоте $h$), к которой
приложено одинаковое во всех точках внешнее давление $p_0$, то эта поверхность
должна быть горизонтальной плоскостью $z=h$. Из условия $p=p_0$ при $z=h$ имеем
\[
   \const = p_0 + \rho gh
\]
так что
\begin{equation}
   \label{eq:3.2}
   p = p_0 + \rho g(h-z)
\end{equation}

Для больших масс жидкости или газа плотность $\rho$ нельзя, вообще говоря,
считать постоянной; это в особенности относится к газам (например, к воздуху).
Предположим, что жидкость находится не только в механическом, но и в тепловом
равновесии. Тогда температура одинакова во всех точках жидкости, и уравнение
(\ref{eq:3.1}) может быть проинтегрировано следующим образом. Воспользуемся известным
термодинамическим соотношением
\[
   d\Phi = -s\;dT + V\; dp
\]
где $\Phi$ — термодинамический потенциал, отнесенный к единице массы жидкости.
При постоянной температуре
\[
   d\Phi = V\; dp = \frac1{\rho}dp
\]
Отсюда видно, что выражение $\frac1{\rho}\nabla p$   можно написать в
рассматриваемом случае как $\nabla \Phi$, так что уравнение равновесия (\ref{eq:3.1})
принимает вид
\[
   \nabla \Phi = \vect g
\]
Для постоянного вектора $\vect g$, направленного вдоль оси $z$ (в отрицательном
ее направлении), имеет место тождество
\[
   \vect g = - \nabla (gz)
\]
Таким образом,
\[
   \nabla(\Phi + gz) = 0,
\]
откуда находим, что вдоль всего объема жидкости должна быть постоянной сумма
\begin{equation}
   \label{eq:3.3}
   \Phi + gz = \const
\end{equation}
$gz$ представляет собой потенциальную энергию единицы массы жидкости в поле
тяжести. Условие (\ref{eq:3.3}) известно уже из статистической физики как условие
термодинамического равновесия системы, находящейся во внешнем поле.

Отметим здесь еще следующее простое следствие из уравнения (\ref{eq:3.1}). Если жидкость
или газ (например, воздух) находятся в. механическом равновесии в поле тяжести,
то давление в них может быть функцией только от высоты $z$ (если бы на данной
высоте давление было различно в различных местах, то возникло бы двилжение).
Тогда из (\ref{eq:3.1}) следует, что и плотность
\begin{equation}
   \label{eq:3.4}
   \rho = - \frac1{g} \frac{dp}{dz}
\end{equation}
тоже является функцией только от $z$. Но давление и плотность однозначно
определяют температуру в данной точке тела. Следовательно, и температура должна
быть функцией только от $z$. Таким образом, при механическом равновесии в поле
тяжести распределение давления, плотности и температуры зависит только от
высоты. Если же, например, температура различна в разных местах жидкости на
одной и той же высоте, то механическое равновесие в ней невозможно.

Наконец, выведем уравнение равновесия очень большой массы жидкости, части
которой удерживаются вместе силами гравитационного притяжения (звезда). Пусть
$\varphi$ — ньютоновский гравитационный потенциал создаваемого жидкостью поля.
Он удовлетворяет дифференциальному уравнению
\begin{equation}
   \label{eq:3.5}
   \Delta \varphi = 4 \pi G \rho
\end{equation}
где $G$ — гравитационная постоянная. Наиряженность гравитационного  поля   
равна $-\grad \varphi$, так что сила, действующая на массу $\rho$, есть
$-\rho \grad \varphi$. Поэтому условие равновесия будет
\[
   \grad p = - \rho \grad \varphi
\]
Разделив это равенство на $\rho$, применив к обеим его сторонам операцию $\Div$
и воспользовавшись уравнением (\ref{eq:3.6}), получим окончательное уравнение равновесия
в виде
\begin{equation}
   \label{eq:3.6}
   \Div \left( \frac1{\rho}\grad p\right) = -4\pi G \rho.
\end{equation}
Подчеркнем, что здесь идет речь только о механическом равновесии; существование
же полного теплового равновесия в уравнении (\ref{eq:3.6}) отнюдь не предполагается.

Если тело не вращается, то в равновесии оно будег иметь сферическую форму, а
распределение плотности и давления в нем будет центрально-симметричным.
Уравнение (\ref{eq:3.6}), написанное в сферических координатах, примет при этом вид
\begin{equation}
   \label{eq:3.7}
   \frac1{r^2}\frac{d}{dr}\left(\frac{r^2}{\rho}\frac{dp}{dr}\right) = -4\pi G \rho.
\end{equation}
