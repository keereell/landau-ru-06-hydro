\section{Уравнение Эйлера}
\label{sec:p2}

Выделим в жидкости некоторый объем. Полная сила, действующая на выделенный
объем жидкости, равна интегралу
\[
   - \oint p d \vect f
\]
взятому по поверхности рассматриваемого объема. Преобразуя его в интеграл
по объему, имеем:
\[
   - \oint p d \vect f = - \int \grad p dV
\]
Отсюда видно, что на каждый элемент объема $dV$ жидкости действует со стороны
окружающей его жидкости сила $-dV \grad p$. Другими словами, можно сказать,
что на единицу объема жидкости действует сила $-\grad p$.

Мы можем теперь написать уравнение движения элемента объема жидкости,
приравняв силу $- \grad p$ произведению массы $\rho$ единицы объема жидкости
на ее ускорение $d \vect v / dt$:
\begin{equation}
   \label{eq:2.1}
   \rho \frac{d \vect v}{dt} = - \grad p
\end{equation}

Стоящая здесь производная $d \vect v / dt$ определяет не изменение скорости
жидкости в данной неподвижной точке пространства, а изменение скорости
определенной передвигающейся в пространстве частицы жидкости. Эту производную
надо выразить через величины, относящиеся к неподвижным в пространстве точкам.
Для этого заметим, что изменение $d \vect v$ скорости данной частицы жидкости в
течение времени $dt$ складывается из двух частей: из изменения скорости в данной
точке пространства в течение времени $dt$ и из разности скоростей (в один и тот
же момент времени) в двух точках, разделенных расстоянием $d \vect r$,
пройденным рассматриваемой частицей жидкости в течение времени $dt$. Первая нз
этих частей равна
\[
   \frac{\partial \vect v}{\partial t}dt
\]
где теперь производная $\partial \vect v / \partial t$ берется при постоянных
$x,y,z$, т.е. в заданной точке пространства. Вторая часть изменения скорости
равна
\[
   \ddv{v}{x}+\ddv{v}{y}+\ddv{v}{z} = (d \vect r \nabla) \vect v
\]
Таким образом,
\[
   d \vv = \frac{\partial\vv}{\partial t}dt + (d \vect r \nabla) \vect v
\]
или, разделив обе стороны равенства на $dt$ \footnote{Определенную таким образом
производную $d/dt$ называют \textit{субстанциональной}, подчеркивая тем самым
ее связь с перемещающимся веществом},
\begin{equation}
   \label{eq:2.2}
   \frac{d \vv}{dt} = \frac{\partial \vv}{\partial t} + \vnv
\end{equation}
Подставив полученное соотношение в (2,1), находим:
\begin{equation}
   \label{eq:2.3}
   \frac{\partial\vv}{\partial t}+\vnv = - \frac1{\rho} \grad p
\end{equation}
Это и есть искомое уравнение движения жидкости, установленное впервые \textit{Л.
Эйлером} в 1755 г. Оно называется \textit{уравнением Эйлера} и является одним из
основных уравнений гидродинамики.

Если жидкость находится в поле тяжести, то на каждую единицу ее объема действует
еще сила $\rho \vect g$, где $\vect g$ есть ускорение силы тяжести. Эта сила
должна быть прибавлена к правой стороне уравнения (\ref{eq:2.1}), так что (\ref{eq:2.3})
приобретает вид
\begin{equation}
   \label{eq:2.4}
   \frac{\partial\vv}{\partial t}+\vnv = - \frac{\nabla p}{\rho} + \vect g
\end{equation}

При выводе уравнений движения мы совершенно не учитывали процессов диссипации
энергии, которые могут иметь место в текущей жидкости вследствие внутреннего
трения (вязкости) в жидкости и теплообмена между различными ее участками.
Поэтому все излагаемое здесь и в следующих параграфах этой главы относится
только к таким движениям жидкостей и газов, при которых несущественны процессы
теплопроводности и вязкости; о таком движении говорят как о движении
\textit{идеальной жидкости}.

Отсутствие теплообмена между отдельными участками жидкости (а также, конечно, и
между жидкостью и соприкасающимися с нею окружающими телами) означает, что
движение происходит адиабатически, причем адиабатически в каждом из участков
жидкости. Таким образом, движение идеальной жидкости следует рассматривать как
адиабатическое.

При адиабатическом движении энтропия каждого участка жидкости остается
постоянной при перемещении последнего в пространстве. Обозначая посредством $s$
энтропию, отнесенную к единице массы жидкости, мы можем выразить адиабатичность
движения уравнением
\begin{equation}
   \label{eq:2.5}
   \frac{ds}{dt} = 0,
\end{equation}
где полная производная по времени означает, как и в (\ref{eq:2.1}),
изменение энтропии заданного перемещающегося участка жидкости. Эту производную
можно написать в виде
\begin{equation}
   \label{eq:2.6}
   \frac{\partial s}{\partial t} + \vv \grad s= 0.
\end{equation}
Это есть общее уравнение, выражающее собой адиабатичность движения идеальной
жидкости. С помощью (\ref{eq:1.2}) его можно написать в виде "уравнения непрерывности"
для энтропии
\begin{equation}
   \label{eq:2.7}
   \frac{\partial{(\rho s)}}{\partial t} + \Div{(\rho s \vv)} = 0.
\end{equation}
Произведение $\rho s \vv$ представляет собой \textit{плотность потока энтропии}.

Обычно уравнение адиабатичности принимает гораздо более простую форму. Если, как
это обычно имеет место, в некоторый начальный момент времени энтропия одинакова
во всех точках объема жидкости, то она останется везде одинаковой и неизменной
со временем и при дальнейшем движении жидкости. В этих случаях можно,
следовательно, писать уравнение адиабатичности просто в виде
\begin{equation}
   \label{eq:2.8}
     s = \const
\end{equation}
что мы и будем обычно делать в дальнейшем. Такое движение называют
\textit{изэнтропическим}.

Изэнтропичностью движения можно воспользоваться для того, чтобы представить
уравнение движения (\ref{eq:2.3}) в несколько ином виде. Для этого воспользуемся
известным термодинамическим соотношением
\[
   dw = T\;ds + V\; dp,
\]
где $w$ — тепловая функция единицы массы жидкости, $V=1/\rho$ — удельный объем,
а $T$ — температура. Поскольку $s=\const$, имеем просто
\[
   dw = V\; dp = \frac1{\rho} dp,
\]
и поэтому
\[
   \frac1 \rho \nabla p = \nabla w.
\]
Уравнение (\ref{eq:2.3}) можно, следовательно, написать в виде
\begin{equation}
   \label{eq:2.9}
   \frac{\partial \vv}{\partial t} + \vnv = - \grad w.
\end{equation}
Полезно заметить еще одну форму уравнения Эйлера, в котором оно содержит только
скорость. Воспользовавшись известной формулой векторного анализа
\[
   \frac{1}{2}\grad v^2 = \vrv + \vnv,
\]
можно написать (\ref{eq:2.9}) в виде
\begin{equation}
   \label{eq:2.10}
   \frac{\partial \vv}{\partial t} - \vrv = -\grad \left( w + \frac{v^2}{2} \right)
\end{equation}
Применив к обеим сторонам этого уравнения операцию $\rot$, получим уравнение
\begin{equation}
   \label{eq:2.11}
   \frac{\partial}{\partial t} \rot \vv = \rot \vrv,
\end{equation}
содержащее только скорость.

К уравнениям движения надо добавить граничные условия, которые должны
выполняться на ограничивающих жидкость стенках. Для идеальной жидкости это
условие должно выражать собой просто тот факт, что жидкость не может проникнуть
за твердую поверхность. Это значит, что на неподвижных стенках должна обращаться
в нуль нормальная к поверхности стенки компонента скорости жидкости:
\begin{equation}
   \label{eq:2.12}
   v_n = 0
\end{equation}
(в общем же случае движущейся поверхности $v_n$ должно быть равно
соответствующей компоненте скорости поверхности).

На границе между двумя несмешивающимися жидкостями должны выполняться условие
равенства давлений и условие равенства нормальных к поверхности раздела
компонент скорости обеих жидкостей (причем каждая из этих скоростей равна
скорости нормального перемещения самой поверхности раздела).

Как уже было указано в начале \S1, состояние движущейся жидкости определяется
пятью величинами: тремя компонентами скорости $\vv$ и, например, давлением $p$ и
плотностью $\rho$. Соответственно этому полная система гидродинамических
уравнений должна содержать пять уравнений. Для идеальной жидкости этими
уравнениями являются уравнения Эйлера, уравнение непрерывности и уравнение,
выражающее адиабатичность движения.

\subsection*{Задача}
Написать уравнения одномерного течения идеальной жидкости в переменных $a,t$,
где $a$ есть $x$-координата частиц жидкости в некоторый момент времени $t=t_0$
(так называемая переменная Лагранжа) \footnote{Хотя эти переменные и принято называть
лагранжевыми, но в действительности уравнения движения жидкости в этих координатах
были впервые получены \textit{Л. Эйлером} одновременно с основными уравнениями \ref{eq:2.3}}.

\texttt{Решение.} В указанных переменных координата $x$ каждой частицы жидкости
в произвольный момент времени рассматривается как функция $t$ и ее же координаты
$a$ в начальный момент: $x = x(a,t)$. Условие сохранения массы элемента жидкости
при его движении (уравнение непрерывноеги) напишется соответственно в виде
$\rho\;dx = \rho_0 da$, или
\[
   \rho \ddp{x}{a}{t}= \rho_0,
\]
где $\rho_0(a)$ есть заданное начальное распределение плотности. Скорость жидкой
частицы есть, по определению, $v = \ddp{x}{t}{a}$, а производная $\ddp{v}{t}{a}$
определяет изменение со временем скорости данной частицы по мере ее движения.
Уравнение Эйлера напишется в виде
\[
   \ddp{v}{t}{a} = - \frac1{\rho_0} \ddp{p}{a}{t}
\]
а уравнение адиабатичности:
\[
   \ddp{s}{t}{a} = 0
\]
