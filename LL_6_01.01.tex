\section{Уравнение непрерывности}
\label{sec:p1}

Изучение движения жидкостей (и газов) представляет собой содержание
гидродинамики. Поскольку явления, рассматриваемые в гидродинамике, имеют
макроскопический характер, то в гидродинамике жидкость
\footnote{Мы говорим здесь и ниже для краткости только о жидкости, имея при этом
в виду как жидкости, так и газы} рассматривается как
сплошная среда. Это значит, что всякий малый элемент объема жидкости считается
все-таки настолько большим, что содержит еще очень большое число молекул.
Соответственно этому, когда мы будем говорить о бесконечно малых элементах
объема, то всегда при этом будет подразумеваться "физически" бесконечно малый
объем, т.е. объем, достаточно малый по сравнению с объемом тела, но большой по
сравнению с межмолекулярными расстояниями. В таком же смысле надо понимать в
гидродинамике выражения "жидкая частица", "точка жидкости". Если, например,
говорят о смещении некоторой частицы жидкости, то при этом идет речь не о
смещении отдельной молекулы, а о смещении целого элемента объема, содержащего
много молекул, но рассматриваемого в гидродинамике как точка.


Математическое описание состояния движущейся жидкости осуществляется с помощью
функций, определяющих распределение скорости жидкости $\vect{v} = \vect v
(x,y,z,t)$ и каких-либо ее двух термодинамических величин, например давления
$p(x,y,z,t)$ и плотности $\rho(x,y,z,t)$. Как известно, все термодинамические
величины определяются по значениям каких-либо двух из них с помощью уравнения
состояния вещества; поэтому задание пяти величин: трех компонент скорости $v$,
давления $p$ и плотности $\rho$, полностью определяет состояние движущейся
жидкости.

Все эти величины являются, вообще говоря, функциями коорг динат $x,y,z$ и
времени $t$. Подчеркнем, что $\vect v (x,y,z,t)$ есть скорость жидкости в каждой
данной точке $x,y,z$ пространства в момент времени $t$, т.е. относится к
определенным точкам пространства, а не к определенным частицам жидкости,
передвигающимся со временем в пространстве; то же самое относится к величинам
$\rho$, $p$.

Начнем вывод основных гидродинамических уравнений с вывода уравнения,
выражающего собой закон сохранения вещества в гидродинамике.

Рассмотрим некоторый объем $V_0$ пространства. Количество (масса) жидкости в
этом объеме есть $\int \rho dV$, где есть плотность жидкости, а интегрирование
производится по объему. Через элемент $\df$ поверхности, ограничивающей
рассматриваемый объем, в единицу времени протекает количество $\rho \vect v \df$
жидкости; вектор $\df$ по абсолютной величине равен площади элемента поверхности
и направлен по нормали к ней. Условимся направлять $\df$ по внешней нормали.
Тогда $\rvdf$ положительно, если жидкость вытекает из объема, и отрицательно,
если жидкость втекает в него. Полное количество жидкости, вытекающей в единицу
времени из объема $V_0$, есть, следовательно,
\[
   \oint \rvdf
\]
где интегрирование производится по всей замкнутой поверхности, охватывающей
рассматриваемый объем.

С другой стороны, уменьшение количества жидкости в объеме $V_0$ можно написать в
виде
\[
  - \frac{\partial}{\partial t} \int \rho dV
\]
Приравнивая оба выражения, получаем:
\begin{equation}
\label{eq:1.1}
	\frac{\partial}{\partial t} \int \rho dV = - \oint \rvdf
\end{equation}
Интеграл по поверхности преобразуем в интеграл по объему:
\[
   \oint \rvdf = \int \Div \rho \vect v dV
\]
Таким образом,
\[
   \int \left( \frac{\partial \rho}{\partial t} + \Div \rho \vect v \right) dV = 0
\]
Поскольку это равенство должно иметь место для любого объёма, то должно быть
равным нулю подынтегральное выражение, т.е.
\begin{equation}
 \label{eq:1.2}
 \frac{\partial \rho}{\partial t} + \Div \rho \vect v = 0
\end{equation}
Это---так называемое \textit{уравнение непрерывности}.

Раскрыв выражение  $\Div \rho \vect v$, (\ref{eq:1.2}) можно написать также
в виде
\begin{equation}
 \label{eq:1.3}
   \frac{\partial \rho}{\partial t} + \rho \Div \vect v + \vect v \grad \rho = 0
\end{equation}
Вектор
\begin{equation}
 \label{eq:1.4}
   \vect j =\rho \vect v
\end{equation}
называют \textit{плотностью потока жидкости}. Его направление совпадает с
направлением движения жидкости, а абсолютная величина определяет количество
жидкости, протекающей в единицу времени через единицу площади, расположенной
перпендикулярно к скорости.

