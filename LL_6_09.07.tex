\section{Эволюционность ударных волн}\label{sec:p88}

Вывод неравенств (\ref{eq:87.1}—\ref{eq:87.4}) в \S\S \ref{sec:p86},
\ref{sec:p87} был связан с определенным предположением о термодинамических
свойствах среды — с положительностью производной $(\partial^2 V/\partial
p^2)_s$.  Весьма существенно, однако, что неравенства
\begin{equation}
	\label{eq:88.1}
	v_1 > c_1, v_2 < c_2
\end{equation}

для скоростей могут быть получены также и из совершенно иных соображений,
показывающих, что ударные волны с нарушенными условиями (\ref{eq:88.1}) все
равно не могли бы существовать, даже если бы это не противоречило изложенным
выше чисто термодинамическим соображениям \footnote{ Напомним в то же время,
что (по крайней мере для ударных волн слабой интенсивности) эти
термодинамические соображения приводят к условиям (\ref{eq:88.1}) также и при
$(\partial ^2 V/\partial p^2)_s < 0$, когда ударная волна является волной
разрежения (а не сжатия); это обстоятельство было отмечено в конце \S 86.}.

Именно необходимо исследовать еще вопрос об устойчивости ударных волн. Наиболее
общее необходимое условие устойчивости состоит в требовании, чтобы любое
бесконечно малое возмущение начального (в некоторый момент $t=0$) состояния
приводило бы лишь к вполне определенным бесконечно малым же изменениям течения,
— по крайней мере в течение достаточно малого промежутка времени $t$. Последняя
оговорка означает недостаточность указанного условия; так, если начальное малое
возмущение возрастает даже экспоненциально (как $e^{\gamma t}$ с положительной постоянной $\gamma$), то в течение времени $t \le 1/\gamma$ возмущение остается малым, хотя в конце концов оно
и приводит к разрушению данного режима движения. Возмущением, не
удовлетворяющим поставленному необходимому условию, является расщепление
ударной волны на два (или более) последовательных разрыва; очевидно, что
изменение движения при этом сразу же оказывается не малым, хотя при малых $t$
(когда оба разрыва не разошлись еще на большое расстояние) оно и занимает лишь
небольшой интервал расстояний $\delta x$.

Произвольное начальное малое возмущение определяется некоторым числом
независимых параметров. Дальнейшая же эволюция возмущения определяется системой
линеаризованных граничных условий, которые должны удовлетворяться на
поверхности разрыва. Поставленное выше необходимое условие устойчивости будет
выполнено, если число этих уравнений совпадает с числом содержащихся в них
неизвестных параметров — тогда граничные условия определяют дальнейшее развитие
возмущения, которое при малых $t > 0$ останется малым. Если же число уравнений
больше или меньше числа независимых параметров, то задача о малом возмущении не
имеет решений вовсе или имеет их бесконечное множество. Оба случая
свидетельствовали бы о неправомерности исходного предположения (малость
возмущения при малых $t$) и, таким образом, противоречили бы поставленному
требованию. Сформулированное таким образом условие называют условием
\emph{эволюционности течения}.

Рассмотрим возмущение ударной волны, представляющее собой ее бесконечно малое
смещение в направлении, перпендикулярном ее плоскости \footnote{Излагаемое ниже
обоснование неравенств (\ref{eq:88.1}) принадлежит \emph{Л. Д. Ландау
(1944).}}. Оно сопровождается бесконечно малым возмущением также и других
величин—давления, скорости и т.д. газа по обеим сторонам поверхности разрыва.
Эти возмущения, возникнув вблизи волны, будут затем распространяться от нее,
переносясь (относительно газа) со скоростью звука; это не относится лишь к
возмущению энтропии, которое будет переноситься только с самим газом. Таким
образом, произвольное возмущение данного типа можно рассматривать как
совокупность звуковых возмущений, распространяющихся в газах 1 и 2 по обе
стороны ударной волны, и возмущения энтропии; последнее, перемещаясь вместе с
газом, будет, очевидно, существовать лишь в газе 2 позади ударной волны. В
каждом из звуковых возмущений изменения всех величин связаны друг с другом
определенными соотношениями, следующими из уравнений движения (как в любой
звуковой волне; \S \ref{sec:p64}); поэтому каждое такое возмущение определяется всего лишь
одним параметром.

Подсчитаем теперь число возможных звуковых возмущений. Оно зависит от
относительной величины скоростей газа $v_1, v_2$ и скоростей звука $c_1, c_2$.
Выберем направление движения газа (со стороны 1 на сторону 2) в качестве
положительного направления оси $x$. Скорость распространения возмущения в газе
1 относительно неподвижной ударной волны есть $u_1 = v_1 \pm c_1$, а в газе 2
$u_2 = v_2 \pm c_2$. Тот факт, что эти возмущения должны распространяться по
направлению от ударной волны, означает, что должно быть $u_1 < 0, u_2 > 0$.

Предположим, что $v_1 > c_1, v_2 < c_2$. Тогда ясно, что оба значения $u_1 =
v_1 \pm c_1$ будут положительными, а из двух значений $u_2$  будут
положительными лишь $v_2 + c_2$. Это значит, что в газе 1 вообще не сможет быть
интересующих нас звуковых возмущений, а в газе 2 — всего одно,
распространяющееся относительно самого газа со скоростью $v_2 + c_2$.
Аналогичным образом производится подсчёт в других случаях.

Результат изображен на рис. 57, где каждая стрелка соответствует одному
звуковому возмущению, распространяющемуся относительно газа в указываемую
стрелкой сторону. Каждое же звуковое возмущение определяется, как было выше
указано, одним параметром. Кроме того, во всех четырех случаях имеется еще по
два параметра: параметр, определяющий распространяющееся в газе 2 возмущение
энтропии, и параметр, определяющий самое смещение ударной волны.

Для каждого из четырех случаев на рис. 57 цифрой в кружке указано получающееся
таким образом полное число параметров, определяющих произвольное возмущение,
возникающее при смещении ударной волны.

С другой стороны, число необходимых граничных условий, которым должно
удовлетворять возмущение на поверхности разрыва, равно трем (условия
непрерывности потоков массы, энергии и импульса). Во всех изображенных на рис.
57 случаях, за исключением лишь первого, число имеющихся независимых параметров
превышает число уравнений. Мы видим, что эволю-ционны лишь ударные волны,
удовлетворяющие условиям (\ref{eq:88.1}). Эти условия, таким образом,
необходимы для существования ударных волн, вне зависимости от термодинамических
свойств среды. Искусственно созданный разрыв, не удовлетворяющий этим условиям,
немедленно распался бы на другие разрывы \footnote{Во всех перечисленных на
рис. 57 неэволюционных случаях возмущение недоопределеио — число произвольных
параметров превышает число уравнений. Упомянем, что в магнитной гидродинамике
ударные волны могут быть неэволюционными в силу как недоопределенности, так и
переопределенности возмущений (см. VIII, \S 73).}.

Эволюционная ударная волна устойчива по отношению к рассмотренному типу
возмущений и в обычном смысле этого слова. Если искать смещение ударной волны
(а с ним и возмущения всех остальных величин) в виде, пропорциональном $e^{-i
\omega t}$, то заранее очевидно, что однозначно определяемое граничными
условиями значение $\omega$ может быть только нулем — уже хотя бы из тех
соображений, что в задаче нет никаких параметров размерности обратного времени,
которые могли бы определить отличное от нуля значение $\omega$.

Мы вернемся к вопросу об устойчивости ударных волн в \S \ref{sec:p90}.

