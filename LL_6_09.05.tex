\section{Ударные волны слабой интенсивности}\label{sec:p86}

Рассмотрим ударную волну, в которой все величины испытывают лишь небольшой
скачок; о таких разрывах мы будем говорить как об ударных волнах слабой
интенсивности. Преобразуем соотношение \ref{eq:85.9}, производя в нем разложение по
степеням малых разностей $s_2-s_1$ и$p_2-p_1$. Мы увидим, что при таком разложении
в \ref{eq:85.9} сокращаются члены первого и второго порядков по $p_2-p_1$;поэтому
необходимо производить разложение по $p_2-p_1$ до членов третьего порядка
включительно. По разности же $s_2-s_1$ достаточно разложить до членов первого
порядка. Имеем:
\[
w_2 - w_1 = \ddp{w}{s_1}{p}(s_2-s_1)+\ddp{w}{p_1}{s}(p_2-p_1)+\frac{1}{2} \ddp{^2 w}{p^2_1}{s}(p_2-p_1)^2 + \frac{1}{6}\ddp{^3 w}{p^3_1}{s}(p_2-p_1)^3.
\]

Но согласно термодинамическому соотношению $dw=Tds+Vdp$ имеем для производных:
\[
\ddp{w}{s}{p} = T, \quad \ddp{w}{p}{s} = V.
\]
Поэтому

\[
w_2-w_1 = T_1(s_2-s_1) + V_1(p_2-p_1) + \frac{1}{2}\ddp{V}{p_1}{s}(p_2-p_1)^2+\frac{1}{6} \ddp{^2 V}{p^2_1}{s}(p_2-p_1)^3.
\]

Объем $V_2$ достаточно разложить только по $p_2-p_1$, поскольку во втором члене
уравнения \ref{eq:85.9} уже имеется малая разность $p_2-p_1$ и разложение по
$s_2-s_1$ дало бы член порядка $(s_2-s_1)(p_2-p_1)$, не интересующий нас. Таким
образом,
\[
V_2-V_1 = \ddp{V}{p_1}{s}(p_2-p_1)+\frac{1}{2}\ddp{^2V}{p^2_1}{s}(p_2-p_1)^2.
\]
Подставляя эти разложения в \ref{eq:85.9}, получим следующее соотношение:

\begin{equation}
	\label{eq:86.1}
	s_2-s_1 = \frac1{12T_1}\ddp{^2V}{p^2_1}{s}(p_2-p_1)^3.
\end{equation}
Таким образом, скачок энтропии в ударной волне слабой интенсивности является
малой величиной третьего порядка по сравнению со скачком давления.

Адиабатическая сжимаемость вещества — $(\partial V/\partial p)_s$ практически
всегда падает с увеличением давления, т. е. вторая произ-

водная \footnote{
Для политропного газа
\[
	\ddp{^V}{p^2}{s} = \frac{\gamma+1}{\gamma^2} \frac{V}{p^2}
\]
Это выражение проще всего можно получить путем дифференцирования уравнения
адиабаты Пуассона $pV^\gamma=\const $.
}
\begin{equation}
	\label{eq:86.2}
	\ddp{^2V}{p^2}{s}>0.
\end{equation}
Подчеркнем, однако, что это неравенство не является термодинамическим
соотношением и, в принципе, возможны его нарушения \footnote{Так, это может
иметь место в области вблизи критической точки жидкость — газ. Ситуация с
нарушением условия \ref{eq:86.2} может быть также имитирована на ударной
адиабате для среды, допускающей фазовый переход (в результате чего иа адиабате
возникает излом). См. об этом в книге: \emph{Зельдович Я.Б., Райзер Ю.П.} Физика
ударных воли и высокотемпературных гидродинамических явлений. — Изд. 2-е. — М.:
Наука, 1966, гл. I, \S 19; гл. XI, \S 20.}. Как мы неоднократно
увидим ниже, в газодинамике знак производной \ref{eq:86.2} весьма существен; в
дальнейшем мы будем всегда считать его положительным.

Проведем через точку $1$ $(p_1,V_1)$ на $p,V$-диаграмме две кривые — ударную
адиабату и адиабату Пуассона. Уравнение адиабаты Пуассона есть $s_2-s_1=0$. Из
сравнения этого уравнения с уравнением \ref{eq:86.1} ударной адиабаты вблизи
точки $1$ видно, что обе кривые касаются в этой точке, причем имеет место
касание второго порядка — совпадают не только первые, но и вторые производные.
Для того чтобы выяснить взаимное расположение обеих кривых вблизи точки $1$,
воспользуемся тем, что согласно \ref{eq:86.1} и \ref{eq:86.2} при $p_2>p_1$ на
ударной адиабате должно быть $s_2>s_1$, между тем как на адиабате Пуассона
остается $s_2=s_1$. Поэтому абсцисса точки на ударной адиабате должна быть при
той же ординате $p_2$ больше абсциссы точки на адиабате Пуассона. Это следует
из того, что согласно известной термодинамической формуле
\[
	\ddp{V}{s}{p} = \frac{T}{c_p}\ddp{V}{T}{p}
\]
энтропия растет с увеличением объема при постоянном давлении — для всех тел,
которые расширяются при нагревании, т. е. у которых $(\partial V/\partial
T)_p>0$. Аналогично убеждаемся в том, что ниже точки $1$ (т.е. при $p_2<p_1$)
абсциссы точек адиабаты Пуассона должны быть больше абсцисс ударной адиабаты.
Таким образом, вблизи точки своего касания обе кривые расположены указанным на
рис. 55 образом ($HH'$ — ударная адиабата, а $PP'$ — адиабаты
Пуассона)\footnote{При $(\partial V/\partial T)_p < 0$ расположение обеих
кривых было бы обратным.}, причем в силу \ref{eq:86.2} обе обращены вогнутостью
вверх.



При малых $p_2-p_1$ и $V_2-V_1$ формулу \ref{eq:85.6} можно написать в первом
приближении в виде
\[
	j^2=-\ddp{p}{V}{s}
\]
(мы пишем здесь производную при постоянной энтропии, имея в виду, что
касательные к адиабатам Пуассона и ударной в точке $1$ совпадают). Далее,
скорости $v_1$ и $v_2$ в том же приближении одинаковы и равны

\[
	v=jV=\sqrt{-V^2 \ddp{p}{V}{s}} = \sqrt{\ddp{p}{\rho}{s}}.
\]
Но это есть не что иное, как скорость звука $c$. Таким образом, скорость
распространения ударных волн слабой интенсивности совпадает в первом
приближении со скоростью звука:
\begin{equation}
	\label{eq:86.3}
	v = c.
\end{equation}

Рис. 55	Из полученных свойств ударной адиабаты в окрестности точки $1$ можно
вывести ряд существенных следствий. Поскольку в ударной волне должно
выполняться условие $s_2>s_1$, то должно быть и
\[
	p_2>p_1,
\]
т. е. точки $2$ $(p_2,V_2)$ должны находиться выше точки $1$. Далее, поскольку
хорда $12$ идет круче касательной к адиабате в точке $1$ (рис. 53), а тангенс
угла наклона этой касательной равен производной $(\partial p_1/\partial
V_1)_{s_1}$ имеем:

\[
	j^2>-\ddp{p}{V_1}{s_1}.
\]
Умножая это неравенство с обеих сторон на $V^2_1$, находим:
\[
	j^2 V^2_1 = v^2_1 > - V^2_1 \ddp{p}{V_1}{s_1} = \ddp{p}{\rho_1}{s_1} = c^2_1,
\]

где $c_1$ — скорость звука, соответствующая точке $1$. Таким образом,
\[
	v_1>c_1.
\]

Наконец, из того, что хорда $12$ расположена менее круто, чем касательная в
точке $2$, аналогичным образом следует, что $v_2<c_2$ \footnote{Последняя
аргументация применима только вблизи точки $1$, где тангенс угла иаклопа
касательной к ударной адиабате в точке $2$ отличается от производной $(\partial
p_2/\partial V_2)_{s_2}$ лишь иа величину второго порядка малости.}.

Упомянем еще, в заключение, что при $(\partial^2 V/\partial p^2)_s<0$ из
условия $s_2>s_1$ для ударных волн слабой интенсивности следовало бы $p_2<p_1$,
а для скоростей — те же неравенства $v_1<c_1$, $v_2<c_1$.
