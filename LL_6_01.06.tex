\section{Поток энергии}
\label{sec:p6}

Выберем какой-нибудь неподвижный в пространстве элемент объема и определим, как
меняется со временем энергия находящейся в этом объеме жидкости. Энергия единицы
объема жидкости равна
\[
   \rho \frac{v^2}{2} + \rho \varepsilon,
\]
где первый член есть кинетическая энергия, а второй — внутренняя энергия
($\varepsilon$ — внутренняя энергия единицы массы жидкости). Изменение этой
энергии определяется частной производной
\[
   \frac{\partial}{\partial t} \left( \frac{\rho v^2}{2}+\rho \varepsilon \right).
\]

Для вычисления этой величины пишем:
\[
   \frac{\partial}{\partial t} \frac{\rho v^2}{2} =
   \frac{v^2}{2}\frac{\partial \rho}{\partial t} +
   \rho \vect v \frac{\partial \vv}{\partial t}
\]
или, воспользовавшись уравнением непрерывности (\ref{eq:1.2}) и уравнением движения (\ref{eq:2.3}),
\[
   \frac{\partial}{\partial t}\frac{\rho v^2}{2} = - \frac{v^2}{2}\Div{\rho \vv}
   - \vv \grad p - \rho \vv \vnv.
\]
В последнем члене заменяем $\vnv=(\vv \nabla v^2)/2$, а градиент давления
согласно термодинамическому соотношению $dw = T\;ds+dp/\rho$ заменяем на $\rho
\nabla w - \rho T \nabla s$ и получаем:
\[
   \frac{\partial}{\partial t}\frac{\rho v^2}{2} = - \frac{v^2}{2}\Div{\rho \vv}
   -\rho \vv \nabla \left(w + \frac{v^2}{2}\right) + \rho T \vv \nabla s.
\]

Для преобразования производной от $\rho \varepsilon$ воспользуемся
термодинамическим соотношением
\[
   d \varepsilon = T\; ds - p\; dV = T\; ds + \frac{p}{\rho^2}d\rho.
\]
Имея в виду, что сумма $\varepsilon + p/\rho = \varepsilon + pV$ есть не
что иное, как тепловая функция $w$ единицы массы, находим:
\[
   d(\rho \varepsilon) = \varepsilon\;d\rho + \rho\; d\varepsilon =
   w\; d\rho + \rho T\; ds,
\]
и потому
\[
   \pd{(\rho\varepsilon)}{t} = w \pd{\rho}{t}+\rho T \pd{s}{t}
   = - w \Div \rho \vv - \rho T \vv \nabla s.
\]
Здесь мы воспользовались также общим уравиеиием адиабатичности (\ref{eq:2.6}).
Собирая полученные выражения, находим для искомого изменения энергии
\[
   \pdt\left(\frac{\rho v^2}{2}+\rho\varepsilon\right) =
   -\left(w + \frac{v^2}{2}\right)\Div{\rho\vv} - \rho (\vv \nabla)
   \left(w + \frac{v^2}{2}\right),
\]
или окончательно
\begin{equation}
   \label{eq:6.1}
   \pdt \left( \frac{\rho v^2}{2}+ \rho \varepsilon \right) =
   - \Div \lbrace \rho \vv \left( \frac{v^2}{2}+w \right) \rbrace.
\end{equation}

Для того чтобы выяснить смысл полученного равенства, проинтегрируем его по
некоторому объему:
\[
   \pdt\int{\left(\frac{\rho v^2}{2}+\rho\varepsilon\right)dV} =
   - \int{\Div \lbrace\rho \vv \left(\frac{v^2}{2}+w\right)\rbrace dV},
\]
или, преобразовав стоящий справа объемный интеграл в интеграл по поверхности:
\begin{equation}
   \label{eq:6.2}
   \pdt\int{\left(\frac{\rho v^2}{2}+\rho\varepsilon\right)dV} =
   - \oint{\rho \vv \left(\frac{v^2}{2}+w\right)\df}.
\end{equation}

Слева стоит изменение в единицу времени энергии жидкости в некотором заданном
объеме пространства. Стоящий справа интеграл по поверхности представляет собой,
следовательно, количество энергии, вытекающей в единицу времени из
рассматриваемого объема. Отсюда видно, что выражение
\begin{equation}
   \label{eq:6.3}
   \rho \vv \left(\frac{v^2}{2}+w\right)
\end{equation}
представляет собой вектор \textit{плотности потока энергии}. Его абсолютная
величина есть количество энергии, протекающей в единицу времени через единицу
поверхности, расположенную перпендикулярно к направлению скорости.

Выражение (\ref{eq:6.3}) показывает, что каждая единица массы жидкости как бы переносит с
собой при своем движении энергию $w+v^2/2$. Тот факт, что здесь стоит тепловая
функция $w$, а не просто внутренняя энергия $\varepsilon$, имеет простой
физический смысл. Подставив $w = \varepsilon + p/\rho$, напишем полный поток
энергии через замкнутую поверхность в виде
\[
   - \oint{\rho \vv \left(\frac{v^2}{2}+\varepsilon\right)\df}-
   \oint{p\vv \df}.
\]
Первый член есть энергия (кинетическая и внутренняя), непосредственно
переносимая (в единицу времени) проходящей через поверхность массой жидкости.
Второй же член представляет собой работу, производимую силами давления над
жидкостью, заключенной внутри поверхности.

