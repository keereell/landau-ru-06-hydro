\section{Уравиеиие Бернулли}
\label{sec:p5}

Уравнения гидродинамики заметно упрощаются в случае стационарного течения
жидкости. Под стационарным (или установившимся) подразумевают такое течение, при
котором в каждой точке пространства, занятого жидкостью, скорость течения
остается постоянной во времени. Другими словами, у является функцией одних
только координат, так $\partial\vv/\partial t$. Уравнение (\ref{eq:2.10}) сводится теперь
к равенству
\begin{equation}
   \label{eq:5.1}
   \frac{1}{2}\grad{v^2} - \vrv = - \grad w.
\end{equation}

Введем понятие о \textit{линиях тока} как линиях, касательные к которым
указывают направление вектора скорости в точке касания в данный момент времени;
они определяются системой дифференциальных уравнений
\begin{equation}
   \label{eq:5.2}
   \frac{dx}{v_x}=\frac{dy}{v_y}=\frac{dz}{v_z}
\end{equation}
при стационарном движении жидкости линии тока остаются неизменными во времени и
совпадают с траекториями частиц жидкости. При нестационарном течении такое
совпадение, разумеется, не имеет места: касательные к линии тока дают
направления скорости разных частиц жидкости в последовательных точках
пространства в определенный момент времени, в то время как касательные к
траектории дают направления скорости определенных частиц в последовательные
моменты времени.

Умножим уравнение (\ref{eq:5.1}) на единичный вектор касательной к линии тока в каждой ее
точке; этот единичный вектор обозначим $\vect l$. Проекция градиента на некоторое
направление равна, как известно, производной, взятой по этому направлению.
Поэтому искомая проекция от $\grad w$ есть $\partial w/ \partial l$. Что
касается вектора $\vrv$, то он перпендикулярен к скорости $\vect v$, и потому
его проекция на направление $\vect l$ равна нулю.

Таким образом, из уравнения (\ref{eq:5.1}) мы получаем:
\[
   \frac{\partial}{\partial l}\left( \frac{v^2}{2} + w \right) = 0.
\]
Отсюда следует, что величина $\frac{v^2}{2} + w$ постоянна вдоль линии тока:
\begin{equation}
   \label{eq:5.3}
   \frac{v^2}{2} + w = \const
\end{equation}
Значение $\const$, вообще говоря, различно для разных линий тока. Уравнение
(\ref{eq:5.3}) называют \textit{уравнением Бернулли} \footnote{Оно было установлено для несжимаемой жидкости (см. \S \ref{sec:p10}) \textit{Д. Бернулли} в 1788 г.}.

Если течение жидкости происходит в поле тяжести, то к правой части уравнения
(\ref{eq:5.1}) надо прибавить еще ускорение силы тяжести $\vect g$. Выберем направление
силы тяжести в качестве направления оси $z$, причем положительные значения $z$
отсчитываются вверх. Тогда косинус угла между направлениями $\vect g$ и $\vect
l$ равен производной $-dz/dl$, так что проекция $\vect g$ и $\vect l$ есть
\[
   -g \D{z}{l}.
\]
Соответственно этому будем иметь теперь
\[
   \frac{\partial}{\partial l}\left( \frac{v^2}{2} + w + gz\right) = 0.
\]
Таким образом, уравнение Бернулли гласит, что вдоль линий
тока остается постоянной сумма
\begin{equation}
   \label{eq:5.4}
    \frac{v^2}{2} + w + gz = \const
\end{equation}

