\section{Поверхности разрыва}\label{sec:p84}

В предыдущих главах мы рассматривали только такие течения, при которых распределение всех величин (скорости, давления, плотности и т.д.) в газе непрерывно.
Возможны, однако, и движения, при которых возникают разрывы непрерывности в распределении этих величин.

Разрыв непрерывности в движении газа имеет место вдоль некоторых поверхностей; при прохождении через такую поверхность указанные величины испытывают скачок.
Эти поверхности называют \emph{поверхностями разрыва}.
При нестационарном движении газа поверхности разрыва не остаются, вообще говоря, неподвижными; необходимо при этом подчеркнуть, что скорость движения поверхности разрыва не имеет ничего общего со скоростью движения самого газа. Частицы газа при своем движении могут проходить через эту поверхность, пересекая ее.

На поверхностях разрыва должны выполняться определенные граничные условия. Для формулирования этих условий рассмотрим какой-нибудь элемент поверхности разрыва и воспользуемся связанной с этим элементом системой координат с осью $x$, направленной по нормали к нему \footnote{Если движение нестационарно, то мы рассматриваем элемент поверхности в течение малого интервала времени.}.

Во-первых, на поверхности разрыва должен быть непрерывен поток вещества: количество газа, входящего с одной стороны, должно быть равно количеству газа, выходящему с другой стороны поверхности.
Поток газа через рассматриваемый элемент поверхности (отнесенный на единицу площади) равен $\rho v_x$.
Поэтому должно выполняться условие $\rho_1 v_{1x} = \rho_2 v_{2x}$, где индексы 1 и 2 относятся к двум сторонам поверхности разрыва.

Разность значений какой-либо величины с обеих сторон поверхности разрыва мы будем ниже обозначать посредством квадратных скобок; так,
\[
    \lbrack \rho v_x \rbrack = \rho_1 v_{1x} - \rho_2 v_{2x},
\]
и полученное условие напишется в виде

\begin{equation}
    \label{eq:84.1}
    \lbrack \rho v_x \rbrack = 0.
\end{equation}

Далее, должен быть непрерывным поток энергии. Поток энергии определяется выражением (\ref{eq:6.3}).
Поэтому мы получаём условие

\begin{equation}
    \label{eq:84.2}
    \left\lbrack \rho v_x ( \frac{v^2}{2} + w) \right\rbrack = 0.
\end{equation}


Наконец, должен быть непрерывен поток импульса, т.е. должны быть равны силы, с которыми действуют друг на друга газы по обеим сторонам поверхности разрыва.
Поток импульса через единицу площади равен (см. \S \ref{sec:p7})
\[
    pn_i + \rho v_i v_k n_k.
\]
Вектор нормали $\V n$ направлен по оси $x$. Поэтому непрерывность $x$ - компоненты потока импульса приводит к условию

\begin{equation}
    \label{eq:84.3}
    \lbrack p + \rho v^2_x \rbrack = 0,
\end{equation}

а непрерывность $y$- и $z$ - компонент дает

\begin{equation}
    \label{eq:84.4}
    \lbrack \rho v_x v_y \rbrack = 0, \quad \lbrack \rho v_x v_z \rbrack = 0.
\end{equation}

Уравнения (\ref{eq:84.1} - \ref{eq:84.4}) представляют собой полную систему граничных условий на поверхности разрыва.
Из них можно сразу сделать вывод о возможности существования двух типов поверхностей разрыва.

В первом случае через поверхность разрыва нет потока вещества.
Это значит, что $\rho_1 v_{1x} = \rho_2 v_{2x} = 0$. Поскольку $\rho_1$ и $\rho_1$ отличны от нуля, то это значит, что должно быть $v_{1x} = v_{2x} = 0$.


Условия (\ref{eq:84.2}) и (\ref{eq:84.4}) в этом случае удовлетворяются автоматически, а условие (\ref{eq:84.3}) дает $p_1 = p_2$.
Таким образом, на поверхности разрыва в этом случае непрерывны нормальная компонента скорости и давление газа:

\begin{equation}
    \label{eq:84.5}
    v_{1x} = v_{2x} = 0, \quad \lbrack p \rbrack = 0.
\end{equation}
Тангенциальные же скорости $v_y, v_z$ и плотность (а также другие термодинамические величины, кроме давления) могут испытывать произвольный скачок. Такие разрывы будем называть \emph{тангенциальными}.

Во втором случае поток вещества, а с ним и $v_{1x}$ и $v_{2x}$ отличны от нуля.
Тогда из (\ref{eq:84.1}) и (\ref{eq:84.4}) имеем:
\begin{equation}
    \label{eq:84.6}
    \lbrack v_y \rbrack = 0, \quad \lbrack v_z \rbrack = 0,
\end{equation}
т.е. тангенциальная скорость непрерывна на поверхности разрыва.
Плотность же, давление (а потому и другие термодинамические величины) и нормальная скорость испытывают скачок, причем скачки этих величин связаны соотношениями (\ref{eq:84.1} - \ref{eq:84.3}).
В условии (\ref{eq:84.2}) мы можем в силу (\ref{eq:84.1}) сократить $\rho v_x$, а вместо $v^2$ можно в силу непрерывности $v_x$ и $v_z$ писать $v^2_x$. Таким образом, на поверхности разрыва в рассматриваемом случае должны иметь место условия:

\begin{equation}
    \label{eq:84.7}
    \lbrack \rho v_x \rbrack = 0, \quad \left\lbrack \frac{v^2_x}{2} + w \right\rbrack = 0, \quad
    \lbrack p + \rho v^2_x \rbrack = 0.
\end{equation}
Разрывы этого типа называют \emph{ударными волнами}.

Если теперь вернуться к неподвижной системе координат, то вместо $v_x$ надо везде писать разность между нормальной к поверхности разрыва компонентой $v_n$ скорости газа и скоростью $u$ самой поверхности, направленной, по определению, по нормали к ней:

\begin{equation}
    \label{eq:84.8}
    v_x = v_n - u.
\end{equation}
Скорости $v_n$ и $u$ берутся относительно неподвижной системы отсчета.
Скорость $v_x$ есть скорость движения газа относительно поверхности разрыва; иначе можно сказать, что $-v_x = u - v_n$ есть скорость распространения самой поверхности разрыва относительно газа.
Обращаем внимание на то, что эта скорость различна по отношению к газу с обеих сторон поверхности (если $v_x$ испытывает разрыв).

Тангенциальные разрывы, на которых испытывают скачок касательные компоненты скорости, рассматривались нами уже в \S \ref{sec:p29}.
Там было показано, что в несжимаемой жидкости такие разрывы неустойчивы и должны размываться в турбулентную область.
Аналогичное исследование для сжимаемой жидкости показывает, что такая неустойчивость имеет место и в общем случае произвольных скоростей (см. задачу 1).

Частным случаем тангенциальных разрывов являются разрывы, в которых скорость непрерывна и испытывает скачок только плотность (а с ней и другие термодинамические величины за исключением давления); такие разрывы называют \emph{контактными}.
Сказанное выше о неустойчивости, к ним не относится.


\subsubsection*{Задачи}
\paragraph*{1}
Исследовать устойчивость (по отношению к бесконечно малым возмущениям) тангенциальных разрывов в однородной сжимаемой среде (газ или жидкость).

\texttt{Решение.} Вычисления аналогичны произведенным в \S \ref{sec:p29} для несжимаемой жидкости.
Как и там, по нормали к поверхности направим ось $z$.

В среде 2 (со скоростью $\V v_2 = 0, z<0$) давление удовлетворяет уравнению

\[
    \ddot p'_2 - c^2 \nabla p'_2 = 0
\]
(вместо уравнения Лапласа (\ref{eq:29.2}) в несжимаемой жидкости). Ищем $p'_2$ в виде

\[
    p'_2 = \const \cdot \exp ( -i\omega t + iqx + ix_2z),
\]
где волновое число "ряби" иа поверхности обозначено через $q$ (вместо $k$ в \S \ref{sec:p29}); если $x_2$ комплексно, то оно должно быть выбрано так, чтобы было $\Im x_2 <0$.

Волновое уравнение приводит к соотношению

\begin{equation}
    \label{eq:84T1}
    \omega^2 = c^2 (q^2 + x^2_2).
\end{equation}
Вместо (\ref{eq:29.7}) тем же образом находим теперь

\[
    p'_2 = \xi\rho\omega^2/ix_2.
\]

В газе 1, движущемся со скоростью $\V v_1 = \V v \quad (z>0)$, ищем $p'_i$ в виде
\[
    p'_` = \const \cdot \exp ( -i\omega t + iqx - ix_1z),
\]

Для упрощения выводов предположим сначала, что скорость $\V v$ направлена тоже по оси $x$. Соотношение между $\omega$, $q$, $x_1$ дается формулой

\begin{equation}
    \label{eq:84T2}
    (\omega -vq)^2 = c^2(q^2+x^2_1)
\end{equation}
(ср. (\ref{eq:68.1})). Вместо (\ref{eq:29.6}) получаем теперь
\[
    p'_1 = -\xi(\omega-qv)^2\rho/ix_i,
\]
и условие $p'_1 = p'_2$ приводит к уравнению

\begin{equation}
    \label{eq:84T3}
    \frac{x_1}{(\omega - qv)^2} + \frac{x_2}{\omega^2} = 0.
\end{equation}
От сделанного выше предположения о направлении скорости $\V v$ можио избавиться, заметив, что невозмущеиная скорость входит в исходные линеаризованные уравнение непрерывности и уравнение Эйлера только в комбинации $(\V v \nabla)$ (соответственно в членах $(\V{v}\nabla)p'$ и $(\V{v}\nabla)\V{v}'$.
Поэтому для перехода к произвольному направлению $\V{v}$ (в плоскости $xy$) достаточно заменить в \ref{eq:84T1} - \ref{eq:84T3} $v$ на $v\cos\varphi$, где $\varphi$ - угол между $\V{v}$ и $\V{q}$ (ср. примечание на с. 155).


Исключив $x_1, x_2$ из (\ref{eq:84T1})--(\ref{eq:84T3}), получим следующее дисперсионное уравнение для определения частоты возмущения $\omega$ по волновому числу $q$:

\begin{equation}
    \label{eq:84T4}
    \left\lbrack \frac1{\omega^2} -  \frac1{(\omega - qv\cos\varphi)^2}\right\rbrack
    \left\lbrack \frac1{c^2q^2} - \frac1{\omega^2} -  \frac1{(\omega - qv\cos\varphi)^2}\right\rbrack
    = 0.
\end{equation}
Корень первого множителя
\begin{equation}
    \label{eq:84T5}
    \omega = \frac{1}{2}qv\cos\varphi
\end{equation}
всегда веществен. Корни второго множителя:
\begin{equation}
    \label{eq:84T6}
    \omega = \frac{1}{2}qv\cos\varphi\pm
    q \left\lbrack \frac{1}{4}v^2\cos^2\varphi+c^2 \pm c(c^2+v^2\cos^2\varphi)^{1/2}\right\rbrack^{1/2};
\end{equation}
эти корни вещественны только при $v\cos\varphi>v_k$, где
\begin{equation}
    \label{eq:84T7}
    v_k = 2^{3/2}c.
\end{equation}

Таким образом, при $v\cos\varphi<v_k$ дисперсиоииое уравнение имеет пару комплексно-сопряжеииых корней, длй одного из которых будет $\Im\omega>0$;
Соответствующие возмущения приводят к неустойчивости.
При $v<v_k$ таковы возмущения с любым углом $\varphi$, а при $v>v_k$ неустойчивы только возмущения с $\cos\varphi<v_k/v$.
В результате тангенциальный разрыв неустойчив всегда.
Отметим, что сам факт неустойчивости (если не интересоваться по отношению к каким именно возмущениям) очевиден уже из неустойчивости в случае несжимаемой жидкости в совокупности с тем обстоятельством, что в дисперсионное уравнение скорость $v$ входит только в комбинации $v\cos\varphi$:
какова бы не была скорость $v$, найдутся такие углы $\varphi$, для которых $v\cos\varphi\gg c$, так что по отношению к таким возмущениям среда ведет себя как несжимаемая
\footnote{Значение (\ref{eq:84T7}) получено Л. Д. Ландау (1944). Необходимость учета в этой задаче неколлииеарности $v$ и $q$ указана С. И. Сыроватским (1954).}.

\paragraph*{2} На тангенциальный разрыв в однородной сжимаемой среде падает плоская звуковая волна; определить иитенсивиости отраженной от разрыва волны и волны, преломленной на нем (\emph{J.W. Miles}, 1957; \emph{Н.S. Ribner}, 1957).

\texttt{Решение.} Выбираем оси координат, как в предыдущей задаче, причем скорость $v$ (в среде 1, $z>0$) направлена по оси $x$.
Пусть звуковая волиа падает из неподвижной среды (среда 2, $z<0$);
направление ее волнового вектора к задается сферическими углами $\theta$ и $\varphi$;
угол $\theta$ -- между $\V k$ и осью $z$, угол $\varphi$ -- между проекцией $k$ на плоскость $xy$ (обозначим ее $\V q$) и скоростью $\V v$:
\[
   k_x=q\cos\varphi,\quad
   k_y=q\sin\varphi,\quad
   k_z=\frac{\omega}{c}\sin\varphi,\quad
   q=\frac{\omega}{c}\sin\theta,
   \]
причем $0<\theta<\pi/2$ (волна падает в положительном направлении оси $z$).
В среде 2 ищем давление в виде
\[
p'_2 = e^{i(k_xx+k_yy-\omega t)}(e^{ik_zz}+Ae^{-ik_zz}),
\]
где $A$ -- амплитуда отраженной волны, а амплитуда падающей волны условно принята за единицу.
В среде 1 имеем одиу преломленную волну:
\[
p'_1 =Be^{i(k_xx+k_yy+xz-\omega t)},
\]
где $x$ удовлетворяет уравнению
\[
(\omega - vk_x)^2 = c^2 (k_x^2+k_y^2+x^2)
\]
(ср.  (\ref{eq:84T2}).
Амплитуды $A$ и $B$ определяются из условий непрерывности давления и вертикального смещения жидких частиц по обе стороны разрыва: $p_1' = p_2'$ при $ z = 0, \xi_1 = \xi_2 \equiv \xi.$ Это дает два уравнения
\[
    1 + A = B, \frac{\varkappa}{(\omega - vk_x)^2} B = \frac{k_z}{\omega^2}(1-A),
\]
откуда


\begin{equation}
    \label{eq:84T8}
    A = \frac{(\omega - vk_x)^2/\varkappa - \omega^2/k_z}
             {(\omega - vk_x)^2/\varkappa + \omega^2/k_z}, \quad
    B = \frac{2(\omega - vk_x)^2/\varkappa}
             {(\omega - vk_x)^2/\varkappa + \omega^2/k_z},
\end{equation}
чем и решается поставленная задача. Знак величины $\varkappa$,
\[
    \varkappa^2 = \frac{\omega^2}{c^2}
    \lbrack(1-M\cos\theta\cos\varphi)^2 - \sin^2\theta \rbrack, \quad
    M = \frac{v}{c},
\]
должен быть выбран с учетом предельных условий при $z \rightarrow \infty$: скорость преломленной волны должна быть направлена от разрыва, т.е.

\[
    U_z = \frac{\partial\omega}{\partial\varkappa} =
    \frac{c^2\varkappa}{\omega - vk_x}<0.
\]

Из полученных формул видно, что возможны три различных режима отражения.

1) При $M\cos\varphi<1/\sin\theta-1$ величина $\varkappa$ вещественна, а поскольку $\omega-vk_x>0$, то согласно условию (9) $\varkappa>0$.
Из (\ref{eq:84T8}) видно, что при этом $|A|<1$ --- отражение происходит с ослаблением волны.

2) При $ 1/\sin\theta-1<M\cos\varphi<1/\sin\theta+1 $ величина $\varkappa$ мнима и $|A|=1$, --- происходит полное внутреннее отражение звуковой волны.

3) При $ M\cos\varphi>1+1/\sin\theta $ (что возможно лишь при $ M>2 $) величина $\varkappa$ снова вещественна, но теперь надо выбрать $ \varkappa<0 $.
Согласно (\ref{eq:84T8}) при этом $ |A|>1 $, т.е. отражение происходит с усилением волны.
Более того, знаменатели выражений (\ref{eq:84T8}) $ \varkappa<0 $ могут обратиться в нуль при определенных углах падения волны, и тогда коэффициент отражения обращается в бесконечность.
Поскольку этот знаменатель совпадает (с точностью до обозначений) с левой стороной уравнения (\ref{eq:84T3}) предыдущей задачи, то можно сразу заключить, что "резонансные" углы падения определяются равенствами (\ref{eq:84T5}) и (\ref{eq:84T6} (последнее --- при $M>2^{3/2}$).
В свою очередь, бесконечность коэффициента отражения (и прохождения), т.е. конечность амплитуды отраженной волны прн стремящейся к нулю амплитуде падающей волны, означает возможность спонтанного излучения звука поверхностью разрыва: раз созданное на ней возмущение (рябь) неограниченно долго продолжает излучать звуковые волны, не затухая и не усиливаясь прн этом; энергия, уносимая излучаемым звуком, черпается из всей движущейся среды.

Плотность потока энергии (усредненная по времени) в преломленной волне
\[
    \overline{q_2} = U_z\overline{E_2}=\frac{c^2\varkappa}{\omega-vk_x}
    \frac{\omega}{\omega-vk_x}\frac{{|B|}^2}{2\rho c^2}
\]
($E_z$ из (\ref{eq:68.3})).
В случае $3$ имеем $x<0$, а потому и $\overline{q_2}<0$, --- энергия приходит к разрыву из движущейся среды, что и служит источником усиления.
При спонтанном излучении звука эта приходящая энергия совпадает с энергией, уносимой волной, уходящей в неподвижную среду.

В изложенном решении задачи неустойчивость поверхности разрыва не учитывается.
Формальная корректность такой постановки задачи связана с тем, что звуковые волны и неустойчивые поверхностные (затухающие при $z\rightarrow\pm\infty$) волны представляют собой линейно независимые колебательные моды.
Физическая же корректность требует соблюдения специальных условий (например, начальных), в которых поверхностные волны еще достаточно слабы.

\begin{comment}
\end{comment}

% vim: filetype plugin off
