\section{Сила сопротивления при потенциальном обтекании}
\label{sec:p11}

Рассмотрим задачу о потенциальном обтекании несжимаемой идеальной жидкостью
какого-либо твердого тела. Такая задача, конечно, полностью эквивалентна задаче
об определении течения жидкости при движении через нее того же тела. Для
получения второго случая из первого достаточно перейти к системе координат, в
которой жидкость на бесконечности покоится. Мы будем говорить ниже именно о
движении твердого тела через жидкость.

Определим характер распределения скоростей в жидкости на больших расстояниях от
движущегося тела. Потенциальное движение несжимаемой жидкости определяется
уравнением Лапласа $\nabla \varphi = 0$. Мы должны рассмотреть такие решения
этого уравнения, которые обращаются на бесконечности в нуль, поскольку жидкость
на бесконечности неподвижна. Выберем начало координат где-нибудь внутри
движущегося тела (эта система координат движется вместе с телом; мы, однако,
рассматриваем распределение скоростей в жидкости в некоторый заданный момент
времени). Как известно, уравнение Лапласа имеет решением $1/r$, где $r$ -
расстояние от начала координат. Решением являются также градиент $\nabla(1/r)$ и
следующие производные от $1/r$ по координатам. Все эти решения (и их линейные
комбинации) обращаются на бесконечности в нуль. Поэтому общий вид искомого
решения уравнения Лапласа на больших расстояниях от тела есть
\[
   \varphi = - \frac{a}{r} + \vect A \nabla \frac1{r} + \dots ,
\]
где $a$, $\vect A$ не зависят от координат; опущенные члены содержат
производные высших порядков от $1/r$. Легко видеть, что постоянная $a$ должна
быть равной нулю. Действительно, потенциал $\varphi = -a/r$ дает скорость
\[
   \vect v = - \nabla \frac{a}{r} = \frac{a \vect r}{r^3}.
\]
Вычислим соответствующий поток жидкости через какую-нибудь замкнутую
поверхность, скажем, сферу с радиусом $R$. На этой поверхности скорость
постоянна и равна $a/R^2$; поэтому полный поток жидкости через нее равен
$\rho(a/R^2)4\pi R^2 = 4\pi \rho a$. Между тем, поток несжимаемой жидкости
через всякую замкнутую поверхность должен, очевидно, обращаться в нуль. Поэтому
заключаем, что должно быть $a = 0$.

Таким образом, $\varphi$ содержит члены, начиная с членов порядка $1/r^2$.
Поскольку мы ищем скорость на больших расстояниях, то члены более высоких
порядков можно опустить, и мы получаем:
\begin{equation}
   \label{eq:11.1}
   \varphi = \vect A \nabla \frac1{r} = - \frac{\vect{An}}{r^2},
\end{equation}
а для скорости $\vect v = \grad \varphi$
\begin{equation}
   \label{eq:11.2}
   \vect v = (\vect A \nabla) \nabla \frac1{r} = \frac{3\vect{(An)n - A})}{r^3}
\end{equation}
($\vect n$ — единичный вектор в направлении $\vect r$). Мы видим, что на больших
расстояниях скорость падает, как $1/r^3$. Вектор $\vect A$ зависит от конкретной
формы и скорости движения тела и может быть определен только путем полного
решения уравнения $\nabla \varphi = 0$ на всех расстояниях, с учетом
соответствующих граничных условий на поверхности движущегося тела.

Входящий в (\ref{eq:11.2}) вектор $\vect A$ связан определенным образом с полным
импульсом и с полной энергией жидкости, обтекающей движущееся в ней тело. Полная
кинетическая энергия жидкости (внутренняя энергия несжимаемой жидкости
постоянна) есть
\[
   E = \frac{\rho}{2}\int v^2\; dV,
\]
где интегрирование производится по всему пространству вне тела. Выделим из
пространства часть $V$, ограниченную сферой большого радиуса $R$, с центром в
начале координат и будем интегрировать сначала только по объему $V$, имея в виду
стремить затем $R$ к бесконечности. Имеем тождественно
\[
   \int v^2\; dV = \int u^2\; dV + \int \vect{(v+u)(v-u)\;dV},
\]
где $\vect u$ - скорость тела. Поскольку $u$ есть не зависящая от координат
величина, то первый интеграл равен просто $u^2(V-V_0)$, где $V_0$ - объем тела.
Во втором же интеграле пишем сумму $\vect{v+u}$ в виде
$\nabla(\varphi+\vect{ur})$ и, воспользовавшись также тем, что $\Div \vv =0$ в
силу уравнения непрерывности, а $\Div \vect u \equiv 0$, имеем:
\[
   \int v^2\; dV = u^2(V-V_0) + \int \Div \lbrace \vect{(\varphi + ur)(v-u)} \rbrace dV.
\]
Второй интеграл преобразуем в интеграл по поверхности $S$ сферы и поверхности $S_0$
тела:
\[
   \int v^2\; dV = u^2(V-V_0) + \oint_{S+S_0} \vect{(\varphi+ur)(v-u)}\;\df .
\]
На поверхности тела нормальные компоненты $\vv$ и $\vect u$ равны друг другу в
силу граничных условий; поскольку вектор $\df$ направлен как раз по нормали к
поверхности, то ясно, что интеграл по $S_0$ тождественно обращается в нуль. На
удаленной же поверхности $S$ подставляем для $\varphi$ и $\vv$ выражения
(\ref{eq:11.1} - \ref{eq:11.2}) и опускаем члены, обращающиеся в нуль при переходе к пределу по $R \to
\infty$. Написав элемент поверхности сферы $S$ в виде $\df = \vect n R^2\; do$,
где $do$ — элемент телесного угла, получим:
\[
   \int v^2\; dV = u^2 \left( \frac{4\pi}{3}R^3 - V_0 \right) +
   \int \lbrace 3 \vect{(An)(un) - (un)^2}R^3 \rbrace\; do .
\]
Наконец, произведя интегрирование
\footnote{
Интегрирование по $do$ эквивалентно усреднению подинтегрального выражения по всем
направлениям вектора $\vect n$ и умножению затем на $4\pi$. Для усреднения выражений
типа $(\vect{An})(\vect{Bn}) \equiv A_i n_i B_k n_k$ ($\vect{A,B}$ - постоянные векторы),
пишем
\[
\overline{\vect{(An)(Bn)}} = A_i B_k \overline{n_i n_k} =
\frac1{3} \delta_{ik} A_i B_k = \frac1{3}\vect{AB}.
\]
}
и умножив на $\rho/2$, получаем окончательно
следующее выражение для полной энергии жидкости:
\begin{equation}
   \label{eq:11.3}
   E = \frac{\rho}{2}(4\pi \vect{Au} - V_0 u^2) .
\end{equation}

Как уже указывалось, точное вычисление вектора $\vect A$ требует полного решения
уравнения $\nabla \varphi = 0$ с учетом конкретных граничных условий на
поверхности тела. (Эбщий характер зависимости $\vect A$ от скорости $\vect u$
тела можно, однако, установить уже непосредственно из факта линейности уравнения
для $\varphi$ и линейности (как по $\varphi$, так и по $\vect u$) граничных
условий к этому уравнению. Из этой линейности следует, что $\vect A$ должно быть
линейной же функцией от компонент вектора $u$. Определяемая же формулой (\ref{eq:11.3})
энергия $E$ является, следовательно, квадратичной функцией компонент вектора
$\vect u$ и потому может быть представлена в виде
\begin{equation}
   \label{eq:11.4}
       E = \frac{m_{ik}u_i u_k}{2},
\end{equation}


где $m_{ik}$ - некоторый постоянный симметрический тензор, компоненты которого
могут быть вычислены с помощью компонент вектора $\vect A$; его называют
тензором \textit{присоединенных масс}.

Зная энергию $E$, можно получить выражение для полного импульса $\vect P$
жидкости. Для этого замечаем, что бесконечно малые изменения $E$ и $\vect P$
связаны друг с другом соотношением $dE = \vect u\; d\vect P$
\footnote{
Действительно, пусть тело ускоряется под влиянием какой-либо внешней силы $F$.
В результате импульс жидкости будет возрастать; пусть $d\vect{P}$ есть его приращение
в течение времени $dt$. Это приращение связано с силой посредством $d\vect{P} = \vect{F}dt$,
в умноженное на скорость $\vect u$ дает $\vect u d \vect P$, т.е. работу силы $\vect F$ на
пути $\vect u dt$, которая в свою очередь должна быть равна увеличению энергии $dE$ жидкости.

Следует заметить, что вычисление импульса непосредственно как интеграла $\int \rho \vect v dV$
по всему объему жидкости было бы невозможным. Дело в том, что этот интеграл (со скоростью $\vect v$, распределенной по (\ref{eq:11.2})) расходится в том смысле, что результат интегрирования, хотя и конечен, но зависит от способа взятия интеграла: производя
интегрирование по большой области, размеры которой устремляют затем к бесконечности, мы
получили бы значение, зависящее от формы области (сфера, цилиндр и т.п.). Используемый же
нами способ вычисления импульса, исходя из соотношения $u \vect P = dE$, приводит ко
воплне определенному конечному значению (даваемому формулой (\ref{eq:11.6}),
заведомо удовлетворящему физическому условию о связи изменения импульса с действующим на тело
силами.
}; отсюда следует,
что если $E$ выражено в виде (\ref{eq:11.4}), то компоненты $\vect P$ должны иметь вид
\begin{equation}
   \label{eq:11.5}
       P_i = m_{ik} u_k .
\end{equation}
Наконец, сравнение формул (\ref{eq:11.3} - \ref{eq:11.5}) показывает, что $\vect P$ выражается
через $\vect A$ следующим образом:
\begin{equation}
   \label{eq:11.6}
   \vect P = 4\pi \rho \vect A - \rho V_0 \vect u.
\end{equation}
Следует обратить внимание на то, что полный импульс жидкости оказывается вполне
определенной конечной величиной.

Передаваемый в единицу времени от тела к жидкости импульс есть $d\vect P/dt$.
Взятый с обратным знаком, он определяет, очевидно, реакцию $\vect F$ жидкости,
т. е. действующую на тело силу:
\begin{equation}
   \label{eq:11.7}
       \vect F = - \D{\vect P}{t}.
\end{equation}
Параллельная скорости тела составляющая $\vect F$ называется \textit{силой
сопротивления}, а перпендикулярная составляющая - \textit{подъемной силой}.

Если бы было возможно потенциальное обтекание равномерно движущегося в идеальной
жидкости тела, то было бы $\vect P = \const$ (так как $\vect u = \const$) и
$\vect F = 0$. Другими словами, отсутствовала бы как сила сопротивления, так и
подъемная сила, т. е. действующие на поверхность тела со стороны жидкости силы
давления взаимно компенсируются (так называемый парадокс Даламбера).
Происхождение этого "парадокса" в особенности очевидно для силы сопротивления.
Действительно, наличие этой силы при равномерном движении тела означало бы, что
для поддержания движения какой-либо внешний источник должен непрерывно
производить работу, которая либо диссипи-руется в жидкости, либо преобразуется в
ее кинетическую энергию, приводя к постоянно уходящему на бесконечность потоку
энергии в движущейся жидкости. Но никакой диссипации энергии в идеальной
жидкости, по определению, нет, а скорость приводимой телом в движение жидкости
настолько быстро убывает с увеличением расстояния от тела, что никакого потока
энергии на бесконечности тоже нет.

Следует, однако, подчеркнуть, что все эти соображения относятся лишь к движению
тела в неограниченной жидкости. Если же, например, жидкость имеет свободную
поверхность, то равномерно движущееся параллельно этой поверхности тело будет
испытывать силу сопротивления. Появление этой силы (называемой \textit{волновым
сопротивлением}) связано с возникновением на свободной поверхности жидкости
системы распространяющихся по ней волн, непрерывно уносящих энергию на
бесконечность.

Пусть некоторое тело совершает под влиянием действующей на него внешней силы
$\vect f$ колебательное движение. При соблюдении рассмотренных в предыдущем
параграфе условий окружающая тело жидкость совершает потенциальное движение, и
для вывода уравнений движения тела можно воспользоваться полученными выше
соотношениями. Сила $\vect f$ должна быть равна производной по времени от
полного импульса системы, равного сумме импульса $M\vect u$ тела ($M$ - масса
тела) и импульса $P$ жидкости:
\[
    M \D{\vect u}{t} + \D{\vect P}{t} = \vect f.
\]
С помощью (\ref{eq:11.5}) получаем отсюда:
\[
   M \D{u_i}{t} + m_{ik}\D{u_k}{t} = f_i ,
\]
что можно написать также и в виде
\begin{equation}
   \label{eq:11.8}
       \D{u_k}{t}(M \delta_{ik} + m_{ik}) = f_i.
\end{equation}
Это и есть уравнение движения тела, погруженного видеальную жидкость.

Рассмотрим теперь в некотором смысле обратный вопрос. Пусть сама жидкость
производит под влиянием каких-либо внешних (по отношению к телу) причин
колебательное движение. Под влиянием этого движения погруженное в жидкость тело
тоже начинает двигаться.
\footnote{Реч может идти, например, о движении тела в жидкости, по которой распространяется звуовая волна с длиной волны, большей по сравнению с размерами тела.}
Выведем уравнение этого движения.

Будем предполагать, что скорость движения жидкости мало меняется на расстояниях
порядка величины линейных размеров тела. Пусть $\vv$ есть скорость жидкости в
месте нахождения тела, которую она имела бы, если бы тела вообще не было;
другими словами, $\vv$ есть скорость основного движения жидкости. По сделанному
предположению $\vv$ можно считать постоянной вдоль всего объема, занимаемого
телом. Посредством $\vect u$ по-прежнему обозначаем скорость тела.

Силу, действующую на тело и приводящую его в движение, можно определить из
следующих соображений. Если бы тело полностью увлекалось жидкостью (т. е. было
бы $\vect v = \vect u$), то на него действовала бы такая же сила, которая бы
действовала на жидкость в объеме тела, если бы тела вовсе не было. Импульс этого
объема жидкости есть $\rho V_0 \vv$, и потому действующая на него сила равна
$\rho V_0 \D{\vv}{t}$. Но в действительности тело не увлекается полностью
жидкостью; возникает движение тела относительно жидкости, в результате чего сама
жидкость приобретает некоторое дополнительное движение. Связанный с этим
дополнительным движением импульс жидкости равен $m_{ik}(u_k-v_k)$ (в выражении
(\ref{eq:11.5}) надо теперь писать вместо $\vect u$ скорость $\vect{u-v}$ движения тела
относительно жидкости). Изменение этого импульса со временем приводит к
появлению дополнительной силы реакции, действующей на тело и равной
$-m_{ik}d(u_k-v_k)$. Таким образом, полная сила, действующая на тело, равна
\[
   \rho V_0 \D{v_i}{t} - m_{ik} \frac{d}{dt}(u_k-v_k).
\]
Эту силу надо приравнять производной по времени от импульса тела. Таким образом,
приходим к следующему уравнению движения:
\[
   \frac{d}{dt}Mu_i = \rho V_0 \D{v_i}{t} - m_{ik} \frac{d}{dt}(u_k-v_k).
\]
Интегрируя с обеих сторон по времени, получаем отсюда:
\begin{equation}
    \label{eq:11.9}
   (M\delta_{ik}+m_{ik})u_k = (m_{ik}+\rho V_0 \delta_{ik})v_k.
\end{equation}
Постоянную интегрирования полагаем равной нулю, поскольку скорость $\vect u$
тела, приводимого жидкостью в движение, должна обратиться в нуль вместе со
скоростью жидкости $\vv$. Полученное соотношение определяет скорость тела по
скорости жидкости. Если плотность тела равна плотности жидкости ($M=\rho V_0$),
то. как и следовало ожидать, $\vect{u=v}$.

\subsection*{Задачи}
\paragraph*{1}
Получить уравнение движения для шара, совершающего колебательное движение в
идеальной жидкости, и для шара, приводимого в движение колеблющейся жидкостью.

\texttt{Решение.} Сравнивая (\ref{eq:11.1}) с выражением для $\varphi$, полученным для
обтекания шара в задаче 2 \S10, видим, что
\[
   \vect A = \vect u R^3/2
\]
($R$ — радиус шара). Полный импульс приводимой шаром в движение жидкости есть
согласно (\ref{eq:11.6}) $\vect P = 2\pi \rho R^3 \vect u/3$, так что теизор $m_{ik}$
равен
\[
   m_{ik} = \frac{2\pi}{3}\rho R^3 \delta_{ik}.
\]
Испытываемая движущимся шаром сила сопротивления равна
\[
   \vect F = - \frac{2\pi}{3}\rho R^3 \D{\vect u}{t},
\]
а уравнение движения колеблющегося в жидкости шара гласит:
\[
   \frac{4\pi R^3}{3}\left( \rho_0 + \frac{\rho}{2} \right)\D{\vect u}{t} =
   \vect f
\]
($\rho_0$ - плотность вещества шара). Коэффициент при $\vect u$ можно
рассматривать как некоторую эффективную массу шара; она складывается из массы
самого шара и из присоединенной массы, равной в данном случае половине массы
жидкости, вытесняемой шаром.

Если шар приводится в движение жидкостью, то для его скорости
получаем из (\ref{eq:11.9}) выражение
\[
   \vect u =\frac{3\rho}{\rho + 2\rho_0}\vv.
\]
Если плотность шара превышает плотность жидкости ($\rho_0 > \rho$), то $u<v$, т,
е, шар отстает от жидкости; если же $\rho_0 < \rho$, то шар опережает ее,

\paragraph*{2}
Выразить действующий на движущееся в жидкости тело момент сил через вектор
$\vect A$.

\texttt{Решение.} Как известно из механики, действующий на тело момент сил
$\vect M$ определяется по его функции Лагранжа (в данном случае - по энергии
$E$) соотношением $\delta E = \vect{M\delta\theta}$, где $\delta\theta$ - вектор
бесконечно малого угла поворота тела, а $\delta E$ - изменение $E$ при этом
повороте. Вместо того чтобы поворачивать тело на угол $\delta\theta$ (и
соответственно менять компоненты $m_{ik}$), можно повернуть на угол
$-\delta\theta$ жидкость относительно тела и соответственно изменить скорость
$\vect u$. Имеем при повороте $\delta\vect u = - \lbrack \delta\theta\vect u
\rbrack$, так что
\[
   \delta E = \vect P \delta \vect u = - \delta\theta \lbrack \vect{uP} \rbrack
\]
Используя выражение (\ref{eq:11.6}) для $\vect P$, получаем отсюда искомую
формулу
\[
   \vect M = - \lbrack \vect{uP} \rbrack = 4\pi\rho\lbrack\vect{Au}\rbrack
\]

