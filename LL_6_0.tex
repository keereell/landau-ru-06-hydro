\chapter{Предисловие к третьему изданию}

В двух предыдущих изданиях (1944 и 1953 гг.) гидродинамика составляла первую
часть "Механики сплошных сред"; теперь она выделена в отдельный том. Характер
содержания и изложения в этой книге определен в воспроизводимом ниже предисловии
к предыдущему изданию. Моей основной заботой при переработке и дополнении было
не изменить этот характер. Несмотря на протекшие 30 лет материал, содержавшийся
во втором издании, фактически не устарел --- за очень незначительными
исключениями. Этот материал подвергся лишь сравнительно небольшим добавлениям и
изменениям. В то же время добавлен ряд новых параграфов --- около пятнадцати по
всей книге. За последние десятилетия гидродинамика развивалась чрезвычайно
интенсивно и соответственно необычайно расширилась литература по этой науке. Но
ее развитие в значительной степени шло по прикладным направлениям, а также в
направлении усложнения доступных теоретическому расчету (в том числе с
использованием ЭВМ) задач. К последним относятся, в частности, разнообразные
задачи о неустойчивостях и их развитии, в том числе в нелинейном режиме. Все эти
вопросы лежат вне рамок данной книги; в частности вопросы устойчивости
излагаются (как и в предыдущих изданиях), в основном, результативным образом.

Не включена в книгу также и теория нелинейных волн в диспергирующих средах,
составляющая в настоящее время значительную главу математической физики. Чисто
гидродинамическим объектом этой теории являются волны большой амплитуды на
поверхности жидкости. Основные же ее физические применения связаны с физикой
плазмы, нелинейной оптикой, различными электродинамическими задачами и др.; в
этом смысле она относится к другим томам.

Существенные изменения произошли в понимании механизма возникновения
турбулентности. Хотя последовательная теория турбулентности принадлежит еще
будущему, есть основания полагать, что ее развитие вышло, наконец, на правильный
путь. Относящиеся сюда основные существующие к настоящему времени идеи и
результаты изложены в трех параграфах (\S30--32), написанных мной совместно с
М.И. Рабиновичем; я глубоко благодарен ему за оказанную таким образом большую
помощь. В механике сплошных сред возникла в последние десятилетия новая область
--механика жидких кристаллов. Она несет в себе одновременно черты, свойственные
механикам жидких и упругих сред. Изложение ее основ предполагается включить в
новое издание "Теории упругости".

Среди книг, которые мне довелось написать совместно с Львом Давидовичем Ландау,
эта книга занимает особое место. Он вложил в нее часть своей души. Новая для
Льва Давидовича в то время область теоретической физики увлекла его, и---как это
было для него характерно---он принялся заново продумывать и выводить для себя ее
основные результаты. Отсюда родился ряд его оригинальных работ, опубликованных в
различных журналах. Но ряд принадлежащих Льву Давидовичу и вошедших в книгу
оригинальных результатов или точек зрения не были опубликованы отдельно, а в
некоторых случаях даже его приоритет выяснился лишь позднее. В новом издании
книги во всех известных мне подобных случаях я добавил соответствующие указания
на его авторство.

При переработке этого, как и других томов "Теоретической физики", меня 
поддерживали помощь и советы многих моих друзей и товарищей по работе. Я хотел 
бы в первую очередь упомянуть многочисленные обсуждения с Г. И. Баренблаттом, 
Я. Б. Зельдовичем, Л. П. Питаевским, Я. Г. Синаем. Ряд полезных указаний я 
получил от А. А. Андронова, С. И. Анисимова, В. А. Белоконя, В. П. Крайнова, 
А. Г. Куликовского, М. А. Ли-бермана, Р. В. Половина, А. В. Тимофеева, 
А. Л. Фабриканта. Всем им я хочу выразить здесь свою искреннюю благодарность.

\begin{flushright}
Е. М. Лифшиц 
\end{flushright}
\begin{flushright}
Институт физических проблем АН СССР Август 1984
\end{flushright}

\chapter{Предисловие ко второму изданию}

Предлагаемая книга посвящена изложению механики сплошных сред, т.е. теории
движения жидкостей и газов (гидродинамике) и твердых тел (теории упругости).
Являясь по существу областями физики, эти теории благодаря ряду своих
специфических особенностей превратились в самостоятельные науки.

В теории упругости существенную роль играет решение математически четко
поставленных задач, связанных с линейными дифференциальными уравнениями в
частных производных; поэтому теория упругости содержит в себе много элементов
так называемой математической физики.

Гидродинамика имеет существенно иной характер. Ее уравнения нелинейны, и потому
прямое их исследование и решение возможны лишь в сравнительно редких случаях.
Благодаря этому развитие современной гидродинамики возможно лишь в непрерывной
связи с экспериментом. Это обстоятельство сильно сближает ее с другими областями
физики.

Несмотря на свое практическое обособление от других областей физики,
гидродинамика и теория упругости тем не менее имеют большое значение как части
теоретической физики. С одной стороны, они являются областями применения общих
методов и законов теоретической физики, и ясное понимание их невозможно без
знания основ других разделов последней. С другой стороны, сама механика сплошных
сред необходима для решения задач из совершенно других областей теоретической
физики.

Мы хотели бы сделать здесь некоторые замечания о характере изложения
гидродинамики в предлагаемой книге. Эта книга излагает гидродинамику как часть
теоретической физики, и этим в значительной мере определяется характер ее
содержания, существенно отличающийся от других курсов гидродинамики. Мы
стремились с возможной полнотой разобрать все представляющие физический интерес
вопросы. При этом мы старались построить изложение таким образом, чтобы создать
по возможности более ясную картину явлений и их взаимоотношений. В соответствии
с таким характером книги мы не излагаем в ней как приближенных методов
гидродинамических расчетов, так и тх из эмпирических теорий, которые не имеют
более глубокого физического обоснования. В то же время здесь излагаются такие
предметы, как теория теплопередачи и диффузия в жидкостях, акустика и теория
горения, которые обычно выпадают из курсов гидродинамики.

В настоящем, втором, издании книга подвергнута большой переработке. Добавлено
значительное количество нового материала, в особенности в газодинамике, почти
полностью написанной заново. В частности, добавлено изложение теории
околозвукового движения. Этот вопрос имеет важнейшее принципиальное значение для
всей газодинамики, так как изучение особенностей, возникающих при переходе через
звуковую скорость, должно дать возможность выяснения основных качественных
свойств стационарного обтекания твердых тел сжимаемым газом. В этой области до
настоящего времени еще сравнительно мало сделано; многие важные вопросы могут
быть еще только поставлены. Имея в виду необходимость их дальнейшей разработки,
мы даем подробное изложение применяемого здесь математического аппарата.

Добавлены две новые главы, посвященные релятивистской гидродинамике и
гидродинамике сверхтекучей жидкости. Релятивистские гидродинамические уравнения
(глава XV) могут найти применение в различных астрофизических вопросах, например
при изучении объектов, в которых существенную роль играет излучение;
своеобразное поле применения этих уравнений открывается также и в совершенно
другой области физики, например, в теории множественного образования частиц при
столкновениях. Излагаемая в главе XVI "двухскоростная" гидродинамика дает
макроскопическое описание движения сверхтекучей жидкости, каковой является
жидкий гелий при температурах, близких к абсолютному нулю...

Мы хотели бы выразить искреннюю благодарность Я. Б. Зельдовичу и Л. И. Седову 
за ценное для нас обсуждение ряда гидродинамических вопросов. Мы благодарим 
также Д.В. Сивухина, прочитавшего книгу в рукописи и сделавшего ряд замечаний,
использованных нами при подготовке второго издания книги.


\begin{flushright}
1952 г.

Л. Ландау, Е. Лифшиц
\end{flushright}
