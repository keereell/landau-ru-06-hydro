\section{Условие отсутствия конвекции}
\label{sec:p4}

Жидкость может находиться в механическом равновесии (т.е. в ней может
отсутствовать макроскопическое движение), не находясь при этом в тепловом
равновесии. Уравнение (\ref{eq:3.1}), являющееся условием механического равновесия, может
быть удовлетворено и при непостоянной температуре в жидкости. При. этом, однако,
возникает вопрос о том, будет ли такое равновесие устойчивым. Оказывается, что
равновесие будет устойчивым лишь при выполнении определенного условия. Если это
условие не выполняется, то равновесие неустойчиво, что приводит к появлению в
жидкости беспорядочных течений, стремящихся перемешать жидкость так, чтобы в ней
установилась постоянная температура. Такое движение носит название конвекции.
Условие устойчивости механического равновесия является, другими словами,
условием отсутствия \textit{конвекции}. Оно может быть выведено следующим
образом.

Рассмотрим элемент жидкости, находящийся на высоте $z$ и обладающий удельным
объемом $V(p,s)$, где $p$ и $s$ — равновесные давление и энтропия на этой
высоте. Предположим, что этот элемент жидкости подвергается адиабатическому
смещению на малый отрезок $\xi$ вверх; его удельный объем станет при этом равным
$V(p',s)$, где $p'$ — давление на высоте $z+\xi$. Для устойчивости равновесия
необходимо (хотя, вообще говоря, и не достаточно), чтобы возникающая при этом
сила стремилась вернуть элемент в исходное положение. Это значит, что
рассматриваемый элемент должен оказаться более тяжелым, чем "вытесненная" им в
новом положении жидкость. Удельный объем последней есть $V(p',s')$, где $s'$ —
равновесная энтропия жидкости на высоте $z+\xi$. Таким образом, имеем условие
устойчивости
\[
   V(p',s') - V(p',s) > 0.
\]
Разлагая эту разность по степеням
\[
   s' - s = \frac{ds}{dz}\xi,
\]
получим:
\begin{equation}
   \label{eq:4.1}
   \ddp{V}{s}{p}\frac{ds}{dz} > 0.
\end{equation}

Согласно термодинамическим формулам имеем:
\[
   \ddp{V}{s}{p} = \frac{T}{c_p}\ddp{V}{T}{p},
\]
где $c_p$ — удельная теплоемкость при постоянном давлении. Теплоемкость $c_p$,
как и температура $T$, есть величина всегда положительная; поэтому мы можем
переписать (\ref{eq:4.1}) в виде
\begin{equation}
   \label{eq:4.2}
   \ddp{V}{T}{p}\frac{ds}{dz} > 0
\end{equation}
Большинство веществ расширяется при нагревании, т. е. $\ddp{V}{T}{p}>0$; тогда
условие отсутствия конвекции сводится к неравенству
\begin{equation}
   \label{eq:4.3}
   \frac{ds}{dz} > 0,
\end{equation}
т. е. энтропия должна возрастать с высотой.

Отсюда легко найти условие, которому должен удовлетворять градиент температуры
$\frac{dT}{dz}$. Раскрыв производную $\frac{ds}{dz}$, пишем:
\[
   \D{s}{z}=\ddp{s}{T}{p}\D{T}{z} + \ddp{s}{p}{T}\D{p}{z} =
   \frac{c_p}{T}\D{T}{z} - \ddp{V}{T}{p}\D{p}{z} > 0.
\]
Наконец, подставив согласно (\ref{eq:3.4})
\[
   \D{p}{z} = - \frac{g}{V},
\]
получим:
\begin{equation}
   \label{eq:4.4}
   - \D{T}{z} < \frac{g \beta T}{c_p},
\end{equation}
где $\beta = \frac1{V}\ddp{V}{T}{p}$ - температурный коэффициент расширения.
Если речь идет о равновесии столба газа, который можно считать идеальным (в
термодинамическом смысле слова), то $\beta T = 1$ и условие (\ref{eq:4.4}) принимает вид
\begin{equation}
   \label{eq:4.5}
   - \D{T}{z} < \frac{g}{c_p}.
\end{equation}
Конвекция наступает при нарушении этих условий, т. е. если температура падает по
направлению снизу вверх, причем ее градиент превышает по абсолютной величине
указанное в (\ref{eq:4.4} - \ref{eq:4.5}) значение.

